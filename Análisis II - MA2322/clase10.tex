\subsection{Series de potencia. Región y radio de convergencia}

\begin{defn}
    Una \ul{serie de potencias} es una serie de funciones donde consideramos polinomios. Es decir que nos dedicaremos a estudiar expresiones de este tipo:

    \[
    \sumtoinfty{n=0}{a_n(x-a)^n}
    \]
    
    \noindent donde $\{a_n\}_{n=0}^{\infty} \subset \R$ es una sucesión de números reales. Al valor $a$ le llamamos \ul{centro}, y es el lugar alrededor del cual estamos desarrollando la serie de potencias.
\end{defn}

\begin{teo}
    Dada $\sumtoinfty{n=0}{a_nx^n}$ entonces solo una de las siguientes afirmaciones es cierta:
    
    \begin{itemize}
        \item $\sumtoinfty{n=0}{a_nx^n} < \infty$, $\forall x \in \R$.
        \item $\sumtoinfty{n=0}{a_nx^n} < \infty$ unicamente en $x=0$.
        \item Existe $R > 0$ tal que $\sumtoinfty{n=0}{a_nx^n} < \infty$ en $(-R, R)$.
    \end{itemize}
    
    Al valor $R$ se le conoce como el \ul{radio de convergencia}.
\end{teo}

Antes de demostrar este teorema, necesitaremos el siguiente lema:

\begin{lem}\label{lem:pot1}
    Si una serie de potencias $\sumtoinfty{n=0}{a_nx^n}$ converge en $x = x_0$ (con $x_0 \neq 0$) entonces la serie converge absolutamente\marginfootnote{Recordemos: Una serie $\sumtoinfty{n=0}{\beta_n}$ \ul{converge absolutamente} sii $\sumtoinfty{n=0}{|\beta_n|} < \infty$.} para cada $y$ tal que $|y| < |x_0|$.
\end{lem}

\begin{proof}
    Sabemos que la serie numérica $\sumtoinfty{n=0}{a_nx_0^n}$ converge, entonces
    
    \[
    \limtoinfty{n}{a_nx^n} = 0
    \]
    
    Luego existe $M > 0$ tal que
    
    \[
    |a_nx_0^n| < M \quad \forall n \in \N
    \]
    
    Ahora, si $|y| < |x_0|$ entonces
    
    \[
    |a_ny^n| = |a_nx_0^n|\left|\frac{y}{x_0}\right|^n \leq M\left|\frac{y}{x_0}\right|^n
    \]
    
    \noindent como esto se cumple para todo $n \in \N$, y además $\left|\frac{y}{x_0}\right| < 1$ por hipótesis, esto implica que
    
    \[
    \sumtoinfty{n=0}{|a_ny^n|} \leq M\sumtoinfty{n=0}{\left|\frac{y}{x_0}\right|^n} < \infty
    \]
    
    La serie converge para cada valor de $y$ porque tenemos una serie geométrica $\sumtoinfty{n=0}{r^n}$ con $0 < r < 1$\marginfootnote{Si $|y| > |x_0|$, la serie diverge.}, y es bien sabido que este tipo de series es convergente. Así, queda demostrado.
\end{proof}

\begin{proof}[Demostración del teorema]
    Si la serie converge en $x=0$, es trivial. Igualmente si la serie converge $\forall x \in \R$.
    
    Demostremos el tercer inciso. Queremos ver que $\sumtoinfty{n=0}{a_nx^n}$ converge para algún $R > 0$ si $|x| < R$ y diverge si $|x| > R$. Sean $x_0, x_1 \in \R$ (con $x_0 \neq 0$) tales que $\sumtoinfty{n=0}{a_nx_0^n} < \infty$ y $\sumtoinfty{n=0}{|a_nx_1^n|}$ diverge. Consideremos ahora un conjunto $S$ tal que
    
    \[
    S = \left\{ x \in \R : x > 0 \wedge \sumtoinfty{n=0}{a_nx^n} < \infty \right\}
    \]
    
    Vemos que $S \subseteq \R$ y además $S \neq \emptyset$ puesto que $x_0 \in S$. Y $S$ es acotado por lo siguiente: Supongamos que no es acotado. Entonces existe $x_2 \in \R$ tal que $x_2 > |x_1|$ con
    
    \[
    \sumtoinfty{n=0}{a_nx_2^n} < \infty
    \]
    
    Pero esto es contradicción por el lema \ref{lem:pot1} ya que estamos encontrando un valor más grande que $x_1$ para el cual la serie converge. Por lo tanto, $S$ es acotado.
    
    Por el axioma de completitud, $S$ tiene un supremo. Sea $R = \sup S$. Ahora, si $|x| < R$, existe $y \in S$ tal que
    
    \[
    |x| < y < R \quad \text{tal que} \quad \sumtoinfty{n=0}{a_ny^n} < \infty \implies \sumtoinfty{n=0}{|a_nx^n|} < \infty
    \]
    
    \noindent esto por el lema \ref{lem:pot1}.
    
    En conclusión, para todo $x \in (-R, R)$ la serie de potencias converge. Análogamente, si $|x| > R$ y al mismo tiempo que la serie converge, esto contradice el hecho de que $R$ es el supremo de $S$, por lo que tenemos una contradicción. Luego si $|x| > R$ la serie de potencias diverge.
    
    Así, queda demostrado el teorema.
\end{proof}

\begin{defn}
    El conjunto $S$ definido como en el teorema anterior se conoce como \ul{dominio de convergencia}.
\end{defn}

\subsection{Criterios de convergencia}

A continuación, veremos una serie de criterios para determinar la convergencia o divergencia de estas series. Estos criterios descansan fuertemente en los criterios vistos para series numéricas.

\begin{teo}[Criterio de Cauchy]\label{teo:critCauchy}
    Sean
    
    \[
    \sumtoinfty{n=0}{a_nx^n} \quad \text{y} \quad L = \limtoinfty{n}{\left|\frac{a_{n+1}}{a_n}\right|}
    \]
    
    Entonces
    
    \begin{enumerate}
        \item Si $L = \infty$ entonces la serie converge sólo en $x=0$.
        \item Si $L = 0$ entonces la serie converge $\forall x \in \R$.
        \item Si $0 < L < \infty$ entonces la serie converge absolutamente $\forall x$ tal que $|x| < R$ con $R = 1/L$.
    \end{enumerate}
\end{teo}

\begin{proof}
    Utilicemos el criterio del cociente para series numéricas, entonces
    
    \[
    \limtoinfty{n}{\left|\frac{a_{n+1}x^{n+1}}{a_nx^n}\right|} = |x| \limtoinfty{n}{\left|\frac{a_{n+1}}{a_n}\right|} = |x|L
    \]
    
    Si $|x|L < 1$ entonces la serie converge, y si $|x|L > 1$ entonces la serie diverge. Luego
    
    \[
    |x|L < 1 \iff |x| < 1/L \quad \text{y} \quad |x|L > 1 \iff |x| > 1/L
    \]
    
    \noindent en consecuencia la serie converge si $|x| < 1/L$ y diverge si $|x| > 1/L$. Cuando tenemos $|x| = 1/L$ el criterio no decide.
\end{proof}

\begin{ejem}
    Consideremos la serie
    
    \[
    \sumtoinfty{n=1}{(-1)^{n+1} \left(\frac{2}{3}\right) \frac{x^n}{n}}
    \]
    
    Entonces, el límite queda como
    
    \[
    \limtoinfty{n}{\left|\frac{a_{n+1}x^{n+1}}{a_nx^n}\right|} = \limtoinfty{n}{\left| \left(\frac{2}{3}\right)^{n+1}\left(\frac{n}{n+1}\right)\left(\frac{3}{2}\right)^n \right||x|} = \frac{2}{3}|x|\limtoinfty{n}{\cancelto{1}{\frac{n}{n+1}}} = \frac{2|x|}{3}
    \]
    
    Luego, $\frac{2}{3}|x| < 1$ sii $|x| < 3/2$. Por lo tanto, el radio de convergencia es $3/2$.
    
    ¿Qué pasa en los extremos?:
    
    \begin{itemize}
        \item Si $x = -3/2$, entonces la serie queda como
        
        \[
        \sumtoinfty{n=1}{(-1)^{2n+1} \left(\frac{2}{3}\right) \left(\frac{3}{2}\right)^n \frac{1}{n}} = -\sumtoinfty{n=1}{\frac{1}{n}}
        \]
        
        Y esta es la serie armónica, la cual diverge.
        
        \item Si $x = 3/2$, entonces la serie queda como
        
        \[
        \sumtoinfty{n=1}{(-1)^{n+1} \left(\frac{2}{3}\right) \left(\frac{3}{2}\right)^n \frac{1}{n}} = \sumtoinfty{n=1}{\frac{(-1)^{n+1}}{n}}
        \]
        
        Y esta es la serie alternada, la cual converge condicionalmente gracias al criterio de Leibniz\marginfootnote{Recordar de análisis 1 y cálculo 1.}.
    \end{itemize}
    
    De esta forma, la región de convergencia es $I = (-3/2, 3/2]$.
\end{ejem}

\begin{teo}[Criterio de la Raíz]\label{teo:raiz}
    Sea una serie de potencias $\sumtoinfty{n=0}{a_nx^n}$ y $L = \limtoinfty{n}{\sqrt[n]{|a_n|}}$. Entonces
    
    \begin{itemize}
        \item Si $L = \infty$ entonces la serie converge solo en $x=0$.
        \item Si $L=0$ entonces la serie converge $\forall x \in \R$.
        \item Si $0 < L < \infty$ la serie converge si $|x|<R$ con $R=1/L$.
    \end{itemize}
\end{teo}

\begin{proof}
    Primero, por propiedades de la potenciación, tenemos que
    
    \[
    \limtoinfty{n}{\sqrt[n]{|a_nx^n|}} = |x|\limtoinfty{n}{|a_n|} = |x|L
    \]
    
    Ahora, aplicando el criterio de la raíz para series númericas, la serie converge si $|x|L < 1$ y diverge si $|x|L>1$. Entonces, el radio de convergencia será de la forma $R=1/L$.
\end{proof}

\begin{ejem}
    Sea la serie
    
    \[
    \sumtoinfty{n=1}{\left(\frac{n+1}{n}\right)^{n^2}x^n}
    \]
    
    Entonces aplicando el criterio de la raíz,
    
    \[
    \limtoinfty{n}{\sqrt[n]{ \left| \left(\frac{n+1}{n}\right)^{n^2}x^n\right|}} = |x|\limtoinfty{n}{\left(1 + \frac{1}{n}\right)^n} = |x|e
    \]
    
    \noindent este último resultado porque tenemos un límite notable.
    
    Entonces, la serie converge para $|x| < 1/e$ y diverge en el complemento. Falta estudiar qué pasa en los extremos:
    
    \begin{itemize}
        \item Sea $x = 1/e$: Al sustituir, queda
        
        \[
        \sumtoinfty{n=1}{\left(\frac{n+1}{n}\right)^{n^2}\left(\frac{1}{e}\right)^n}
        \]
        
        Utilicemos el criterio de comparación. Sabemos que $e$ está acotado de la siguiente manera:
        
        \[
        \left(\frac{n+1}{n}\right)^n < e < \left(\frac{n+1}{n}\right)^{n+1} \quad \footnotemark
        \]\footnotetext{Esto se ve en el curso de análisis 1.}
        
        \noindent y esto para cada $n\geq1$. Luego, esto implica que
        
        \[
        \left(\frac{n}{n+1}\right)^n > 1/e > \left(\frac{n}{n+1}\right)^{n+1} \implies \left(\frac{n}{n+1}\right)^{n^2} > 1/e^n > \left(\frac{n}{n+1}\right)^{n^2+n}
        \]
        
        Multipliquemos lo anterior por el factor $\left(\frac{n+1}{n}\right)^{n^2}$ y nos queda
        
        \[
        1 > \left(\frac{n+1}{n}\right)^{n^2} 1/e^n > \left(\frac{n}{n+1}\right)^n \implies \sumtoinfty{n=1}{\left(\frac{n+1}{n}\right)^{n^2} 1/e^n} > \sumtoinfty{n=1}{\left(\frac{n}{n+1}\right)^n}
        \]
        
        Por otro lado
        
        \[
        \limtoinfty{n}{\left(\frac{n}{n+1}\right)^n} = \limtoinfty{n}{e^{\displaystyle n\ln\left(\frac{n}{n+1}\right)}} = e^{\displaystyle \limtoinfty{n}{n\ln\left(\frac{n}{n+1}\right)}} = 1/e
        \]
        
        \noindent este resultado es fácilmente verificable por L'Hopital.
        
        Como $\limtoinfty{n}{\left(\frac{n}{n+1}\right)^n} \neq 0$, el término general de la serie numérica no tiende a cero y por lo tanto la serie diverge.
        
        En conclusión la serie $\sumtoinfty{n=1}{\left(\frac{n+1}{n}\right)^{n^2}\left(\frac{1}{e}\right)^n}$ diverge.
        
        \item Sea $x = -1/e$. Estamos mirando ahora la siguiente serie
        
        \[
        \sumtoinfty{n=1}{(-1)^n\left(\frac{n+1}{n}\right)^{n^2}\left(\frac{1}{e}\right)^n}
        \]
        
        La cual es una serie alternada, pero como el factor $\left(\frac{n+1}{n}\right)^{n^2}\left(\frac{1}{e}\right)^n$ no tiende a cero por el mismo análisis que realizamos anteriormente. Luego los términos oscilan y nuevamente la serie diverge.
    \end{itemize}
    
    En conclusión, el radio de convergencia es $R=1/e$ y la región de convergencia es $(-1/e, 1/e)$.
\end{ejem}