\section{Clase 9}
\subsection{Núcleo de Féjer}

Haciendo uso de la fórmula explícita del núcleo de Dirichlet, desarrollemos el término $K_N(t)$ que vimos al final de la clase pasada cuando hablamos de sumas Césaro:

\begin{equation*}
    \begin{aligned}
        K_N(t) &= \frac{1}{N+1} \sum_{j=0}^{N} D_j(t) \\
        &= \frac{1}{2(N+1)\sin^2(t/2)} \left( \sum_{j=0}^{N} 2\LaTeXunderbrace{\sin( (2j + 1)t/2 ) \sin(t/2)}_{\cos(jt)-\cos((j+1)t)} \right) \\
        &= \frac{1 - \cos((N+1)t)}{2(N+1)\sin^2(t/2)} \\
        &= \frac{2}{N+1}  \left[ \frac{\sin((N+1)t/2)}{2\sin(t/2)} \right]^2
    \end{aligned}
\end{equation*}

Así, tenemos la fórmula explícita del núcleo de Féjer. Tenemos varias propiedades:

\begin{aco}
    Recordemos que el núcleo de Dirichlet podemos expresarlo como

    \[
        D_j(t) = \frac{1}{2} + \sum_{m=0}^{j} \cos(mt)
    \]

    Esto va a implicar que $K_N(t)$ nos queda como

    \begin{equation*}
        \begin{aligned}
            K_N(t) &= \frac{1}{N+1} \sum_{j=0}^{j=0} \left( \frac{1}{2} \sum_{m=0}^{j} \cos(mt) \right) \\
            &= \frac{\frac{1}{2} + \left(\frac{1}{2} + \cos(t)\right) + \dots + \left(\frac{1}{2} + \cos(t) + \dots + \cos(mt)\right)}{N+1} \\
            &= \frac{1}{2} + \frac{1}{N+1} \sum_{k=1}^{n} (n+1-k)\cos(kt)
        \end{aligned}
    \end{equation*}

    Por la fórmula de Euler, sabemos que $\cos(kt) = \frac{1}{2} (e^{ikt} + e^{-ikt})$ para cada $k$. Luego

    \[
        K_n(t) = \frac{1}{2} + \sum_{k=-n}^{n} \left( 1 - \frac{|k|}{n+1} \right)e^{ikt}
    \]
\end{aco}

\begin{aco}
    Se cumplen las siguientes propiedades:

    \begin{enumerate}
        \item $K_n(t) \geq 0$ para todo $t \in [-\pi, \pi]$.
        \item $K_n(t) = K_N(-t)$.
        \item $\frac{1}{\pi} \int_{-\pi}^{\pi} K_n(t)dt = 1$.
        \item $K_n(t) \leq (n+1) / 2$ con $t \in [-\pi,\pi]$.
        \item Existe $A$ constante tal que $K_n(t) \leq \frac{A}{(n+1)t^2}$ con $t \in (0,\pi]$.
        \item Para cada $\delta > 0$ fijo y arbitrario tal que $\delta \in (0, \pi]$, se tiene que $\int_{\delta}^{\pi} K_n(t)dt \to 0$ cuando $n \to \infty$.
    \end{enumerate}
\end{aco}

Ahora, pensemos en la familia $\{ K_n(t) \}_{n=0}^{\infty}$ tal que se cumplen las propiedades del párrafo anterior. Entonces diremos que esta es una \ul{familia de aproximaciones de la identidad}. Esto usualmente se estudia de una forma general, pero únicamente nos va a interesar aplicarlo a las series de Fourier.

\begin{teo}
    Si $f \in [-\pi, \pi]$ entonces

    \[
        \lim_{n \to \infty} \theta_n f(x) = f(x)
    \]

    en cada punto de continuidad de $f$.
\end{teo}

\begin{proof}
    Sea $x_0$ un punto de continuidad de la función $f$, fijo y arbitrario. Así, dado $\eta > 0$, escogemos un $\delta > 0$ tal que

    \[
        |f(x-t) - f(x)| < \eta \quad \text{si,} \quad |t| < \delta
    \]

    Por otro lado,

    \begin{equation*}
        \begin{aligned}
            |\theta_n f(x) - f(x)| &= \left| \frac{1}{\pi} \int_{-\pi}^{\pi} f(x-t) K_n(t)dt - f(x) - \int_{-\pi}^{\pi} f(x) K_n(t)dt - f(x) \right| \\
            &\leq \frac{1}{\pi} \int_{\pi}^{\pi} |f(x-t) - f(x)| K_n(t) dt \\
            &= \frac{1}{\pi} \int_{|t|<\delta} |f(x-t) - f(x)| K_n(t) dt \\
            &+ \frac{1}{\pi} \int_{|t|>\delta} |f(x-t) - f(x)| K_n(t) dt
        \end{aligned}
    \end{equation*}

    Como $x$ es un punto de continuidad, entonces

    \[
        \frac{1}{\pi} \int_{|t|<\delta} |f(x-t) - f(x)| K_n(t) dt \leq \frac{\eta}{\pi} \int_{|t|<\delta} K_n(t)dt
    \]

    Como el núcleo es par, y por desigualdad triangular,

    \[
        \frac{1}{\pi} \int_{|t|>\delta} |f(x-t) - f(x)| K_n(t) dt \leq \frac{2|f(x)|}{\pi} \int_{\delta}^{\pi} K_n(t)dt + \frac{1}{\pi} \int_{|t|>\delta} |f(x-t)| K_n(t) dt
    \]

    Acotando cada término,

    \[
        \frac{\eta}{\pi} \int_{|t|<\delta} K_n(t)dt \leq \eta
    \]

    \[
        \frac{2|f(x)|}{\pi} \int_{\delta}^{\pi} K_n(t)dt \leq \frac{2A}{(n+1)\delta^2}
    \]

    \[
        \frac{1}{\pi} \int_{|t|>\delta} |f(x-t)| K_n(t) dt \leq \frac{A||f||}{\delta^2(n+1)}
    \]

    Finalmente, para $\eta$ también existe un $n_0 \in \N$ tal que

    \[
        |\theta_n f(x) - f(x)| < \eta
    \]

    Finalmente, hemos demostrado la convergencia puntual.
\end{proof}

Gracias a este teorema, ya vemos una característica de las sumas Césaro que no cumplen las sumas parciales.