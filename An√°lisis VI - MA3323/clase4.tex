\section{Clase 4}
\subsection{Estudio de los Coeficientes de Fourier}

En esta clase nos enfocaremos en el estudio de los Coeficientes de Fourier $\{a_n\}$, $\{b_n\}$ y $\{c_n\}$ de una función $f$ cuando estudiamos su desarrollo de Fourier. Veremos a continuación unas definiciones previas que nos serán necesarias para este estudio.

De antemano, recordemos que $Q \subset \R$ con $\mathrm{long}(Q) = 2\pi$ \footnote{Aunque podemos considerar cualquier intervalo finito, solo tomaremos en cuenta aquellos de longitud $2\pi$.}. En particular, $Q = (-\pi, \pi)$.

\begin{defn}
    Definiremos

    \[
        L^p(Q) = \left\{ f: Q \rightarrow \C \bigg| \int_Q |f(x)|^p dx < \infty \right\}  
    \]

    \noindent 
    con $1 \leq p < \infty$. Estas se conocen como las \ul{funciones p-integrables} y el caso más importante es cuando $p = 2$.
\end{defn}

\begin{defn}
    Dada $f$ continua y $2\pi$ - periódica, definiremos también \ul{el módulo de continuidad de $f$}.

    \[
        \omega (\delta, f) = \sup\{ |f(x + h) - f(x)| : |h| < \delta \wedge x \in Q \}
    \]

    Si la función $f \in L^p(Q)$, entonces el módulo de continuidad queda como

    \[
        \omega(\delta, f) = \sup_{|h| < \delta} \left( \displaystyle\frac{1}{2\pi} \int_Q |f(x+h) - f(x)|^p dx \right)
    \]

    Esto se conoce como el \ul{p-módulo de continuidad de $f$}.
\end{defn}

\begin{defn}
    Sea $\Gamma = \{x_0, \dots, x_n\} \in \Parts(Q)$ una partición de $Q$. Y consideremos por

    \[
        S_{\Gamma}(f, Q) = \sum_{i=1}^n \left|f(x_i) - f(x_{i-1})\right|
    \]

    Entonces la \ul{variación de $f$ sobre el intervalo $Q$} estará dada por

    \[
        V(f, Q) = \sup_{\substack{\Gamma \in \Parts(Q)}} (S_{\Gamma})
    \]

    Si $V(f, Q)$ es finita, entonces se dice que $f$ es de \ul{variación acotada}\footnote{En otras palabras, lo que quiere decir esto es que la función oscila de forma controlada}.
\end{defn}

\begin{defn}
    Una función continua, $2\pi$-periódica es de \ul{Lipzchitz de orden $\alpha$} con $0 < alpha \leq 1$ si $\omega(\delta, f) = O(\delta^{\alpha})$; es decir, $\exists M > 0 : f(x+h) - f(x)| \leq M|h|^{\alpha}$
\end{defn}

\begin{defn}
    $f$ es \ul{absolutamente continua} en $Q$ si dado $\varepsilon > 0$, existe $\delta > 0$ y $\Gamma = \{ x_0, \dots, x_n \}$ en $Q$ tal que

    \[
        \sum_{i=1}^n \left| f(x_i) - f(x_{i-1}) \right| < \varepsilon
    \]

    \noindent
    si $\sum_{i=1}^n |x_i - x_{i-1}| < \delta$.
\end{defn}

De esta definición podemos aseverar lo siguiente:

\begin{nota}
    Si $f$ es absolutamente continua entonces es de $f$ es de variación acotada.

    Y si $f$ es de variación finita, entonces $f \in L^1(Q)$ (es Lebesgue-integrable).
\end{nota}

\begin{teo}[Derivación Término a Término]
    Sea $f$, $2 \pi$-periódica y absolutamente continua. Si $S[f](x) = \sum_{k \in \Z} c_k e^{ikx}$ entonces

    \[
        S[f'](x) = \sum_{k \in \Z} c_k ike^{ikx} 
    \]

    Es decir, que la serie de Fourier de la derivada equivale a la derivada de la serie de Fourier.
\end{teo}

\begin{teo}[Integración Término a Término]
    Sea $f$, $2\pi$-periódica y absolutamente continua y $F$ su ingral indefinida. Si $S[f](x) = \sum_{k \in \Z} c_k e^{ikx}$ y converge uniformemente, entonces

    \[
        F(x) - c_0x 
    \]

    \noindent
    es $2\pi$-periódica y su serie de Fourier es dada por

    \[
        \alpha_0 + \sum_{k \in \Z - \{0\}} \frac{c_k}{ik} e^{ikx}
    \]
\end{teo}

%TODO: Copiar las demostraciones de ambos teoremas término a término

\subsection{Teorema de Riemann - Lebesgue}

\begin{teo}[Riemann-Lebesgue]
    Sea $f \in L'(Q)$. Consi deremos su serie de Fourier,

    \[
        S[f] = \sum_{z \in \Z} c_k e^{ik}
    \]

    Entonces los coeficientes de Fourier de $f$, denotados por $c_k$ (ó $a_k$ y $b_k$) tienden a $0$ si $|k| \to \infty$.

    Es decir, que

    \[
    \lim_{|k| \to \infty} |c_k| = 0
    \]
\end{teo}

\begin{teo}
    Si $f \in L^p(Q)$, $1 \leq p < \infty$ entonces $\omega_p (f, \delta) \to 0$ cuando $\delta \to 0$.
\end{teo}

\begin{cor}
    Si $f \in L'(Q)$, entonces

    \[
        \lim_{|k| \to \infty} c_k = 0
    \]
\end{cor}

\begin{cor}
    Si $f$ es de Lipschitz de orden $\alpha \in (0,1)$, Entonces

    \[
        |c_k|  = O(|k|^{-\alpha})
    \]
\end{cor}

%TODO: Copiar las demostraciones de estos 4 teoremas