\subsection{Teorema de Ramsey}

En primer lugar, demostraremos la versión infinita del teorema de Ramsey para pares, pero antes necesitamos establecer un par de definiciones y un resultado importante.

\begin{defn}
    Un \ul{árbol} es un conjunto parcialmente ordenado tal que para cada elemento, el conjunto de predecesores está bien ordenado.
\end{defn}

\begin{defn}
    Sea $T$ un árbol y $R \subset T$. $R$ es una \ul{rama} si
        
    \begin{itemize}
        \item $R$ está totalmente ordenado.
        \item $R$ es maximal (no se bifurca, porque en dado caso tendremos elementos que no son comparables).
    \end{itemize}
\end{defn}

\begin{teo}[König]\label{teo:arboles1}
    Un árbol infinito cuyos niveles son todos finitos tiene una rama infinita
\end{teo}

\begin{proof}
    Como el primer nivel del árbol es finito por definición, debe haber un elemento $a_0$ con una colección infinita de sucesores. Entre los sucesores inmediatos de $a_0$ (que deben ser de una cantidad finita por hipótesis), debe haber al menos un elemento con infinitos sucesores. Sea $a_1$ este elemento. Continuando con este proceso, supongamos que hemos definido $a_0, a_1, \dots, a_k$, con $a_k$ con infinitos sucesores en el árbol. Como $a_k$ tiene una cantidad finita de sucesores inmediatos, podemos tomar uno de ellos. Sea este elemento $a_{k+1}$, como el árbol es infinito podemos continuar para obtener una sucesión infinita $a_0, a_1, \dots$. Dicha sucesión es una rama del árbol por construcción.
    
    De esta manera, queda demostrado el teorema.
\end{proof}

Ahora si podemos pasar a enunciar y demostrar este caso especial del teorema de Ramsey:

\begin{teo}\label{teo:omega1}
    Sea $A$ un conjunto con $|A| = \omega$ ($A$ es un conjunto infinito numerable), y sea una coloración $f: A \rightarrow 2$ entonces
    
    \[
    \omega \rightarrow (\omega)_2^2
    \]
\end{teo}

\begin{proof}
    Como $A$ es un conjunto numerable, podemos elegir un primer elemento. Sea $a \in A$ el primer elemento del conjunto $A$. Ahora al conjunto $A$ le quitaremos $a$, y con este elemento tendremos la base de un árbol. Al resto del conjunto $A$ lo partimeros en dos: $A_0$ y $A_1$, tales que
    
    \[
    b \in A_i \iff f\left(\{a,b\}\right) = i \qquad i = \{0,1\}
    \]
    
    \noindent es posible que $A_1$ o $A_0$ sea vacío, pero no es posible que los dos sean vacíos a la vez.
    
    Ahora, tomemos el primer elemento de $A_0$ y el primer elemento de $A_1$, $a_0$ y $a_1$ respectivamente\marginfootnote{Recordemos que no estamos hablando de árboles en el sentido visto en matemáticas discretas. Aquí estamos utilizando una definición distinta.}: Ellos formarán parte del nivel 1 del árbol. Partamos en dos $A_0$ y $A_1$ de la misma manera que partimos $A$, entonces sean $A_{00}$ y $A_{01}$ tales que
    
    \[
    b \in A_{ij} \iff f\left(\{a_i,b\}\right) = j \qquad j = \{0,1\}
    \]
    
    Notemos que si tenemos algún elemento $b$ en el nivel 2, y lo únimos con $a$, tendremos que $f(\{a,b\}) = i$, porque $b \in A_i$. El mismo razonamiento aplica para todos los elementos que formen parte de niveles subsiguientes. De esta manera, iremos construyendo este árbol de forma iterativa.
    
    Entonces, el árbol que hemos construído tiene una cantidad finita de elementos en cada nivel, y como además $A$ es un conjunto infinito, por el teorema \ref{teo:arboles1}, debe existir una rama infinita. Sea $b_0, b_1, \dots, b_i, \dots$ dicha rama infinita. La propiedad que cumple esta rama es que $f(\{b_i,b_j\}) = f(\{b_i,b_k\})$ para $j > i$, $k > i$. Los elementos pueden ser de color 0 o de color 1, es decir, que tenemos una partición de tamaño 2 de un conjunto infinito, y \ref{teo:casillero} nos dice que hay un subconjunto infinito que es de tipo 1 o de tipo 0. Este conjunto es homogéneo para $f$. Esto demuestra que
    
    \[
    \omega \rightarrow (\omega)_2^2
    \]
\end{proof}

\begin{figure}
    \centering
    \begin{forest}
    before drawing tree={
    tikz+={\coordinate (a) at (current bounding box.east);},
    for nodewalk={fake=r, L, ancestors}{
      if={>O+t_+t={content}{\vdots}}{
        tikz+={\node [anchor=base west] at (.base -| a) { };}
      }{%
        tikz+/.process={Ow+Pw}{level}{#1}{\node [anchor=base west] at (.base -| a) {Nivel #1};}
        }
      }
    }
        [$a$, for tree={grow=90}, color=red
            [$a_0$, color=red
                [$a_{00}$, color=red
                    [$a_{000}$, color=red
                    [\vdots]
                    [\vdots]
                    ]
                    [$a_{001}$, color=blue
                    [\vdots]
                    [\vdots]
                    ]
                ]
                [$a_{01}$, color=blue
                    [$a_{010}$, color=red
                    [\vdots]
                    [\vdots]
                    ]
                    [$a_{011}$, color=blue
                    [\vdots]
                    [\vdots]
                    ]
                ]
            ]
            [$a_1$, color=blue
                [$a_{10}$, color=red
                    [$a_{100}$, color=red
                    [\vdots]
                    [\vdots]
                    ]
                    [$a_{101}$, color=blue
                    [\vdots]
                    [\vdots]
                    ]
                ]
                [$a_{11}$, color=blue
                    [$a_{110}$, color=red
                    [\vdots]
                    [\vdots]
                    ]
                    [$a_{111}$, color=blue
                    [\vdots]
                    [\vdots]
                    ]
                ]
            ]
        ]
    \end{forest}
    \caption{Representación gráfica del árbol construído en el teorema \ref{teo:omega1}}
    \label{fig:omega1}
\end{figure}

\vspace{5mm}

Con esto establecido, vale la pena hacernos la siguiente pregunta:

\begin{pre}\label{pre:ramsay1}
    Sea $r \in \Nastk$. ¿Podremos hacer inducción sobre $r$ para demostrar que $\omega \rightarrow (\omega)_r^2$?
\end{pre}

\begin{proof}[Respuesta]
    Para $r=1$, es obvio por el principio del casillero. Para $r=2$, considerar el teorema que acabamos de ver. Supongamos ahora que vale para $k < r$. Demostremos que esto vale para $k = r$, es decir, queremos encontrar un conjunto homogéneo para $f: A^{[2]} \rightarrow r$. Para ello, definamos una función auxiliar $g: A^{[2]} \rightarrow 2$ de la siguiente manera:
    
    \[
    g(x) =
    \begin{cases}
        0 \quad \text{si} \quad f(x) = r-1 \\
        1 \quad \text{si} \quad f(x) \neq r-1
    \end{cases}, \quad \forall x \in A^{[2]}
    \]
    
    Esta función $g$ cumple con la hipótesis inductiva porque $r_g = 2$, por lo tanto hay un $B \subset A$ infinito numerable y homogéneo para $g$. Por lo tanto $g\left(B^{[2]}\right) = 0$ o $g\left(B^{[2]}\right) = 1$. Si $g\left(B^{[2]}\right) = 0$, como $r-1 < r$, es homogéneo para $f$. Y está completa la demostración.
    
    Si $g\left(B^{[2]}\right) = 1$, entonces $B^{[2]}$ bajo $f$ alcanza menos de $k-1$ valores. Ahora, si $h: B^{[2]} \rightarrow r-1$ es la coloración que envía cada elemento de $B^{[2]}$ a su correspondiente imagen bajo $f$, se verifica la hipótesis inductiva. Luego tenemos un conjunto $H \subset B \subset A$ para el cual es homogéneo para $h$, y por lo tanto homogéneo para $f$.
    
    De esta forma, queda demostrado por inducción.
\end{proof}