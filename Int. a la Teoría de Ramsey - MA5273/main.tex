\documentclass{tufte-handout}
\usepackage[utf8]{inputenc}
\usepackage{parskip}
\usepackage{amssymb, amsthm, amsmath, fdsymbol, mathtools} % Varios paquetes para símbolos y fuentes
\usepackage{tikz, pgfplots} % Para dibujar
\usepackage{tkz-graph, tkz-berge} % Para dibujar grafos
\usepackage[linguistics]{forest} % Para dibujar árboles
\usepackage{algorithm2e} % Para escribir pseudocódigo
\usepackage{titling} % Para estilizar el título
\usepackage{scrextend} % Añade márgenes para hacer bloques de texto
\usepackage{enumitem} % Para enumerar sin sangría
\usepackage{graphicx, subcaption, wrapfig} % Para colocar figuras e imágenes
\usepackage{lastpage}
\usepackage{fancyhdr} % Para hacer encabezados y pie de página más estilizados
\usepackage{color} % Para usar colores en el texto
\usepackage{soul} % Para subrayar con colores
\usepackage{soulutf8}
\usepackage{titling} % Cambia los parámetros del título
\usepackage{hyperref}
\usepackage{booktabs} % Para hacer tablas un poco más estilizadas
\usepackage{multirow}

% Cambia el tamaño de los captions
\usepackage[font={footnotesize}]{caption}

% Establece los enviroments para teoremas, ejemplos, definiciones, etc
\newtheorem{teo}{Teorema}
\newtheorem{cor}[teo]{Corolario}
\newtheorem{lem}[teo]{Lema}
\newtheorem{pro}[teo]{Proposición}
\newtheorem{pre}{Pregunta}

\theoremstyle{definition}
\newtheorem{defn}{Definición}
\newtheorem{ejem}{Ejemplo}
\newtheorem{ejer}{Ejercicio}
\newtheorem{notn}{Notación}
\newtheorem{nota}{Nota}
\newtheorem{prob}{Problema}

% Comandos para símbolos
\newcommand{\Nastk}{\mathbb{N}^*}
\newcommand{\N}{\mathbb{N}}
\newcommand{\R}{\mathbb{R}}
\newcommand{\Z}{\mathbb{Z}}
\newcommand{\T}{\mathcal{T}}
\newcommand{\C}{\mathcal{C}}
\newcommand{\X}{\mathcal{X}}
\newcommand{\F}{\mathcal{F}}
\newcommand{\U}{\mathcal{U}}
\newcommand{\baire}{\mathcal{N}}
\newcommand{\frechet}[1]{\mathcal{FR}(#1)}
\newcommand{\partes}[1]{\mathcal{P}(#1)}
\newcommand{\met}[2]{\rho(#1, #2)}
\newcommand{\con}{\sim_{c}}
\newcommand{\rest}{\upharpoonright}
\newcommand{\pred}{\operatorname{predecesores}}
\newcommand{\FS}{\operatorname{FS}}
\newcommand{\FU}{\operatorname{FU}}
\newcommand{\UnFi}{[< \infty]}
\newcommand{\TR}{\hyperref[teo:TR]{TR}}
\newcommand{\GS}{\hyperref[teo:GS]{\textbf{GS}}}
\newcommand{\GSF}{\hyperref[teo:GSF]{\textbf{GS(fin)}}}
\newcommand{\GSC}{\hyperref[teo:GSC]{\textbf{GS(con)}}}
\newcommand{\TS}{\hyperref[teo:Schur]{TS}}
\newcommand{\TT}{\hyperref[teo:TT]{Tychonov}}

% Cambia el nombre de varios comandos
\renewcommand{\contentsname}{Contenido}
\renewcommand*{\proofname}{Demostración}
\renewcommand{\figurename}{Fig.}

% Establece las notas de margen
\newcommand{\marginfootnote}[1]{\footnotemark\footnotetext{#1}}

% Establecemos cómo será el encabezado y el pie de página
\fancyhf{}
\pagestyle{fancy}
\fancyhf{}
\fancyhead[L]{MA-5273}
\fancyhead[C]{Eduardo José Gavazut Pinto}
\fancyhead[R]{13-10524}
\fancyfoot[L]{Sección 1}
\fancyfoot[R]{Profesor: Jesús Nieto}
\fancyfoot[C]{\thepage\ de \pageref{LastPage}}
\renewcommand{\headrulewidth}{2pt} 
\renewcommand{\footrulewidth}{2pt}

% Establece los entornos para los bloques de pseudocódigo
\RestyleAlgo{ruled}
\newenvironment{algoritmo}[1][htb]
  {\renewcommand{\algorithmcfname}{Algoritmo}% Update algorithm name
   \begin{algorithm}[#1]%
  }{\end{algorithm}}

% Define la geometría del margen
\geometry{
	left=13mm, % left margin
	textwidth=130mm, % main text block
	marginparsep=8mm, % gutter between main text block and margin notes
	marginparwidth=55mm % width of margin notes
}

% Añade color a los lados de los grafos tikz
\tikzset{
    EdgeStyle/.append style = {blue}
}

% Permite agregar etiquetas a los niveles de un árbol
\forestset{%
  label tree/.style={
    for tree={tier/.option=level},
    level label/.style={
      before typesetting nodes={
        for nodewalk={current,tempcounta/.option=level,group={root,tree breadth-first},ancestors}{if={>OR={level}{tempcounta}}{before drawing tree={label me=##1}}{}},
      }
    },
    before drawing tree={
      tikz+={\coordinate (a) at (current bounding box.east);},
    },
  },
  label me/.style={tikz+={\node [anchor=base north] at (.parent |- a) {#1};}},
}

% Establece el subrayado de color rojo
\definecolor{ferrari}{rgb}{1,0.17,0}
\setulcolor{ferrari}

% Reduce el espacio entre el título y el header
\setlength{\droptitle}{-5.5em}
\renewcommand\maketitlehookc{\vspace{-3ex}}

% Define el espaciado entre párrafos
\setlength{\parskip}{1.5em}

% Definimos nuestro título
\pretitle{\begin{flushleft}\LARGE\sffamily}
\title{
Notas de Int. a la Teoría de Ramsey, Abril-Julio 2022 \\
Universidad Simón Bolívar
}
\posttitle{\par\end{flushleft}\vskip 0.5em}
\preauthor{\begin{flushleft}\large\scshape}
\author{
Eduardo Gavazut \\
Carnet: 13-10524}
\postauthor{\par\end{flushleft}}
\predate{\begin{flushleft}\large\scshape}
\date{Abril-Julio 2022}
\postdate{\par\end{flushleft}}

% Aquí empieza el documento
\begin{document}

\maketitle
\thispagestyle{fancy}

\tableofcontents
\break
\section*{Clase 1}

\subsection*{Definiciones Básicas}

\begin{defn}
    Sea $x$ una variable independiente, $y$ una variable dependiente de $x$ y $D^{(i)}$ (con $i \in \N - \{0, +\infty\}$) el operador derivación. Una relación
    
    \[
    \Phi\big( x, y, Dy, D^{(2)}y, \dots \big) = 0
    \]

    es llamada una \ul{ecuación diferencial}.

    Una ecuación diferencial se dice que es \ul{ordinaria} si el operador derivada es el operador derivada en una variable.
\end{defn}

\begin{defn}
    El orden de una EDO es el orden de la mayor derivada presente en la ecuación.
\end{defn}

\begin{defn}
    Una EDO se dice \ul{lineal} si es lineal en términos de las variables $y, y', y'', \dots, y^{(n)}, \dots$.

    En otro caso, se dice que la EDO es \ul{no lineal}.
\end{defn}

\begin{defn}
    Una solución de una EDO es una función $y = y(x)$ tal que

    \[
    \Phi\big( x, y(x), y'(x), \dots \big) = 0
    \]
\end{defn}

\subsection*{EDO de Orden 1}

\begin{defn}
    Una EDO de orden 1 es una ecuación diferencial de tipo

    \[
    \Phi\big( x, y, y' \big) = 0
    \]

    Si la EDO es lineal, entonces tiene la forma

    \[
    a_1(x)\frac{dy}{dx} + a_0(x)y = f(x), \quad a_1 \neq 0
    \]
\end{defn}

Pasaremos ahora a resolver la EDO general lineal de primer orden mediante el método de los \ul{factores integrantes}:

Consideremos la ecuación

\begin{equation}\label{eq:met_integrantes1}
    \frac{dy}{dt} + p(t)y = g(t)
\end{equation}

donde $p$ y $g$ son funciones cualesquiera. Multipliquemos esta expresión por una función $\mu(t)$ y nos queda

\begin{equation}\label{eq:met_integrantes2}
    \mu(t)\frac{dy}{dt} + \mu(t)p(t)y = \mu(t)g(t)
\end{equation}

Ahora, si consideramos $\mu$ tal que satisface

\begin{equation}\label{eq:met_integrantes3}
    \frac{d\mu(t)}{dt} = p(t)\mu(t)
\end{equation}

vemos que el lado izquierdo de \refeq{eq:met_integrantes2} es la derivada del producto $\mu(t)y$.

Si además asumimos que $\mu(t)$ es positivo entonces nos queda

\[
    \frac{d\mu(t)/dt}{\mu(t)} = p(t)
\]

La derivada del logaritmo es conocida, entonces por el TFC esto implica que

\[
    \ln\mu(t) = \int p(t)dt + k
\]

Escogiendo $k = 0$ obtenemos

\[
    \mu(t) = \exp \Bigg( \int p(t)dt \Bigg)
\]

Así, la ecuación \refeq{eq:met_integrantes2} equivale a

\[
    \frac{d}{dt}(\mu(t)y) = \mu(t)g(t)
\]

Y se sigue que

\[
    \mu(t)y = \int \mu(s)g(s) + c
\]

Y la solución general de \refeq{eq:met_integrantes1} es

\[
    y = \displaystyle \frac{\int \mu(s)g(s) + c}{\mu(t)}
\]

donde $\mu$ es el factor integrante de la ecuación.

Para aplicar este método vemos que necesitamos dos integraciones: Una para obtener $\mu$ y otra para obtener $y$.
\begin{teo}
    Sean $f: A \subseteq \R^n \rightarrow \R^m$ con $A$ abierto, conexo y convexo, y $x_0 \in A$. Si $\frac{\partial f_i}{\partial x_j}(x_0)$ existen y además $\frac{\partial f_i}{\partial x_j} \in C(A)$, entonces $f$ es diferenciable.
\end{teo}

\begin{proof}
    Por razones de simplicidad, solamente se demostrará el caso para $f: A \subset \R^n \rightarrow \R$ (el caso general se realiza tomando coordenada a coordenada). Queremos demostrar que
    
    \[
    \lim_{h \to 0} \frac{\left| f(x + h) - f(x) - \nabla f(x) \cdot h \right|}{\normaeuc{h}} = 0
    \]
    
    Evaluemos primero lo siguiente
    
    \begin{align*}
        f(x + h) - f(x) &= f(x_1 + h_1, \dots, x_n + h_n) \pm f(x_1, \dots, x_n + h_n) \\
            &\pm f(x_1, x_2, \dots, x_n + h_n) \pm \dots \pm f(x_1, x_2, \dots, x_{n-1}, x_n + h_n) \\
            &- f(x_1, \dots, x_n)
    \end{align*}
    
    Aplicando ahora el \TVM, en cada coordenada, nos queda que
    
    \[
    f(x + h) - f(x) = \frac{\partial f}{\partial x_1}(y_1)h_1 + \frac{\partial f}{\partial x_2}(y_2)h_2 + \dots + \frac{\partial f}{\partial x_n}(y_n)h_n
    \]
    
    Entonces podemos concluir que
    
    \begin{gather*}
        \big| f(x + h) - f(x) - \nabla f(x) \cdot h \big| = \Bigg| \sum_{j=1}^n \bigg( \frac{\partial f}{\partial x_j} (y_j) - \frac{\partial f}{\partial x_j} (x) \bigg) h_j \Bigg| \\
            \leq \sum_{j=1}^n \Bigg| \bigg( \frac{\partial f}{\partial x_j} (y_j) - \frac{\partial f}{\partial x_j} (x) \bigg) \Bigg| |h_j|
    \end{gather*}
    
    \noindent como $|h_j| \leq \normaeuc{h}$ (para todo $j$), entonces lo anterior queda así
    
    \begin{equation}\label{eq:2.1.1}
        \leq \left(\sum_{j=1}^n \Bigg| \bigg( \frac{\partial f}{\partial x_j} (y_j) - \frac{\partial f}{\partial x_j} (x) \bigg) \Bigg| |h_j|\right) \leq \sum_{j=1}^n \Bigg| \bigg( \frac{\partial f}{\partial x_j} (y_j) - \frac{\partial f}{\partial x_j} (x) \bigg) \Bigg| \normaeuc{h}
    \end{equation}
    
    Ahora, observemos que para cada $j = 1, \dots, n$, el teorema del valor medio garantiza la existencia de un $c_j \in (x_j, x_j + h_j)$ tal que $y_j = (x_1, \dots, c_j, \dots, x_n)$. Si $h \to 0$, entonces $h_j \to 0$ para todo $j = 1, \dots, n$. Por lo tanto $y_j \to x_j$.
    
    Como todas las derivadas parciales son continuas por hipótesis, obtenemos que
    
    \[
    \lim_{h_j \to 0} \frac{\partial f}{\partial x_j} (y_j) = \frac{\partial f}{\partial x_j} (x_j)
    \]
    
    Y por \ref{eq:2.1.1}, tenemos que si hacemos a $x_j \to 0$, nos queda que
    
    \[
    \left| f(x + h) - f(x) \right| \leq \normaeuc{h} \cancelto{0}{\sum_{j=1}^n \Bigg| \bigg( \frac{\partial f}{\partial x_j} (y_j) - \frac{\partial f}{\partial x_j} (x) \bigg) \Bigg|}
    \]
    
    Y en conclusión,
    
    \[
    \lim_{h \to 0} \frac{\left| f(x + h) - f(x) - \nabla f(x) h \right|}{\normaeuc{h}} = 0
    \]
\end{proof}

\subsection{Regla de la Cadena}
\stepcounter{subsec}
Lo que hemos hecho es probar un caso particular de un resultado que se le debe a F.P Ramsey. Enunciemos el teorema de Ramsey de la siguiente manera:

\begin{teo}[Ramsey (TR)]\label{teo:TR}
    Para todo $n \in \N$ y todo $k \in \N$
    
    \[
    \omega \rightarrow (\omega)_r^n
    \]
\end{teo}

\begin{proof}
    Procedamos igual que en la demostración anterior con un argumento inductivo: El caso $r=1$ es trivial, así que pasemos a demostrar el caso $r=2$ con inducción sobre $n$.
        
    Si $n=1$, nuevamente es trivial; si $n=2$, ya hemos hecho la demostración en \ref{pre:ramsay1}. Supongamos entonces que el teorema vale para $n$ y probemos para $\omega \rightarrow (\omega)_2^{n+1}$. Como $\N$ es un conjunto infinito numerable, estudiar sus coloraciones vale para cualquier conjunto infinito numerable, así que por conveniencia nos quedaremos estudiando $\N$. Luego, tendremos $f: \N \rightarrow 2$, y nuestro objetivo será conseguir un conjunto homogéneo para $f$.
    
    \begin{marginfigure}
        \centering
        \begin{forest}
            [$0$, for tree={grow=90}, red
                [$1$, red
                    [\vdots
                        [$n-1$, red
                            [$X_1$
                                [$Y_3$
                                    [\vdots]
                                ]
                                [$Y_2$
                                    [\vdots]
                                ]
                            ]
                            [$X_0$
                                [$Y_1$
                                    [\vdots]
                                ]
                                [$Y_0$
                                    [\vdots]
                                ]
                            ]
                        ]
                    ]
                ]
            ]
        \end{forest}
        \caption{Representación de la construcción del árbol construído en la demostración del teorema de Ramsey.}
        \label{fig:ramseyfig1}
    \end{marginfigure}
    
    Entonces, construyamos un árbol de la siguiente manera: Sabemos que los primero $n$ elementos tienen el mismo color, es decir $f(n-1) = i$, donde $i = 0$ o $i = 1$. Definamos entonces dos particiones de $\N$, $X_0$ y $X_1$ tales que
    
    \[
    t \in X_i \iff f\left( n \cup \{t\} \right) = i, \quad i \in \{0,1\}
    \]
      
    \noindent ahora, partiremos ambos conjuntos $X_0$ y $X_1$ de la siguiente manera: Sea $t_0 \in X_0$ el menor elemento de $X_0$ y consideremos $C_{t_0} = \pred(t_0) \cup \{t_0\}$, definiremos dos conjuntos $Y_0$ y $Y_1$ tales que
    
    \[
    t \in Y_i \iff f\left( x \cup \{t\} \right) = i, \quad \forall x \in C_{t_0}^{[n]}
    \]
    
    Supongamos que en este árbol que hemos estado construyendo, tenemos en el nivel $m$ a un elemento $s$, ¿cómo es la partición del conjunto al que pertenece?. Pues consideremos $C_{s} = \pred(t_0) \cup \{s\}$ y sean entonces $Z_0$, $Z_1$ dichas particiones definidas de esta manera:
    
    \[
    t \in Z_i \iff f\left( x \cup \{t\} \right) = i, \quad \forall x \in C_{s}^{[n]}
    \]
    
    Los sucesores inmediatos de $s$ son elegidos de tal forma que tomamos el menor de cada $Z_i$. Entonces cada elemento del nivel $m$ tiene a lo sumo $2^{\binom{m}{n}}$ sucesores inmediatos, ya que para cada elemento del nivel $m$, $\left| C_s^{[n]} \right| = \binom{m}{n}$ y tenemos 2 colores.
    
    Para cada uno de los pares de particiones que hemos construído, puede ocurrir que uno de ellos sea vacío, pero no que los dos sea vacío. Esto es así por la cardinalidad de $\N$.
    
    Como resultado, hemos construído inductivamente un árbol $T$ infinito donde cada nivel es finito, entonces por el teorema \ref{teo:arboles1}, existe una rama $R \subset T$ infinita. Ahora, sea $x \in R^{[n]}$, para todo $s > n$, $t > s$, por construcción tenemos que
    
    \[
    f\left(x \cup \{s\}\right) = f\left(x \cup \{t\}\right)
    \]
    
    Con esto, podemos definir una nueva partición $g: R^{[n]} \rightarrow 2$, donde
    
    \begin{gather*}
        g(x) = 0, \quad \text{si} \quad f(x) = 0 \\
        g(x) = 1, \quad \text{si} \quad f(x) = 1
    \end{gather*}
    
    Luego, recordemos que por hipótesis inductiva, $\omega \rightarrow (\omega)_2^{n}$ y $g$ cumple las condiciones para la hipótesis inductiva. De esta manera, hay un $H \subset R$ infinito tal que $H^{[n]}$ es monocromático para $g$, y por lo tanto monocromático para $f$.
    
    De esta forma, hemos demostrado el caso $r=2$. Supongamos ahora que el teorema es válido para $r \leq k$, es decir que $\omega \rightarrow (\omega)_r^n$ es válido para todo $n$. Sea ahora $f: \N^{[n]} \rightarrow k+1$, podemos definir otra partición auxiliar $G: \N^{[n]} \rightarrow 2$ definida como
    
    \begin{gather*}
        G(x) = 0, \quad \text{si} \quad f(x) = 0 \\
        G(x) = 1, \quad \text{si} \quad f(x) = 1
    \end{gather*}
    
    \noindent entonces, por hipótesis inductiva, $G$ tiene un conjunto homogéneo $H$ infinito. Si $G\left( H^{[n]} \right) = 0$, $H$ es homogéneo para $f$. Si $G\left( H^{[n]} \right) = 1$, entonces $f|H^{[n]}$ es una partición de $k$ partes, y por hipótesis inductiva existe un conjunto $H' \subseteq H$ tal que es infinito y homogéneo. Este conjunto $H'$ también es homogéneo para $f$.
\end{proof}

Con esto demostrado, podemos pasar a demostrar una consecuencia de carácter finita del teorema de Ramsey:

\begin{teo}[Teorema de Ramsey finito (TFR)]\label{teo:TRF}
    Dados números enteros positivos $n, r$ y $m$, existe un entero positivo $N$ tal que
    
    \[
    N \rightarrow (m)_r^n
    \]
\end{teo}

\marginnote{La versión que se usará durante el transcurso del curso es la verión infinita del \TR}.

\begin{proof}
    Supongamos que el teorema no se cumple, es decir que para cada $N \in \N$, existe una coloración $f_N: N^{[n]} \rightarrow k$ tal que $\forall x \in N^{[m]}$, $x^{[n]}$ \textbf{no} es homogéneo para $f_N$.
    
    Con esto, definamos una función $f: \N^{[n+1]} \rightarrow k$ tal que para cualquier sucesión de números naturales $\{a_0, a_1, \dots, a_n\}$ tenemos
    
    \[
    f\left( \{a_0, a_1, \dots, a_n\} \right) = f_{a_n} \left( \{a_0, a_1, \dots, a_{n-1}\} \right)
    \]
    
    \noindent es decir, $f$ asigna a una sucesión creciente de $n+1$ elementos, la coloración será la que le asigne $f_{a_n}$ a los primeros $n$ elementos de esa lista.
    
    Luego, el \hyperref[teo:TR]{TR} nos dice que existe un $H \subset \N$ infinito tal que $H$ es homogéneo para $f$, con $H = \{h_0, h_1, \dots \}$. De esta forma, un conjunto $h$ de $m$ elementos tal que $h \subset H$ es también homogéneo para $f_{h_m}$. Al hacer la restricción $f_N|f_{h_m}$, tenemos que $h$ es homogéneo. Esto es una contradicción, ya que habíamos supuesto que no hay homogéneos de tamaño $m$ para $f_N$.
    
    Por lo tanto, el teorema de Ramsey finito es cierto.
\end{proof}
\noindent Y a continuación, demostraremos algunas propiedades que ya conocemos de los curso de cálculo.

\begin{teo}\label{teo:intamasb}
    Sean $f, g: [a,b] \rightarrow \R$ acotadas e integrables, entonces $f + g$ es integrable.\footnotemark
\end{teo}

\begin{proof}\footnotetext{De la demostración se deducirá la propiedad de linealidad:
    
\[
\intab (f+g)(x)dx = \intab f(x)dx \intab g(x)dx
\]}
    Como $f,g \in \Rint$ entonces dado $\varepsilon > 0$, existen particiones $P_1, P_2$ (correspondientes a $f, g$ respectivamente) tales que
    
    \[
    U(f,P_1) - L(f,P_1) < \frac{\varepsilon}{2} \quad U(f,P_2) - L(f,P_2) < \frac{\varepsilon}{2}
    \]
    
    \noindent definamos $\Pe = P_1 \cup P_2$, entonces veamos cuánto resulta la diferencia $U(f+g, \Pe) - L(f+g, \Pe)$ . Sea entonces $\Pe = \{x_j\}_{j=1}^n$, con sus respectivos intervalos definidos como $(I_j)_{j=1}^n$. Sobre cada intervalo tendremos 
    
    \begin{gather}\label{eq:amasb}
        M(f+g, I_j) \leq M(f, I_j) + M(g, I_j) \\
        m(f+g, I_j) \geq m(f, I_j) + m(f, I_j) \nonumber
    \end{gather}
    
    \marginnote{En la desigualdad \ref{eq:amasb} se utiliza la siguiente propiedad de los supremos e ínfimos
    
    \begin{teo}\label{teo:supamasb}
    Sean $A, B$ acotadas, y sea el conjunto $A+B = \{a+b : a \in A, b \in B\}$, entonces
    
    \begin{gather*}
    \sup(A+B) = \sup A + \sup B \\
    \inf(A+B) = \inf A + \inf B
    \end{gather*}
    \end{teo}
    
    \begin{proof}
        Demostremos cada desigualdad por separado:
        
        \begin{itemize}
            \item[($\leq$)] Como $A, B$ están acotadas, podemos encontrar un supremo para cada conjunto, sean $\alpha, \beta$ estos supremos, respectivamente. Entonces
            
            \begin{gather*}
                \alpha \geq a \quad \beta \geq b, \quad \forall a \in A, \forall b \in B \\
                \implies \alpha + \beta \geq a + b, \forall a + b \in A + B
            \end{gather*}
            
            \noindent pero por definición, tenemos que $\sup(A+B)$ es la menor cota superior para los elementos de $A+B$, incluyendo $\alpha + \beta$. De esta forma, $\sup(A+B) \geq \sup A + \sup B$.
            
            \item[($\geq$)] Sean $\alpha = \sup A$, $\beta = \sup B$. Supongamos que $\sup(A+B) < \alpha + \beta$. Entonces, sin pérdida de generalidad, podemos encontrar un $\varepsilon > 0$ tal que
            
            \[
            \varepsilon = \alpha + \beta - \sup(A+B)
            \]
            
            \noindent luego, sean $a \in A$, $b \in B$ tales que
            
            \[
            a > \alpha - \varepsilon/2 \quad b > \beta - \varepsilon/2
            \]
            
            Entonces, podemos establecer lo siguiente
            
            \begin{gather*}
            a + b > \alpha + \beta - \varepsilon \\
            = \alpha + \beta - \big( \alpha + \beta - \sup(A+B) \big) \\
            \implies a + b > \sup(A+B)
            \end{gather*}
            
            \noindent esto es un sinsentido, por lo que tenemos una contradicción. En conclusión, $\sup(A+B) \geq \alpha + \beta$.
        \end{itemize}
        
    Esta demostración es análoga para el ínfimo.
    \end{proof}}
    
    Luego, tendremos lo siguiente para la sumas superiores e inferiores
    
    \begin{gather*}
        U(f+g, \Pe) - L(f+g, \Pe) < U(f+g, \Pe) - L(f+g, \Pe) \\
        < U(f, P_1) + U(g, P_2) - (L(f, P_1) + L(g, P_2)) \\
        = U(f, P_1) - L(f, P_2) + U(g, P_1) - L(g, P_2) \\
        < \frac{\varepsilon}{2} + \frac{\varepsilon}{2} = \varepsilon
    \end{gather*}
    
    \noindent y de esta forma, tenemos que esto implica que $U(f+g, \Pe) - L(f+g, \Pe) < \varepsilon$, por lo que se cumple la condición de Riemannn y $f+g$ es integrable. De esta forma, queda demostrado el teorema.
\end{proof}

\begin{teo}
    Si $f$ es integrable entonces $-f$ es integrable.
\end{teo}

\begin{proof}
    Por propiedades del supremo e ínfimo, tenemos lo siguiente
    
    \begin{equation}\label{eq:infmsup}
    U(-f, P) = -L(f, P) \quad L(-f, P) = -U(f, P) \quad \footnotemark
    \end{equation}
    
    Ahora, como $f \in \Rint$, dado $\varepsilon > 0$, existe $\Pe$:
    
    \[
    U(f, \Pe) - L(f, \Pe) < \varepsilon
    \]
    
    \noindent pero, tenemos que por \ref{eq:infmsup}, la desigualdad queda como\footnotetext{Acá se utiliza la siguiente propiedad:
    
    \begin{teo}
    Para $A \subset \R$ acotado, $-\inf A = \sup -A$.
    \end{teo}
    
    Demostrar esta propiedad aparece como uno de los ejercicios de la guía del curso.}
    
    \[
    U(-f, P) - L(-f, P) < \varepsilon
    \]
    
    De esta forma, se cumple la condición de Riemann para $-f$, por lo tanto $-f$ es Riemann integrable.
\end{proof}

\begin{teo}
    Sea $f: [a,b] \rightarrow \R$ acotada. Sea $\alpha \in \R$ fijo. Entonces $\alpha f \in \Rint$ si $f$ es integrable.
\end{teo}

\begin{proof}
    Esta demostración se puede dividir en dos casos, el caso $\alpha > 0$ y el caso $\alpha < 0$. Trabajaremos solamente $\alpha > 0$.
    
    Por propiedades del supremo e ínfimo, sabemos que
    
    \begin{equation}\label{eq:supinfc}
    \sup (\alpha A) = \alpha \sup A \quad \inf (\alpha A) = \alpha \inf A \quad \footnotemark
    \end{equation}\footnotetext{Esta propiedad se enuncia de la siguiente manera:
    
    \begin{teo}
        Si $\alpha > 0$, entonces $\inf (\alpha A) = \alpha\inf A$ y $\sup (\alpha A) = \alpha\sup A$.
    \end{teo}
    
    La demostración de este teorema también aparece en la guía como ejercicio.}
    
    \noindent entonces, dado $\varepsilon > 0$, podemos definir una partición $\Pe$ para la cual
    
    \[
    U(f, \Pe) - L(f, \Pe) < \frac{\varepsilon}{\alpha}
    \]
    
    \noindent y por \ref{eq:supinfc} nos queda
    
    \begin{gather*}
        U(\alpha f, \Pe) - L(\alpha f, \Pe) = \alpha U(f, \Pe) - \alpha L(f, \Pe) < \epsilon \\
        \implies U(\alpha f, \Pe) - L(\alpha f, \Pe) < \epsilon
    \end{gather*}
    
    \noindent y así, queda demostrado.
\end{proof}\marginnote{Hay que tomar en cuenta que para estas demostración no se desglosa el argumento para el supremo, y nos saltamos hablar de las sumas inferiores y superiores. Queda el caso $\alpha < 0$ como ejercicio.}

\begin{teo}\label{teo:xcua}
    Sea $f: [a,b] \rightarrow \R$ acotada, si $f \in \Rint$, entonces $f^2 \in \Rint$.
\end{teo}

Antes de demostrar este teorema, necesitaremos algunos resultados previos que estableceremos a continuación.

\begin{lem}
    Sea $P = \{x_j\}_{j=1}^n$ una partición del intervalo $[a,b]$, y consideremos los intervalos $I_j \in P$ (con $j = 1, \dots, n$). Supongamos que tenemos una función $f$ acotada, con $M(f, I_j), m(f, I_j)$ (para cada $j = 1, \dots, n$). Entonces
    
    \begin{itemize}
        \item $M(f^2, I_j) \leq M^2(|f|, I_j)$.
        \item $m(f^2, I_j) \geq m^2(|f|, I_j)$.
    \end{itemize}
\end{lem}\marginnote{Se deja la demostración del lema como ejercicio. La clave es tomar en cuenta la siguiente propiedad de los supremos:

\[
\sup_{x \in I_j} |f(x)|^2 \leq \left( \sup_{x \in I_j} |f(x)| \right)^2
\]

\noindent esta propiedad es análoga para los ínfimos:

\[
\inf_{x \in I_j} |f(x)|^2 \geq \left( \inf_{x \in I_j} |f(x)| \right)^2
\]}

\begin{proof}[Demostración del teorema \ref{teo:xcua}]
    Tenemos por hipótesis que $f$ es integrable y acotada. Como es acotada, entonces sea $M$ tal que
    
    \[
    M = \sup_{x \in [a,b]} |f(x)|
    \]
    
    \noindent y como es integrable, satisface la condición de Riemann. Luego dado $\varepsilon > 0$, existe una partición $\Pe$ tal que
    
    \[
    U(f, \Pe) - L(f, \Pe) < \frac{\varepsilon}{2M}
    \]
    
    Por otro lado, tenemos gracias al lema anterior
    
    \begin{align}\label{eq:2m}
        \begin{split}
            U(f^2, &\Pe) - L(f^2, \Pe) = \sum_{j=1}^n \left(M(f^2, I_j) - m(f^2, I_j)\right)|I_j| \\
            &\leq \sum_{j=1}^n \left( \left(M(|f|, I_j)\right)^2 - \left(m(|f|, I_j)\right)^2 \right)|I_j| \\
            &\leq \sum_{j=1}^n \left( \left(M(|f|, I_j)\right) + \left(m(|f|, I_j)\right)\left(M(|f|, I_j)\right) - \left(m(|f|, I_j)\right) \right)|I_j| \\
            &\leq 2M\left(U(|f|,\Pe) - L(|f|, \Pe)\right)
        \end{split}
    \end{align}
    
    Para dar el siguiente paso, tenemos que primero darnos cuenta de lo siguiente: La diferencia de las sumas superiores e inferiores lo que nos da como resultado es cuánto varía la gráfica de la función $f$ en el intervalo $I_j$, entonces
    
    \[
    M(f, I_j) - m(f, I_j) = \sup_{x, y \in I_j} (f(x) - f(y))
    \]
    
    esta última diferencia se conoce como \textbf{oscilación}\marginfootnote{En la sección 7.26 del Apostol se habla con más formalidad acerca de esta definición.} sobre el intervalo $I_j$. Tomando en cuenta esto y utilizando la desigualdad triangular\marginfootnote{$||f(x)| - |f(y)|| \leq |f(x) - f(y)|$}, tenemos que
    
    \begin{gather*}
    M(|f|, I_j) - m(|f|, I_j) = \sup_{x, y \in I_j} (|f(x)| - |f(y)|) \\
    \leq \sup_{x, y \in I_j} \left| |f(x)| - |f(y)| \right| \leq \sup_{x, y \in I_j} \left| f(x) - f(y) \right| \\
    = |M(f, I_j) - m(f, I_j)|
    \end{gather*}
    
    \noindent así, tenemos que $M(|f|, I_j) - m(|f|, I_j) \leq |M(f, I_j) - m(f, I_j)|$. Luego, continuando lo que dejamos en \ref{eq:2m}, nos queda
    
    \[
    2M\left(U(|f|,\Pe) - L(|f|, \Pe)\right) \leq 2M\left(U(f,\Pe) - L(f, \Pe)\right) \leq \frac{2M\varepsilon}{2M} = \varepsilon
    \]
    
    \noindent por lo tanto, podemos concluir que $U(f^2, \Pe) - L(f^2, \Pe) < \varepsilon$. Y así, queda demostrado que $f^2$ es Riemann-integrable.
\end{proof}
\subsection{Coloración de vértices de un grafo}

\begin{prob}
    Suponga que queremos organizar 6 eventos de una hora en una convención, estos eventos son $v_1, v_2, v_3, v_4, v_5, v_6$, y entre la audiencia hay gente que quiere ir al mismo tiempo a:
    
    \break
    \begin{itemize}
        \item $v_1$ y $v_2$.
        \item $v_1$ y $v_4$.
        \item $v_3$ y $v_5$.
        \item $v_2$ y $v_6$.
        \item $v_4$ y $v_5$.
        \item $v_5$ y $v_6$.
        \item $v_1$ y $v_6$.
    \end{itemize}
    
    ¿Cuántas horas harán falta para que los eventos puedan darse sin chocar entre la audiencia?
    \end{prob}
    
    \begin{proof}[Solución]
    La situación general se puede representar mediante el siguiente grafo:
    
    \begin{figure}
    \centering
    \begin{tikzpicture}
        \GraphInit[vstyle=Welsh]
        \Vertices[unit=2]{circle}{$v_3$, $v_2$, $v_1$, $v_6$, $v_5$, $v_4$}
        \Edges($v_1$,$v_2$,$v_6$,$v_1$,$v_4$,$v_5$,$v_6$,$v_5$,$v_3$)
    \end{tikzpicture}
    \caption{Cada vértice de este grafo representa un evento, y cada lado un potencial choque.}
    \label{fig:nocolor}
    \end{figure}
    
    \noindent donde los vértices representan los eventos y los lados representan los potenciales choques. El problema se resuelve al tomar los pares de vértices que \textbf{no} están conectados. Luego, una solución puede ser la siguiente:
    
    \begin{center}
        \begin{tabular}{cccc}
            \text{Hora 1} & \text{Hora 2} & \text{Hora 3} & \text{Hora 4} \\ \toprule
            $v_1$ \text{ y } $v_3$ & $v_2$ \text{ y } $v_4$ & $v_5$ & $v_6$
        \end{tabular}
    \end{center}
\end{proof}

En términos matemáticos, lo que hemos hecho es asignar una partición de cuatro partes al conjunto de vértices del grafo, con la condición de que ninguna de dichas partes contenga un par de vértices adyacentes.

A estas partes les llamamos \textbf{colores} en lugar de horas, pero la naturaleza exacta de los objetos no es importante.

\begin{defn}
    Sea $G=(V,L)$ un grafo. Una \ul{$r$-coloración} de $V$ es una función $f: V \rightarrow \{ 1, 2, \dots, r \}$ con $r \in \Nastk$ tal que $xy \in L \implies f(x) \neq f(y)$.
    
    La idea de este concepto será encontrar el mínimo $r$ tal que existe una $r$-coloración de los vértices de un grafo. Este es el \ul{número cromático}, y se representa con $\chi(G)$.
\end{defn}

De esta forma, si decidimos asignar un color a cada uno de los vértices del grafo \ref{fig:nocolor}, nos queda

\begin{figure}
    \centering
    \begin{tikzpicture}
        \GraphInit[vstyle=Welsh]
        \Vertices[unit=2]{circle}{$v_3$, $v_2$, $v_1$, $v_6$, $v_5$, $v_4$}
        \Edges($v_1$,$v_2$,$v_6$,$v_1$,$v_4$,$v_5$,$v_6$,$v_5$,$v_3$)
        \SetVertexNoLabel
        \AddVertexColor{red}{$v_1$,$v_3$}
        \AddVertexColor{blue}{$v_2$, $v_4$}
        \AddVertexColor{green}{$v_5$}
        \AddVertexColor{yellow}{$v_6$}
    \end{tikzpicture}
    \caption{El mismo grafo con la coloración planteada en la tabla.}
    \label{fig:color}
\end{figure}

\noindent sin embargo, este no es el menor número de colores que se pueden usar. Como los vértices $v_1$, $v_2$ y $v_6$ forman un triángulo, es decir, un grafo $K_3$, cada uno de ellos necesita un color distinto por ser adyacente a los otros dos, por lo que se necesitan al menos 3 colores para pintar el grafo completo.

\begin{figure}
    \centering
    \begin{tikzpicture}
        \GraphInit[vstyle=Welsh]
        \Vertices[unit=2]{circle}{$v_3$, $v_2$, $v_1$, $v_6$, $v_5$, $v_4$}
        \Edges($v_1$,$v_2$,$v_6$,$v_1$,$v_4$,$v_5$,$v_6$,$v_5$,$v_3$)
        \SetVertexNoLabel
        \AddVertexColor{red}{$v_1$}
        \AddVertexColor{blue}{$v_2$, $v_5$}
        \AddVertexColor{green}{$v_3$, $v_4$, $v_6$}
    \end{tikzpicture}
    \caption{Nuevamente el mismo grafo, pero utilizando menos colores.}
    \label{fig:color2}
\end{figure}

\begin{prob}
    ¿Cuál es el número cromático de un grafo cíclico $C_{2r}$ con un número par de vértices, y de uno $C_{2r+1}$ con un número impar de vértices?
\end{prob}

\begin{proof}[Solución]
    En el caso de los grafos cíclicos, si tenemos un ciclo par con una cantidad par de vértices, los vértices que tengan índice par tendrán un color distinto a los vecinos que tengan índice impar, pero como estos vértices se van turnando, con dos colores es suficiente para pintar el grafo. Si tenemos una cantidad impar de vértices, necesitaremos dos colores, y con ellos será suficiente para pintar $2r$ vértices, pero al agregar el vértice que falta, tendremos que este es vecino de un vértice con índice par, y de otro con vértice impar, por lo que necesitaremos un color adicional para pintar este último vértice, dándonos en total tres colores distintos.
\end{proof}

\begin{defn}
    Un grafo se dice que es \ul{bipartito} si se puede hallar una 2-partición de sus vértices tal que los vecinos de una están en la otra.
    
    Se tiene que, por definición
    
    \[
    \chi(G) = 2 \iff \text{$G$ es bipartito}
    \]
\end{defn}

Luego, si $G$ contiene un ciclo impar, entonces $\chi(G) \geq 3$. Por lo que podemos también decir que 

\[
\text{$G$ es bipartito} \implies \text{no hay ciclos impares en $G$}
\]

\begin{pre}
    ¿Si $G$ no tiene ciclos impares, entonces $\chi(G) = 2$?
\end{pre}
    
\begin{proof}[Respuesta]
    Sin pérdida de generalidad, supongamos que $G$ es conexo. Escojamos cualquier vértice y llamemoslo $v_1$ y diremos que está en el \textit{nivel 0}. Tomemos $v_2, \dots, v_r$, los vecinos de $v_1$ y diremos que están en el \textit{nivel 2}. Procediendo de esta forma, en el \textit{nivel $k$} habremos asignado todos los vértices adyacentes a los vértices del \textit{nivel $k-1$}, pero no aquellos del \textit{nivel $k-2$}.
    
    Al ordenar los vértices de esta forma, tendremos que los vértices en el \textit{nivel $k$} son adyacentes únicamente a los vértices del \textit{nivel $k-1$} y el \textit{nivel $k+1$}, y no a los del mismo nivel. Esto indica que, si tomamos dos vértices $x$ e $y$ en el mismo nivel, estos están unidos por un camino $m$ de igual longitud a cualquier vértice $z$ ubicado en algún nivel anterior, y los caminos se pueden elegir de forma tal que $z$ es el único vértice en común.
    
    Si $x$ e $y$ fuesen adyacentes, tendríamos que hay un ciclo de longitud impar $2m+1$, contrario a la hipótesis. Por lo tanto, si asignamos un color a los vértices de niveles pares, y otro a los vértices de niveles impares, tenemos $\chi(G)=2$.
    
    Y de esta forma, queda resuelto. La respuesta es afirmativa.
\end{proof}

Así, ya tenemos la demostración del teorema que enunciaremos a continuación.

\begin{teo}
    Un grafo es bipartito si y sólo si no contiene ciclos de longitud impar.
\end{teo}

\subsection{El algoritmo voraz}

Conseguir el número cromático de un grafo dado es un problema difícil. Sin embargo, si existe un método con el que construir una coloración de vértices utilizando una cantidad razonable de colores.

Este método consiste en asignar colores a los vértices en orden, de tal forma que cada vértice recibe el primer color que no haya sido asignado a uno de sus vecinos. En este algoritmo tomaremos la mejor desición que podamos en cada paso, sin tomar en cuenta si esta elección traerá problemas más adelante. Un algoritmo de este estilo se conoce como \textbf{algoritmo voraz}.

\begin{teo}
    Si $G$ es un grafo con máximo grado $k$, entonces
    
    \begin{enumerate}
        \item $\chi(G) \leq k+1$.
        \item Si $G$ es conexo y no regular, $\chi(G) \leq k$.
    \end{enumerate}
\end{teo}

\begin{proof}
    Pasemos a demostrar cada punto en orden:
    
    \begin{enumerate}
        \item Sean $v_1, \dots, v_n$ los vértices del grafo $G$. Entonces, para cualquiera de estos vértices, este tiene a lo sumo $k$ vecinos, por lo que el algoritmo voraz puede asignar a lo sumo $k$ colores a los vecinos de $v_i$. Por lo tanto, a $v_i$ se le asignará un color distinto. Y es en este sentido donde tenemos $k+1$ colores, por lo tanto $\chi(G) \leq k+1$.
        \item Como $G$ tiene grado máximo $k$ y no es regular entonces habrá al menos un vértice $v_n$ con grado menor a $k$. Entonces, tomemos en cuenta todos los vértices $v_{n-1}, v_{n-2}, \dots, v_{n-r}$ vecinos de $v_n$; hay a lo sumo $k-1$. Ahora, si consideramos los vecinos de $v_{n-1}$, tendremos a lo sumo $k-1$. Procediendo de manera sucesiva (y sin considerar los vértices que ya han sido considerados anteriormente), llegará un punto en el que consideraremos a todos los vértices del grafo $G$, ya que este está conectado. Como para cada uno de estos vértices se tienen a lo sumo $k-1$ vecinos, entonces procediendo con el algoritmo voraz tendremos a lo sumo $k-1$ colores para asignarle a cada uno. Si además contamos el color del vértice que estamos considerando, tendremos en total $k$ colores. Es en este sentido que tendremos $\chi(G) \leq k$.
    \end{enumerate}
    
    De esta forma, queda demostrado el teorema.
\end{proof}
\section{Teorema Fundamental del Cálculo y sus aplicaciones}

Ahora, desarrollaremos los resultados para establecer el Teorema Fundamental del Cálculo (TFC). Este teorema fue establecido por Leibniz y es el resultado que nos permite relacionar la teoría de derivadas con la teoría de integración. Este es el teorema central de la teoría de integración.

\begin{lem}
    Sea $f: [a,b] \rightarrow \R$ y $c \in (a,b)$. Si $f$ es integrable en $[a,c]$ y $[c,b]$ entonces $f$ es integrable en $[a,b]$.
\end{lem}

\begin{proof}
    Como la función es integrable en $[a,c]$, dado $\varepsilon > 0$, entonces existe una partición $P_1$ tal que
    
    \[
    U(f, P_1) - L(f, P_1) < \varepsilon/2
    \]
    
    \noindent análogamente, como $f$ es integrable en $[c,b]$, dado $\varepsilon > 0$, existe una partición $P_2$ tal que
    
    \[
    U(f, P_2) - L(f, P_2) < \varepsilon/2
    \]
    
    Ahora, definamos una partición $\Pe = P_1 \cup P_2$, sabemos que esta partición abarca al intervalo $[a,b]$ en su totalidad. Ahora, por \ref{eq:supamasb}, tenemos que
    
    \begin{gather*}
        U(f, \Pe) = U(f, P_1) + U(f, P_2) \\
        L(f, \Pe) = L(f, P_1) + L(f, P_2)
    \end{gather*}
    
    Así, tenemos que
    
    \[
    U(f, \Pe) - L(f, \Pe) = U(f, P_1) - L(f, P_1) + U(f, P_2) - L(f, P_2) < \varepsilon
    \]
    
    \noindent en consecuencia, la función $f$ será integrable en $[a,b]$.
\end{proof}

\begin{teo}
    Sea $f: [a,b] \rightarrow \R$ tal que $f \in \Rint$ y continua en $[a,b]$. Si $F(x) = \int_a^x f(t)dt$ para cada $x \in [a,b]$, entonces $F$\marginfootnote{Establecer esta función nos lleva a la siguiente definición:
    
    \begin{defn}
        Esta función $F$ definida de esta manera se conoce como la \ul{antiderivada} de $f$.
    \end{defn}} es continua en $[a,b]$.
\end{teo}

\begin{proof}
    Sea $\varepsilon > 0$, y los puntos $x_0, x$ en $[a,b]$ ($x > x_0$). Ahora, si queremos ver que $F$ es continua, para dicho $\varepsilon$ hemos de encontrar un $\delta$ tal que se satisfaga lo siguiente:
    
    \[
    \text{si} \quad |x - x_0| < \delta, \qquad \text{entonces} \quad |F(x) - F(x_0)| < \varepsilon
    \]
    
    Ahora, por el lema que acabamos de demostrar, y el teorema \ref{teo:riemod} tenemos
    
    \begin{align*}
        |F(x) - F(x_0)| &= \left| \int_a^x f(t)dt - \int_a^{x_0} f(t)dt \right| = \left| \int_a^x f(t)dt - \int_a^{x_0} f(t)dt \right| \\
        &= \left| \int_a^{x_0} f(t)dt + \int_{x_0}^x f(t)dt - \int_a^{x_0} f(t)dt \right| = \left| \int_{x_0}^x f(t)dt \right| \\
        &\leq \int_{x_0} |f(t)|dt
    \end{align*}
    
    \noindent como $f$ es continua y acotada en $[a,b]$, será acotada en $[x_0, x]$. Por lo tanto existe $M > 0$ tal que
    
    \[
    |F(x) - F(x_0)| \leq M\int_{x_0}^xdt \quad \text{con} \quad \sup_{x\in[a,b]} |f(x)|
    \]
    
    \noindent pero $M\int_{x_0}^xdt = M(x-x_0)$. Entonces, al escoger $\delta = \varepsilon/M$, tendremos que
    
    \[
    |F(x) - F(x_0)| \leq M(x-x_0) \leq M\delta = \epsilon
    \]
    
    De esta manera, queda demostrado.
\end{proof}

\begin{teo}[Primer Teorema Fundamental del Cálculo]\label{teo:1TFC}
    Sea una función $f: [a,b] \rightarrow \R$ con $f \in \Rint$ y continua. Entonces
    
    \[
    F(x) = \intab f(x)dx \quad \text{es derivable}
    \]
    
    Mas aún, $F'(x) = f(x)$.
\end{teo}

\begin{proof}
    En un principio, por definición,
    
    \[
    F'(x) = \lim_{x \to x_0} \frac{F(x) - F(x_0)}{x_0}
    \]
    
    \noindent es decir, que dado $\delta > 0$, queremos hallar un $\varepsilon > 0$ tal que
    
    \[
    \text{si} \quad |x - x_0| < \delta \quad \text{entonces} \left| \frac{F(x) - F(x_0)}{x-x_0} - f(x_0) \right| < \varepsilon
    \]
    
    \noindent entonces, estimemos cuánto da este último factor
    
    \[
    \left| \frac{F(x) - F(x_0)}{x-x_0} - f(x_0) \right| = \left| \frac{F(x) - F(x_0) - f(x_0)(x-x_0)}{x-x_0} \right|
    \]
    
    Por la definición de $F$, y el lema que acabamos de demostrar, tenemos
    
    \[
    \left|\dfrac{F(x) - F(x_0) - f(x_0)(x-x_0)}{x-x_0}\right| = \dfrac{\left| \int_{x_0}^x f(t)dt - \int_{x_0}^x f(x_0)dt  \right|}{|x-x_0|} \leq \dfrac{\int_{x_0}^x |f(t)dt - f(x_0)|}{|x-x_0|}
    \]
    
    Ahora, por la continuidad de $f$, dado $\epsilon > 0$, existe un $\delta_f > 0$ tal que si $|x-x_0| < \delta_f$ entonces $|f(x)-f(x_0)| < \epsilon$. Entonces
    
    \[
    \dfrac{\int_{x_0}^x |f(t)dt - f(x_0)|}{|x-x_0|} \leq \frac{\epsilon}{|x-x_0|}\int_{x_0}^xdt = \epsilon
    \]
    
    Por lo tanto, concluímos que si $|x-x_0| < \delta_f$, tenemos que
    
    \[
    \left| \frac{F(x) - F(x_0)}{x-x_0} - f(x_0) \right| < \varepsilon
    \]
    
    De esta manera, basta fijar $\delta = \delta_f$ para concluir que $F'(x_0) = f(x)$. Y así queda demostrado el teorema.
\end{proof}

\begin{teo}[Segundo Teorema Fundamental del Cálculo]\label{teo:2TFC}
    Sean $f \in C[a,b]$, $F$ tal que $F'(x) = f(x)$. Entonces
    
    \[
    \intab f(x)dx = F(b) - F(a)
    \]
\end{teo}

\begin{proof}
    Sea $G(x) = \int_a^x f(t)dt$, y por el 1TFC, obtenemos que $G'(x) = f(x)$. Pero por otro lado, también tenemos que $F'(x) = f(x)$. Como $G'(x) = F'(x)$ entonces
    
    \[
    G(x) - F(x) = k, \quad \text{con } k \in \R
    \]
    
    \noindent pero sabemos a qué equivale $G$, luego
    
    \[
    \int_a^x f(t)dt - F(x) = k \implies \cancelto{0}{\int_a^af(t)dt} - F(a) = k \implies -F(a) = k
    \]
    
    De esta manera,
    
    \begin{align*}
        \int_a^x f(t)&dt - F(x) = -F(a) \implies \intab f(t)dt - F(b) = -F(a) \\
        &\implies \intab f(t)dt = F(a) - F(b)
    \end{align*}
    
    \noindent y así queda demostrado el teorema.
\end{proof}

La primera aplicación importante que veremos de estos teoremas es una bastante usada en los cursos de cálculo:

\begin{teo}[Cambio de Variable]
    Sean $I_1, I_2 \subset \R$, y sean $f: I_1 \rightarrow I_2$ tal que $f \in C^1(I_1)$\marginfootnote{Aquí estamos manejando la siguiente notación:
    
    \begin{nota}
        Sea $f$ una función, e $I$ un intervalo cualquiera. Decir que $f \in C^1(I_1)$ equivale a pedir que la función sea continua, derivable y que su derivada sea continua sobre el intervalo $I$.
    \end{nota}}, $g: I_2 \rightarrow \R$ tal que $g \in C(I_2)$. Entonces
    
    \[
    \intab g\left(f(t)\right)f'(t)dt = \int_{f(a)}^{f(b)} g(u)du
    \]
\end{teo}

\begin{proof}
    Sea $G$ derivable en $I_2$ tal que $G'(x) = g(x)$. Luego, por 1TFC tenemos que $G(x) = \int_a^x g(u)du$, y por el 2TFC, podemos decir que
    
    \begin{equation}\label{eq:cl6.1}
        G(f(b))-G(f(a)) = \int_{f(a)}^{f(b)} g(u)du \quad \text{porque $G'(x) = g(x)$ para cada $x$}
    \end{equation}
    
    Por otro lado, la regla de la cadena nos dice que
    
    \[
    \left[ G(f(t)) \right]' = G'(f(t))f'(t)
    \]
    
    \noindent entonces, aplicando nuevamente el 2TFC,
    
    \begin{equation}\label{eq:cl6.2}
        \intab \left[ G(f(t)) \right]'dt = G(f(b)) - G(f(a))
    \end{equation}
    
    Luego, por \ref{eq:cl6.1} y \ref{eq:cl6.2} tenemos que
    
    \[
    \int_{f(a)}^{f(b)} g(u)du = \intab \left[ G(f(t)) \right]'dt = \intab g(f(t))f'(t)dt
    \]
    
    De esta forma, queda demostrado.
\end{proof}

\begin{teo}[Integración por partes]
    Sean $f, g \in C^1[a,b]$. Entonces
    
    \[
    \intab f(t)g'(t)dt = f(b)g(b) - f(a)g(a) - \intab f'(t)g(t)dt
    \]
\end{teo}

\begin{proof}
    Sabemos que
    
    \[
    [f(t)g(t)]' = f'(t)g(t) + g'(t)f(t)
    \]
    
    También sabemos por el 2TFC que
    
    \[
    \intab (f(t)g(t))'dt = f(b)g(b) - f(a)g(a)
    \]
    
    Por otro lado,
    
    \[
    \intab (f(t)g(t))'dt = \intab f'(t)g(t)dt + \intab g'(t)f(t)dt
    \]
    
    De esta forma, despejando y sustituyendo nos queda
    
    \begin{gather*}
        \intab g'(t)f(t)dt = \intab (f(t)g(t))'dt - \intab f'(t)g(t)dt \\
        \implies \intab g'(t)f(t)dt = f(b)g(b) - f(a)g(a) - \intab f'(t)g(t)dt
    \end{gather*}
    
    Así, queda demostrado el teorema.
\end{proof}
\section{Clase 7}
\subsection{Introducción a Ecuaciones diferenciales de Orden Superior}

Una EDO de orden $n$ es una ecuación de la forma

\[
    P_0(t)\frac{d^ny}{dt^n} + \dots + P_{n-1}(t)\frac{dy}{dt} + P_n(t)y = G(t)
\]

Donde las funciones $P_0, \dots, P_n, G$  son reales y continuas en algún intervalo $I: \alpha < t < \beta$ y $P_0 \neq 0$ en este intervalo. Dividir esta ecuación por $P_0(t)$ da como resultado

\begin{equation}\label{eq:edon1}
    L[y] = \frac{d^ny}{dt^n} + \dots + p_{n-1}(t)\frac{dy}{dt} + p_n(t)y = g(t)
\end{equation}

Donde $L[y]$ es un operador diferencial lineal\footnote{Un \ul{operador diferencial lineal} es una notación que ahorra escribir las ecuaciones diferenciales una y otra vez. Se define de esta manera: Para cualquier función $\phi$ que es $n$ diferenciable en un intervalo $I$, el operador diferencial $L$ es una ecuación:

\[
    L[\phi] = \phi'' +  p\phi + q\phi
\]

El valor de $L$ en un punto $t$ es

\[
    L[\phi](t) = \phi''(t) +  p(t)\phi(t) + q(t)\phi(t)
\]}

\begin{teo}
    Sean $p_0, p_1, \dots, p_n \in C(I \subset \R)$ y $x_0 \in I$. Si se tienen $y_0, y_0', \dots, y_0^{(n-1)} \in \R$ dados, entonces

    \[
        \sum_{i=0}^{n} p_i(x) y^{(n-i)} = f(x)
    \]

    \noindent admite una única solución en $I$ que satisface

    \[
        y(x_0) = y_0, \quad y'(x_0) = y_0', \quad \dots, \quad y^{(n-1)} = y_0^{(n-1)} \quad \footnotemark
    \]\footnotetext{Estas serán las $n$ conciciones iniciales de la ecuación.}
\end{teo}

Al igual que la demostración de la existencia y unicidad de la solución para EDOs de primer orden, esta demostración es bastante larga y requiere de muchos resultados previos.

\begin{defn}
    Diremos que \refeq{eq:edon1} es \ul{homogénea} si $f(x) \equiv 0$. En otro caso diremos que es \ul{no homogénea}.
\end{defn}

\begin{pro}
    Sean $y_1, \dots, y_n$ soluciones de \refeq{eq:edon1} con $f(x) \equiv 0$. Entonces

    \begin{equation}\label{eq:combinacionl}
        \sum_{i=1}^{n} c_i y_i(x) = y(x)
    \end{equation}

    \noindent es también solución de dicha EDO.
\end{pro}

\begin{proof}
    \textbf{(!!!!!!)} Leer la demostración de Daniel subida al classroom, hay un par de cosas medio chungas y que vale la pena preguntar.

    Cada $y_i(x)$ es solución de \refeq{eq:edon1}, con $f(x) = 0$. Entonces

    \begin{equation*}
        \begin{aligned}
            L[y_1] = \frac{d^ny_1}{dt^n} + \dots + p_{n-1}(t)\frac{dy_1}{dt} + p_n(t)y_1 &= 0 \\
            L[y_2] = \frac{d^ny_1}{dt^n} + \dots + p_{n-1}(t)\frac{dy_2}{dt} + p_n(t)y_2 &= 0 \\
            &\vdots \\
            L[y_n] = \frac{d^ny_n}{dt^n} + \dots + p_{n-1}(t)\frac{dy_n}{dt} + p_n(t)y_n &= 0
        \end{aligned}
    \end{equation*}

    Ahora, sea una combinación lineal de estas soluciones:

    \[
        \phi = c_1y_1 + c_2y_2 + \dots + c_ny_n
    \]

    Luego,

    \[
        L[y] = L[c_1y_1 + c_2y_2 + \dots c_ny_n] = c_1L[y_1] + c_2L[y_2] + \dots + c_nL[y_n]\footnotemark
    \]\footnotetext{Esto es así ya que los operadores diferenciales son lineales.}

    Como sabemos que $L[y_i] = 0$ (para $i=1,\dots,n$) entonces nos queda que

    \[
        L[y] = 0
    \]

    Y esto implica que $y$ es también solución.
\end{proof}

Ahora vale la pena preguntarse: ¿Toda solución de \refeq{eq:edon1} se puede escribir como una combinación lineal de $y_1, \dots, y_n$? A continuación veremos que seremos capaces de encontrar $c_1, \dots, c_n$ de tal forma que \refeq{eq:combinacionl} satisface las ecuaciones

\begin{equation*}
    \begin{aligned}
        c_1y_1(t_0) + \dots + c_ny_n(t_0) &= y_0 \\
        c_1y_1'(t_0) + \dots + c_ny_n'(t_0) &= y_0' \\
        &\vdots \\
        c_1y_1^{(n-1)}(t_0) + \dots + c_ny_n^{(n-1)}(t_0) &= y_0^{(n-1)} \\
    \end{aligned}
\end{equation*}

\begin{defn}
    El \ul{Wronskiano} de las funciones $\{y_1, \dots, y_n\} \subset C^{(n-1)}(\R)$ es el determinante

    \[
        W(y_1, \dots, y_n) =
        \begin{vmatrix}
            y_1  & y_2  & \dots & y_n \\
            y_1' & y_2' & \dots &' y_n \\
            \vdots&     &     \\
            y_1^{(n-1)} & y_2^{(n-1)} & \dots & y_n^{(n-1)} \\
        \end{vmatrix}
    \]
\end{defn}

\begin{teo}
    Sean $\{y_1, \dots, y_n\}$ soluciones de

    \begin{equation}\label{eq:homog}
        y^{(n)} + p_1(x)y^{(n-1)} + \dots + p_ny = 0
    \end{equation}

    \noindent en $I \subset \R$. Entonces $\{y_1, \dots, y_n\}$ es un conjunto L.I en $I$ sii $W(y_1, \dots, y_n) \neq 0$ para todo $x \in I$.
\end{teo}

\begin{teo}
    Si $\{y_1, \dots, y_n\}$ es un conjunto L.I de soluciones de \refeq{eq:homog}, entonces cualquier otra solución de \refeq{eq:homog} puede ser escrita como

    \[
        y(x) = c_1y_1(x) + \dots + c_ny_n(x)
    \]

    \noindent donde $c_1, \dots, c_n$ son constantes.
\end{teo}

\subsection{Solución de ecuación de orden 2 homogénea}

Si $p_0, p_1, p_2$ son constantes entonces

\[
    p_0y'' + p_1y' + p_2y = 0 \implies y'' + \tilde{p_1}y' + \tilde{p_2}y = 0
\]

\noindent y si $y(x) = e^{rx}$ es solución, entonces

\[
    p(r) = \LaTeXoverbrace{(r^2 + \tilde{p_1}r + \tilde{p_2})e^{rx}}^{\mathclap{\text{Ecuación característica de la EDO}}} = 0
\]

Varias consideraciones:

\begin{enumerate}
    \item Si $p(r) = 0$ posee dos raíces distintas y reales, $y(x) = c_1e^{r_1x} + c_2e^{r_2x}$.
    \item Si $p(r) = 0$ posee una raíz doble real, $y(x) = c_1e^{r_1x} + c_2xe^{r_2x}$.
    \item Si $p(r) = 0$ posee raíces complejas $r_1 = \alpha + i\beta$, $r_2 = \overline{r_1}$, entonces $y(x) = c_1e^{\alpha x}\cos(\beta x) + c_2e^{\alpha x}\sin(\beta x)$.
\end{enumerate}

\textbf{TAREA: Realizar el problema 4.1.20 del Boyce y realizar los problemas resueltos de la sección 3.1 y los problemas 1 al 8.}
\section{Clase 8}
\subsection{Resolución de Ecuaciones no homogéneas: Método de Variación de Parámetros}

Antes de pasar a explicar este método, hacen falta un par de teoremas\footnote{La demostración de estos dos teoremas puede encontrarse en el capítulo 3 del Boyce, y son necesarios para los resultados que se establecerán a continuación.} previos:

\begin{teo}
    Si $Y_1$, $Y_2$ son soluciones de la ecuación no homogénea \refeq{eq:edon1}, entonces la diferencia $Y_1 - Y_2$ es una solución a la ecuación homogénea asociada. Más aún, si $y_1$, $y_2$ conforman un conjunto fundamental de soluciones\footnote{Es decir, que $y_1(x)$ y $y_2(x)$ son soluciones a la ecuación y su Wronskiano es distinto de cero.} Entonces

    \[
        Y_1(x) - Y_2(x) = c_1y_1(x) + c_2y_2(x)
    \]

    \noindent con $c_1, c_2$ constantes.
\end{teo}

\begin{teo}
    La solución general de \ref{eq:edon1} puede escribirse en la forma

    \[
        y = \phi(x) = c_1y_1(x) + c_2y_2(x) + Y
    \]

    \noindent donde $y_1, y_2$ conforman un conjunto fundamental de soluciones de la ecuación homogénea asociada, $c_1, c_2$ son constantes, e $Y$ es una solución en específico de \ref{eq:edon1}.
\end{teo}

Recordemos que una ecuación diferencial no homogénea es aquella de la forma \refeq{eq:edon1}. La idea del método de Variación de Parámetros, es encontrar primero una solución a la ecuación homogénea asociada

\[
    y_c(x) = c_1y_1 + c_2y_2 = 0
\]

La idea básica es reemplazar las constantes $c_1$ y $c_2$ por funciones $u_1(x)$, $u_2(x)$ respectivamente, y hallar estas funciones de tal forma que la expresión

\[
    y = u_1y_1 + u_2y_2
\]

Es una solución de \ref{eq:edon1}.

\textbf{(!!!!) Toda esta charla que viene de $y'$ y $y''$ no la entendí muy bien, y tampoco entiendo muy bien qué es lo que se concluye. PREGUNTAR.}

Desarrollando esta idea, derivemos $y$ para hallar $u_1$ y $u_2$ en el proceso:

\begin{equation*}
    \begin{aligned}
        y' &= u_1'y_1 + u_2'y_2 + u_2y_2' + u_1y_1' \\
        y'' &= u_1''y_1 + u_1'y_1' + u_2''y_2 + u_2'y_2' + u_2'y_2' + u_2y_2'' + u_1'y_1' + u_1y_1''
    \end{aligned}
\end{equation*}

Como $y_1$, $y_2$ son soluciones de la ecuación homogénea asociada, Entonces

\[
    u_1'y_1 + u_2'y_2 = 0
\]

Y esto implica que

\begin{equation*}
    \begin{aligned}
        y' &= u_2y_2' + u_1y_1' \\
        y'' &=  u_1'y_1' + u_2'y_2' + u_2y_2'' + u_1y_1''
    \end{aligned}
\end{equation*}

Luego, gracias a los teoremas visto al inicio de la sección, sabemos que

\[
    f(x) = \LaTeXunderbrace{u_1(y_1'' + py_1' + qy_1) + u_2(y_2'' + py_2' + qy_2)}_{\text{Solución a la ecuación homogénea asociada}} + \LaTeXoverbrace{u_1'y_1' + u_2'y_2'}^{\mathclap{\text{Solución específica de la ecuación no homogénea}}}\footnotemark
\]\footnotetext{Como $y_1, y_2$ conforman un conjunto fundamental de soluciones, entonces también $y_1', y_2'$ también conforman uno.}

Como $W(y_1, y_2) \neq 0$, todo esto nos da como resultado el sistema

\begin{equation}
    \begin{cases*}
        u_1'y_1 + u_2'y_2 = 0 \\
        u_1'y_1' + u_2'y_2' = f(x)
    \end{cases*}
\end{equation}

De donde

\[
    u_1' = -\frac{y_2f(x)}{W(y_1, y_2)} \qquad u_2' = \frac{y_1f(x)}{W(y_1, y_2)}
\]

Todo esto nos quiere decir que para resolver EDO de este tipo, basta con seguir los siguientes pasos:

\begin{enumerate}
    \item Calcular $y_c$.
    \item Calcular $W(y_1, y_2)$.
    \item Aplicar $u_1', u_2'$ y calcular sus integrales.
\end{enumerate}
\section{Clase 9}
\subsection{Núcleo de Féjer}

Haciendo uso de la fórmula explícita del núcleo de Dirichlet, desarrollemos el término $K_N(t)$ que vimos al final de la clase pasada cuando hablamos de sumas Césaro:

\begin{equation*}
    \begin{aligned}
        K_N(t) &= \frac{1}{N+1} \sum_{j=0}^{N} D_j(t) \\
        &= \frac{1}{2(N+1)\sin^2(t/2)} \left( \sum_{j=0}^{N} 2\LaTeXunderbrace{\sin( (2j + 1)t/2 ) \sin(t/2)}_{\cos(jt)-\cos((j+1)t)} \right) \\
        &= \frac{1 - \cos((N+1)t)}{2(N+1)\sin^2(t/2)} \\
        &= \frac{2}{N+1}  \left[ \frac{\sin((N+1)t/2)}{2\sin(t/2)} \right]^2
    \end{aligned}
\end{equation*}

Así, tenemos la fórmula explícita del núcleo de Féjer. Tenemos varias propiedades:

\begin{aco}
    Recordemos que el núcleo de Dirichlet podemos expresarlo como

    \[
        D_j(t) = \frac{1}{2} + \sum_{m=0}^{j} \cos(mt)
    \]

    Esto va a implicar que $K_N(t)$ nos queda como

    \begin{equation*}
        \begin{aligned}
            K_N(t) &= \frac{1}{N+1} \sum_{j=0}^{j=0} \left( \frac{1}{2} \sum_{m=0}^{j} \cos(mt) \right) \\
            &= \frac{\frac{1}{2} + \left(\frac{1}{2} + \cos(t)\right) + \dots + \left(\frac{1}{2} + \cos(t) + \dots + \cos(mt)\right)}{N+1} \\
            &= \frac{1}{2} + \frac{1}{N+1} \sum_{k=1}^{n} (n+1-k)\cos(kt)
        \end{aligned}
    \end{equation*}

    Por la fórmula de Euler, sabemos que $\cos(kt) = \frac{1}{2} (e^{ikt} + e^{-ikt})$ para cada $k$. Luego

    \[
        K_n(t) = \frac{1}{2} + \sum_{k=-n}^{n} \left( 1 - \frac{|k|}{n+1} \right)e^{ikt}
    \]
\end{aco}

\begin{aco}
    Se cumplen las siguientes propiedades:

    \begin{enumerate}
        \item $K_n(t) \geq 0$ para todo $t \in [-\pi, \pi]$.
        \item $K_n(t) = K_N(-t)$.
        \item $\frac{1}{\pi} \int_{-\pi}^{\pi} K_n(t)dt = 1$.
        \item $K_n(t) \leq (n+1) / 2$ con $t \in [-\pi,\pi]$.
        \item Existe $A$ constante tal que $K_n(t) \leq \frac{A}{(n+1)t^2}$ con $t \in (0,\pi]$.
        \item Para cada $\delta > 0$ fijo y arbitrario tal que $\delta \in (0, \pi]$, se tiene que $\int_{\delta}^{\pi} K_n(t)dt \to 0$ cuando $n \to \infty$.
    \end{enumerate}
\end{aco}

Ahora, pensemos en la familia $\{ K_n(t) \}_{n=0}^{\infty}$ tal que se cumplen las propiedades del párrafo anterior. Entonces diremos que esta es una \ul{familia de aproximaciones de la identidad}. Esto usualmente se estudia de una forma general, pero únicamente nos va a interesar aplicarlo a las series de Fourier.

\begin{teo}
    Si $f \in [-\pi, \pi]$ entonces

    \[
        \lim_{n \to \infty} \theta_n f(x) = f(x)
    \]

    en cada punto de continuidad de $f$.
\end{teo}

\begin{proof}
    Sea $x_0$ un punto de continuidad de la función $f$, fijo y arbitrario. Así, dado $\eta > 0$, escogemos un $\delta > 0$ tal que

    \[
        |f(x-t) - f(x)| < \eta \quad \text{si,} \quad |t| < \delta
    \]

    Por otro lado,

    \begin{equation*}
        \begin{aligned}
            |\theta_n f(x) - f(x)| &= \left| \frac{1}{\pi} \int_{-\pi}^{\pi} f(x-t) K_n(t)dt - f(x) - \int_{-\pi}^{\pi} f(x) K_n(t)dt - f(x) \right| \\
            &\leq \frac{1}{\pi} \int_{\pi}^{\pi} |f(x-t) - f(x)| K_n(t) dt \\
            &= \frac{1}{\pi} \int_{|t|<\delta} |f(x-t) - f(x)| K_n(t) dt \\
            &+ \frac{1}{\pi} \int_{|t|>\delta} |f(x-t) - f(x)| K_n(t) dt
        \end{aligned}
    \end{equation*}

    Como $x$ es un punto de continuidad, entonces

    \[
        \frac{1}{\pi} \int_{|t|<\delta} |f(x-t) - f(x)| K_n(t) dt \leq \frac{\eta}{\pi} \int_{|t|<\delta} K_n(t)dt
    \]

    Como el núcleo es par, y por desigualdad triangular,

    \[
        \frac{1}{\pi} \int_{|t|>\delta} |f(x-t) - f(x)| K_n(t) dt \leq \frac{2|f(x)|}{\pi} \int_{\delta}^{\pi} K_n(t)dt + \frac{1}{\pi} \int_{|t|>\delta} |f(x-t)| K_n(t) dt
    \]

    Acotando cada término,

    \[
        \frac{\eta}{\pi} \int_{|t|<\delta} K_n(t)dt \leq \eta
    \]

    \[
        \frac{2|f(x)|}{\pi} \int_{\delta}^{\pi} K_n(t)dt \leq \frac{2A}{(n+1)\delta^2}
    \]

    \[
        \frac{1}{\pi} \int_{|t|>\delta} |f(x-t)| K_n(t) dt \leq \frac{A||f||}{\delta^2(n+1)}
    \]

    Finalmente, para $\eta$ también existe un $n_0 \in \N$ tal que

    \[
        |\theta_n f(x) - f(x)| < \eta
    \]

    Finalmente, hemos demostrado la convergencia puntual.
\end{proof}

Gracias a este teorema, ya vemos una característica de las sumas Césaro que no cumplen las sumas parciales.
\section{Digrafos, flujos y redes}

\subsection{Digrafos}

\begin{defn}
    Un \ul{digrafo (grafo dirigido)} es un grafo consistente de un conjunto de vértices $V$, y un subconjuto $A$ de $V \times V$, cuyos miembros se llaman \ul{arcos}. Utilizaremos la notación $D = (V,A)$ para denotar el digrafo $D$. La diferencia entre digrafos y grafos, es que los arcos son un par ordenado $(u,w)$, mientras que los lados son un par no ordenado $\{u,w\}$.
\end{defn}

Formalmente, un digrafo es simplemente otra forma de descirbir una relación entre los miembros de un mismo conjunto. En lugar de decir que $u$ está relacionado con $w$ por la relación $R$, simplemente pordemos decir que $(u,w)$ es un arco del digrafo cuyo conjunto de arcos es $R$.

\begin{marginfigure}
    \centering
    \begin{tikzpicture}
        \SetGraphUnit{2}
        \Vertex{a}
        \EA(a){b}
        \NO(b){c}
        \SO(a){d}
        \SetUpEdge[style={->,bend right}]
        \Edge(a)(b)
        \Edge(b)(c)
        \Edge(c)(a)
        \Edge(d)(a)
        \Edge(d)(b)
    \end{tikzpicture}
    \caption{Ejemplo de un digrafo con $4$ vértices y $5$ arcos.}
\end{marginfigure}

\begin{defn}
    Un \ul{paseo dirigido} en un digrafo $D = (V,A)$ es una secuencia de vpertices $v_1, \dots, v_k$ con la propiedad de que $(v_i, v_i+1)$ está en $A$, para $1 \leq i \leq k-1$. Un paseo dirigido es un \ul{camino dirigido} si todos sus vértices son distintos, y es un \ul{ciclo dirigido} si todos los vértices son distintos excepto $v_1, v_k$.
\end{defn}

\begin{defn}
    Una ilustración de estas ideas ocurre al analizar un torneo \textit{round-robin}. En esta competición, todos los competidores se enfrentan una vez, sin empate. Si $x$ le gana a $y$, tendremos el arco $(x,y)$, en caso contrario tendremos el arco $(y,x)$. Este digrafo tiene como base un grafo completo. Este tipo de digrafos se conoce como \ul{torneo}.
\end{defn}

\begin{teo}
    En cualquier torneo existe un camino dirigido que contiene todos los vértices.
\end{teo}

\begin{proof}
    La estragegia para demostrar este teorema será extender cualquier camino dirigido $y_1, \dots, y_l$ que no contiene todos los vértices. Empecemos con el arco $(y_1, y_2)$, y extendamos el camino hasta tener un camino dirigido que tenga todos los vértices.
    
    Si $l = |V|$, entonces el camino tiene todos los vértices. Supongamos que $l < |V|$ y sea $z$ cualquier vértice que no esté en el camino dirigido $y_1, \dots, y_l$. Si $(z, y_1)$ es un arco, entonces $z, y_1 \dots, y_l$ es una extensión del camino inicial. Si $(z, y_1)$ no es un arco, entonces como tenemos un torneo necesariamente existirá el arco $(y_1, z)$. Sea $r$ el entero más grande para el cual $(y_1, z), (y_2, z), \dots, (y_r, z)$ son arcos. Si $r=l$, entonces $y_1, \dots, y_l, z$ es una extendión del camino inicial. Si $r<l$ entonces $(y_r, z)$ y $(z, y_{r+1})$ son arcos, y $y_1, \dots, y_r, z, y_{r+1}, \dots y_l$ es una extensión del camino inicial.
    
    En cualquier caso, hemos encontrado una extensión al camino inicial. Por lo que podemos construir de forma inductiva un camino que contenga a todos los vértices
\end{proof}
\subsection{Caminos críticos}

Abrimos este tema con el siguiente problema:

En la siguiente tabla se muestran los días que se requieren para terminar las actividades $\alpha_1, \dots, \alpha_n$ necesarias para la elaboración de un proyecto. Algunas actividades tienen como requisito la culminación de otras. Esto también se muestra en la tabla. ¿Cuál es la mínima cantidad de días que se requieren para completar todas las actividades del proyecto?

\begin{center}
    \begin{tabular}{c|cccccccc}
        \text{Actividad}         & $\alpha_1$ & $\alpha_2$ & $\alpha_3$ & $\alpha_4$ & $\alpha_5$ & $\alpha_6$ & $\alpha_7$ & $\alpha_8$ \\
        \text{Tiempo necesitado} & $4$ & $3$ & $7$ & $4$ & $6$ & $5$ & $2$ & $5$ \\
        \text{Prerequisito}      & $-$ & $-$ & $\alpha_1$ & $\alpha_1$ & $\alpha_2$ & $\alpha_4, \alpha_5$ & $\alpha_3, \alpha_6$ & $\alpha_4, \alpha_5$
    \end{tabular}
\end{center}

\begin{proof}
    Esta situación puede ser modelada con un digrafo ponderado en el cual los lados son las actividades, el vértice $s$ es el inicio del proyecto, $t$ el final y el resto como en la siguiente tabla:
    
    \begin{center}
        \begin{tabular}{c|cccccccc}
            \text{Actividad} & $\alpha_1$ & $\alpha_2$ & $\alpha_3$ & $\alpha_4$ & $\alpha_5$ & $\alpha_6$ & $\alpha_7$ & $\alpha_8$ \\
            \text{Arco} & $(s,r)$ & $(s,p)$ & $(r,z)$ & $(r,q)$ & $(p,q)$ & $(q,z)$ & $(z,t)$ & $(q,t)$
        \end{tabular}
    \end{center}
    
    Luego tenemos el siguiente digrafo:
    
    \begin{figure}
        \centering
        \begin{tikzpicture}
            \SetGraphUnit{2}
            \Vertex{s}
            \NOEA(s){r}
            \SOEA(s){p}
            \EA(r){z}
            \EA(p){q}
            \NOEA(q){t}
            \SetUpEdge[style={->}]
            \tikzset{LabelStyle/.style = {fill=white}}
            \Edge[label=$4$](s)(r)
            \Edge[label=$7$](r)(z)
            \Edge[label=$4$](r)(q)
            \Edge[label=$2$](z)(t)
            \Edge[label=$3$](s)(p)
            \Edge[label=$6$](p)(q)
            \Edge[label=$5$](q)(z)
            \Edge[label=$5$](q)(t)
        \end{tikzpicture}
        \caption{Modelado del problema.}
    \end{figure}
    
    El vértice $s$ es la cola de los lados correspondientes a $\alpha_1$ y $\alpha_2$. A partir de allí, la cola de cada actividad es la cabeza de las actividades requisitos. Ahora, hagamos lo siguiente:
    
    \begin{itemize}
        \item Definamos $T(s) = 0$.
        \item Para cada vértice $v$, definimos $T(v)$ como el tiempo necesario para completar todas las actividades en un camino de $s$ a $v$.
    \end{itemize}
    
    Los valores para $T$ están en la siguiente tabla:
    
    \begin{center}
        \begin{tabular}{c|cccccc}
        $v$    & $s$ & $p$ & $q$ & $r$ & $z$  & $t$ \\
        $T(v)$ & $0$ & $3$ & $9$ & $4$ & $14$ & $16$
        \end{tabular}
    \end{center}
    
    De esta forma, $16$ es el mínimo de días para completar todas las actividades del proyecto.
\end{proof}

Lo que vimos anteriormente es un ejemplo de \ul{búsqueda del camino más largo} en una red modelada con un digrafo ponderado (con peso $p$). La búsqueda es en anchura (BEA) y comenzamos por calcular recursivamente la función $T$ dada por:

\[
T(s) = 0 \quad \text{y} \quad T(v) = \max\left\{T(v) + p((u,v)): u \rightarrow v\right\}
\]

Y definimos una función $L(v)$ como el tiempo límite para el que todas las actividades con cola $v$ deben empezarse para que el proyecto se complete a tiempo. Esta función se calcula de manera recursiva, esta vez hacia atrás:

\[
L(t) = T(t) \quad \text{y} \quad L(v) = \min\left\{L(x) - p((v,x)) : v \rightarrow x\right\}
\]

Fijémonos en que para una actividad dada por el lado $(y,z)$,

\begin{itemize}
    \item No puede comenzar a realizarse antes de $T(y)$.
    \item Se completa antes de $L(z)$.
    \item Utiliza un tiempo $p(y,z)$.
\end{itemize}

Ahora definamos una función \ul{tiempo flotante} como

\[
F = F((y,z)) = L(x) - T(y) - p((y,z))
\]

\noindent entonces la actividad dada por $(y,z)$ puede comenzar a partir de $T(y)$ y antes de $F((y,z))$ sin retrasar el proyecto. Si la actividad $(y,z)$ tiene tiempo flotante cero (es decir el proyecto se completa a tiempo) se dice que es \ul{crítica}. Un camino cuyos lados corresponden sólo a actividades críticas se llama \ul{camino crítico}.

\subsection{Flujos en redes}

A partir de ahora vamos a considerar los arcos de una \ul{red} como "tuberías" por las cuales fluye algún servicio (sea agua, electricidad, gas, etc.). Los pesos asginados a cada uno de ellos serán las capacidades, y estas otorgan límites a las cantidades que pueden fluir en los arcos. Adicionalmente siempre tendremos un vértice $s$ con la propiedad que todo arco que contiene a $s$ tiene a $s$ como cola, y otro vértice $t$ con la propiedad de que todo arco que contiene a $t$ tiene a $t$ como cabeza. A ellos les llamaremos \ul{fuente} y \ul{sumidero}, respectivamente. En resumen, las redes que estudiaremos comprenden

\begin{itemize}
    \item Un digrafo $D=(V,A)$.
    \item Una función de capacidad $c: A \rightarrow \N$.
    \item Una fuente $s$ y un sumidero $t$.
\end{itemize}

Otra cosa que se toma en consideración es que para cada vértice, la cantidad de suministro que \textit{entra} es igual a la cantidad que \textit{sale}. ENtonces si $f(x,y)$ es la cantidad de suministro que fluye a través del lado $(x,y)$, definimos

\[
\ent(v) = \sum_{(x,v) \in A} f(x,v) \quad \text{y} \quad \sal(v) = \sum_{(v,y) \in A} f(x,y)
\]

\noindent y se pide que ambas cantidades sean iguales excepto para $s$ y $t$. Esta es la \ul{regla de conservación} para flujos en redes.

Adicionalmente, también tenemos la \ul{regla de factibilidad}, en la que $f(x,y) \leq c(x,y)$. Con todo esto podemos pasar a definir un flujo.

\begin{figure}
    \centering
    \begin{tikzpicture}
        \SetGraphUnit{3}
        \Vertex{s}
        \NOEA(s){a}
        \EA(s){b}
        \SOEA(s){c}
        \EA(b){d}
        \EA(d){t}
        \SetUpEdge[style={->}]
        \tikzset{LabelStyle/.style = {fill=white}}
        \Edge[label=$5$](s)(a)
        \Edge[label=$4$](s)(b)
        \Edge[label=$3$](s)(c)
        \Edge[label=$2$](b)(d)
        \Edge[label=$6$](a)(d)
        \Edge[label=$7$](c)(d)
        \Edge[label=$4$](d)(t)
        \Edge[label=$3$](a)(t)
        \Edge[label=$5$](c)(t)
    \end{tikzpicture}
    \caption{Ejemplo de una red.}
    \label{fig:red}
\end{figure}

\begin{defn}
    Sea una red con el digrafo asociado $D=(V,A)$, función de capacidad $c: A \rightarrow \N$ y fuente $s$ y sumidero $t$. Un \ul{flujo} desde la fuente $s$ al sumidero $t$ en una red es una función que asigna un número no negativo $f(x,y)$ a cada arco $(x,y)$, el cual cumple con lo siguiente:
    
    \begin{itemize}
        \item Conservación: $\ent(v) = \sal(v)$, con $v \neq s,t$.
        \item Factibilidad: $f(x,y) \leq c(x,y)$ para todo $(x,y) \in A$.
    \end{itemize}
\end{defn}

\begin{ejem}
    En la siguiente tabla se presenta un flujo para la red presentada en \ref{fig:red}.
    
    \begin{tabular}{c|ccccccccc}
        $(x,y)$  & $(s,a)$ & $(s,b)$ & $(s,c)$ & $(a,d)$ & $(b,d)$ & $(c,d)$ & $(a,t)$ & $(c,t)$ & $(d,t)$ \\
        $f(x,y)$ & $3$ & $2$ & $3$ & $1$ & $2$ & $1$ & $2$ & $2$ & $4$
    \end{tabular}
\end{ejem}

Se observa en el ejemplo que $\ent(t) = \sal(t)$. El valor común de estas dos cantidades mide la cantidad total que fluye en la red, y se llama el \ul{valor} del flujo, escrito como $\val(f)$. Para el ejemplo dado tenemos

\begin{gather*}
    \sal(s) = f(s,a) + f(s,b) + f(s,c) = 8 \\
    \ent(t) = f(a,t) + f(d,t) + f(c,t) = 8
\end{gather*}

\noindent entonces $\val(f) = 8$.

\begin{pre}
    Ahora nos preguntamos: ¿Cuál es el máximo valor de un flujo para una red dada?
\end{pre}

\begin{proof}[Respuesta]
    Para contestar la pregunta, primero consigamos una cota superior para este valor en términos de las capacidades. En \ref{fig:red} tenemos que la capacidad total de los arcos que salen de $s$ es 12, así que es claro que ningún flujo puede tener un valor mayor a 12. Más aún, si tenemos una partición en dos partes de $V$ tal que una parte $S$ contiene a $s$ y otra $T$ contiene a $t$. Entonces el flujo neto de $S$ a $T$ es por la regla de conservación el mismo que hay de $s$ a $t$, que coincide con el valor de $f$. Es decir
    
    \[
    \val(f) = \sum_{\substack{x\in S \\ y \in T}} f(x,y) - \sum_{\substack{u\in T \\ v \in S}} f(u,v)
    \]
    
    La primera suma mide el flujo total de $S$ a $T$, y la segunda en sentido contrario. Como solamente tenemos términos positivos,
    
    \[
    \val(f) \leq \sum_{\substack{x\in S \\ y \in T}} f(x,y)
    \]
    
    Más aún, $f(x,y) \leq c(x,y)$ para todos los arcos, y concluimos que
    
    \[
    \val(f) \leq \sum_{\substack{x\in S \\ y \in T}} c(x,y) = c(S,T)
    \]
\end{proof}

\begin{defn}
    Formalmente, definimos $(S,T)$ como un \ul{corte} (que separa a $s$ y $t$) si $S \cup T$ es una partición de $V$ tal que $s \in S$ y $t \in T$. La \ul{capacidad} del corte es
    
    \[
    c(S,T) = \sum_{\substack{x\in S \\ y \in T}} c(x,y)
    \]
\end{defn}

De forma más concisa, hemos demostrado el siguiente resultado:

\begin{teo}\label{teo:cotavalor}
    Sean $s$, $t$ la fuente y sumidero de una red. Si $f$ es cualquier flujo de $s$ a $t$ y $(S,T)$ es un corte, entonces
    
    \[
    \val(f) \leq c(S,T)
    \]
\end{teo}
\section{Ultrafiltros}

Trataremos ahora la idea de \textit{conjuntos "grandes"} en $\N$. Diremos que un conjunto $X$ es grande si para toda sucesión $\{x_n\}_{n \in \N}$ convergente a un punto $x$, tenemos que para toda vecindad $V$ de $x$, la cantidad de elementos que están fuera de esta vecindad es finita, es decir que

\[
\left| \N - \{ n \in \N : x_n \in V \} \right| < \infty
\]

En un contexto general, sea $S$ un conjunto infinito, la \ul{familia de Fréchet} está definida como

\[
\frechet{S} = \{A \subseteq S : |S - A| < \infty\}
\]

Entonces los conjuntos que pertenecen a esta familia pueden considerarse como conjuntos "grandes".

Esta familia tiene las siguientes características\marginfootnote{Verificar estas características es trivial.}:

\begin{enumerate}
    \item $S \in \frechet{S}$.
    \item Si $A \in \frechet{S}$ y $A \subseteq B$. Entonces $B \in \frechet{S}$.
    \item Si tenemos $A \in \frechet{S}$ y $B \in \frechet{S}$. Entonces $A \cap B \in \frechet{S}$.
\end{enumerate}

Dicha lista de características nos da pie a la siguiente definición:

\begin{defn}
    Sean $S$ un conjunto no-vacío, y una colección $\F \subseteq \partes{S}$ tal que satisface las condiciones expuestas anteriormente. Entonces decimos que $\F$ es un \ul{filtro} sobre $S$\marginfootnote{Nos interesan solamente los filtros sobre los números naturales, pero habrán conceptos que serán definidos en general.}.
\end{defn}

\begin{ejem}
    Consideraremos un par de ejemplos a continuación:
    
    \begin{enumerate}
        \item Si $\emptyset \in \F$, entonces $\F = \partes{S}$ y en ese caso decimos que el filtro es \textit{trivial}.
        \item $\frechet{S}$ es un filtro.
        \item Sea $S$ un conjunto no vacío y $A \subseteq S$. La familia
        
        \[
        \F_{A} = \{ B \subseteq S : A \subseteq B \}
        \]
        \noindent es un filtro y se llama \textit{filtro generado} por $A$.
        
        En el caso particular en el que tengamos un filtro generado por $A$, donde $A$ es de la forma $A = \{a\}$. Entonces podemos fijarnos de algo muy especial: Dado $X \subseteq S$, entonces
        
        \[
            a \in X \iff a \notin S - X \implies X \in \F_a \iff S - X \notin \F_a
        \]
        
        \noindent estos filtros definidos de esa manera se llaman \textit{principales}.
    \end{enumerate}
\end{ejem}

Estas características de los filtros principales llevadas a un contexto general nos llevan a la siguiente definición:

\begin{defn}
    Sea $\F$ un filtro sobre $S$. Si para todo $X \subseteq S$, se tiene que
    
    \[
    X \in \F \iff S - X \notin \F
    \]
    
    \noindent entonces se dice que $\F$ es un \ul{ultrafiltro}.
\end{defn}

Fijemos los siguientes hechos\marginfootnote{Demostrarlos queda como ejercicio.}:

\begin{enumerate}
    \item Sea $\U$ un ultrafiltro sobre $S$. Entonces
    
    \begin{enumerate}
        \item Sean $X_1, \dots, X_n$ subconjuntos $S$ tales que
        
        \[
        \bigcup_{j=1}^n X_j \in \U \quad \text{y} \quad (i \neq j) \implies X_i \cap X_j \notin \U
        \]
        
        Entonces uno y sólo uno de los $X_k$'s está en $\U$.
        
        \item Sea $X \in \U$ con $|X| \geq 2$. Entonces existe $Y \subset X$ tal que $Y \in \U$.
        
        \item Si existe $F \in \U$ con $|F| < \infty$. Entonces $\U$ es principal.
    \end{enumerate}
    
    \item Sea $\F$ un filtro sobre $S$. Entonces $\F$ es un ultrafiltro sii $\F$ es un filtro maximal\marginfootnote{Es decir, no está contenido en ningún otro filtro que no sea el trivial.}.
\end{enumerate}

Ahora, pasemos a demostrar el siguiente teorema:

\begin{teo}
    Existe un ultrafiltro no principal sobre $\N$.
\end{teo}

\begin{proof}
    Sea ahora $\frechet{\N}$, por el lema de Zorn, podemos encontrar un filtro maximal no trivial $\U$ tal que $\U \supseteq \frechet{\N}$. Luego, $\U$ es no principal: Supongamos que $\{a\} \subset \U$, como $\U$ contiene al filtro de Fréchet, tenemos que $\N - \{a\} \in \U$. Pero esta es una contradicción: $\U$ no puede contener a la vez a $\{a\}$ y a su complemento porque $\U$ es un ultrafiltro. Luego $\{a\} \notin \U$ y $\U$ es no principal.
\end{proof}

\begin{notn}
    Fijemos un ultrafiltro no principal $\U$ en $\N$. Para cada $k$ definimos un ultrafiltro no principal en $\N^{[k]}$ como sigue: para $X \subseteq \N^{[k]}$, $X \in \U^k$ si y sólo si
    
    \[
    \{ n_0 : \{ n_1 : \dots \{ n_{k-1} : \{ n_0, n_1, \dots, n_{k-1} \} \in X \} \in \U \} \dots \in \U \} \in \U
    \]
\end{notn}

Con esta notación y lo que hemos establecido sobre los ultrafiltros, podemos pasar a demostrar el teorema de Ramsey.

\begin{teo}[Ramsey]
    Si $\U$ es un ultrafiltro no principal sobre $\N$, entonces $\U^2$ es un ultrafiltro no principal sobre $\N^{[2]}$
\end{teo}

\begin{proof}
    
\end{proof}

\end{document}