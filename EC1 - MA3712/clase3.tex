\section{Clase 3}
\subsection{Ecuaciones de Variables Separables}

Son de la forma

\[
    \begin{cases*}
        \frac{\partial y}{\partial x} = f(x, y) = F(x)G(y) \\
        y(x_0) = y_0
    \end{cases*}
\]

Donde se tiene que:

\begin{enumerate}
    \item $f(x, y) \in C\big( D \subset \R^2 \big) \iff F, G \in C\big( D \subset \R^2 \big)$.
    \item $f$ es Lipschitz en segunda variable.
\end{enumerate}

En particular, si tenemos

\[
    \frac{\partial y}{\partial x} = f(a + bx + cy) \iff \begin{cases*}
                                            z = a + bx + cy \\
                                            z' = b + cy'
                                        \end{cases*}
\]

Luego

\[
    \frac{z' - b}{c} = f(z) \implies z' = cf(z) + b = F(z)G(z)
\]

donde $F(z) = cf(z) + b$ y $G(z) = 1$.


% TODO: Copiar el desarrollo de estos ejemplos
\begin{ejem}
    Resolver los siguientes PVI:

    \begin{enumerate}
        \item $y' = \sin(x-y)$.
        \item Ley de Calor de Newton: Siendo $T_0$ la temperatura ambiente, queremos resolver lo siguiente:
        
        \[
            \begin{cases*}
                \frac{dT}{dt} = -k(T - T_0) \\
                T(t_0) = T_1
            \end{cases*}  
        \]
    \end{enumerate}
\end{ejem}

\subsection{Ecuaciones Homogéneas}

\begin{defn}
    Sea $f: \R^2 \rightarrow \R$ tal que para cada $\alpha \in \R$ se tiene

    \[
        f(\alpha x, \alpha y) = \alpha^n f(x, y), \quad n \in \N \cup \{0\}
    \]

    Tal $f$ es llamada \ul{homogénea de grado $n$}. Si $f(\alpha x, \alpha y) = f(x, y)$ entonces $f$ es llamada simplemente \ul{homogénea}.
\end{defn}

% TODO: Copiar el ejemplo de esta proposición
\begin{pro}
    Dada

    \[
        \frac{\partial y}{\partial x} = f(x,y)
    \]

    con $f$ homogéneas, entonces $f(x, y) = \Psi (y/x) $. Además si $u = y/x$,

    \[
        \frac{\partial y}{\partial x} = \Psi \left( \frac{y}{x} \right) \implies \LaTeXunderbrace{x\frac{du}{\partial x} = \Psi(u) - u}_{\text{\textbf{Variables Separables}}}
    \]
\end{pro}

\subsection{Ecuaciones Diferenciales Exactas}

\begin{defn}
    Sean $M, N \in C^1(D \in \R^2)$ tales que

    \begin{equation}\label{eq:exacta}
        M \partial x + N \partial y = 0
    \end{equation}

    Diremos que \ref{eq:exacta} es exacta si y solamente si

    \[
        \displaystyle\frac{\partial M}{\partial y} = \frac{\partial N}{\partial x}
    \]
\end{defn}

\begin{teo}
    Como $\R^2$ es conexo y convexo, $\frac{\partial M}{\partial y} = \frac{\partial N}{\partial x}$ es equivalente a decir que existe $f: \R^2 \rightarrow \R$ tal que $\nabla f(x,y) = (M, N)$. Esto es, toda solución de \ref{eq:exacta} es un campo potencial.
\end{teo}

%TODO: Copiar el ejemplo de este teorema