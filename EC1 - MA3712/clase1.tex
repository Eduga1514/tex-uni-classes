\section*{Clase 1}

\subsection*{Definiciones Básicas}

\begin{defn}
    Sea $x$ una variable independiente, $y$ una variable dependiente de $x$ y $D^{(i)}$ (con $i \in \N - \{0, +\infty\}$) el operador derivación. Una relación
    
    \[
    \Phi\big( x, y, Dy, D^{(2)}y, \dots \big) = 0
    \]

    es llamada una \ul{ecuación diferencial}.

    Una ecuación diferencial se dice que es \ul{ordinaria} si el operador derivada es el operador derivada en una variable.
\end{defn}

\begin{defn}
    El orden de una EDO es el orden de la mayor derivada presente en la ecuación.
\end{defn}

\begin{defn}
    Una EDO se dice \ul{lineal} si es lineal en términos de las variables $y, y', y'', \dots, y^{(n)}, \dots$.

    En otro caso, se dice que la EDO es \ul{no lineal}.
\end{defn}

\begin{defn}
    Una solución de una EDO es una función $y = y(x)$ tal que

    \[
    \Phi\big( x, y(x), y'(x), \dots \big) = 0
    \]
\end{defn}

\subsection*{EDO de Orden 1}

\begin{defn}
    Una EDO de orden 1 es una ecuación diferencial de tipo

    \[
    \Phi\big( x, y, y' \big) = 0
    \]

    Si la EDO es lineal, entonces tiene la forma

    \[
    a_1(x)\frac{dy}{dx} + a_0(x)y = f(x), \quad a_1 \neq 0
    \]
\end{defn}

Pasaremos ahora a resolver la EDO general lineal de primer orden mediante el método de los \ul{factores integrantes}:

Consideremos la ecuación

\begin{equation}\label{eq:met_integrantes1}
    \frac{dy}{dt} + p(t)y = g(t)
\end{equation}

donde $p$ y $g$ son funciones cualesquiera. Multipliquemos esta expresión por una función $\mu(t)$ y nos queda

\begin{equation}\label{eq:met_integrantes2}
    \mu(t)\frac{dy}{dt} + \mu(t)p(t)y = \mu(t)g(t)
\end{equation}

Ahora, si consideramos $\mu$ tal que satisface

\begin{equation}\label{eq:met_integrantes3}
    \frac{d\mu(t)}{dt} = p(t)\mu(t)
\end{equation}

vemos que el lado izquierdo de \refeq{eq:met_integrantes2} es la derivada del producto $\mu(t)y$.

Si además asumimos que $\mu(t)$ es positivo entonces nos queda

\[
    \frac{d\mu(t)/dt}{\mu(t)} = p(t)
\]

La derivada del logaritmo es conocida, entonces por el TFC esto implica que

\[
    \ln\mu(t) = \int p(t)dt + k
\]

Escogiendo $k = 0$ obtenemos

\[
    \mu(t) = \exp \big( \int p(t)dt \big)
\]

Así, la ecuación \refeq{eq:met_integrantes2} equivale a

\[
    \frac{d}{dt}(\mu(t)y) = \mu(t)g(t)
\]

Y se sigue que

\[
    \mu(t)y = \int \mu(s)g(s) + c
\]

Y la solución general de \refeq{eq:met_integrantes1} es

\[
    y = \displaystyle \frac{\int \mu(s)g(s) + c}{\mu(t)}
\]

donde $\mu$ es el factor integrante de la ecuación.

Para aplicar este método vemos que necesitamos dos integraciones: Una para obtener $\mu$ y otra para obtener $y$.