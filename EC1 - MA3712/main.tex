% Plantilla, paquetes, macros, etc. para los proyectos en latex realizados por Eduardo Gavazut, Universidad Simón Bolívar
% Este .tex se deriva directamente del utilizado de las notas, así que probablemente hayan muchos paquetes y macros sin utilizar

\documentclass{article}
\usepackage[utf8]{inputenc}
\usepackage{parskip}
\usepackage{amssymb, amsthm, amsmath, fdsymbol, mathtools} % Varios paquetes para símbolos y fuentes
\usepackage{tikz, pgfplots} % Para dibujar
\usepackage{tkz-graph, tkz-berge} % Para dibujar grafos
\usepackage[linguistics]{forest} % Para dibujar árboles
\usepackage{titling} % Para estilizar el título
\usepackage{scrextend} % Añade márgenes para hacer bloques de texto
\usepackage{enumitem} % Para enumerar sin sangría
\usepackage{graphicx, subcaption, wrapfig} % Para colocar figuras e imágenes
\usepackage{lastpage}
\usepackage{fancyhdr} % Para hacer encabezados y pie de página más estilizados
\usepackage{color} % Para usar colores en el texto
\usepackage{soul} % Para subrayar con colores
\usepackage{geometry}
\usepackage{hyperref}
\usepackage{booktabs} % Para hacer tablas un poco más estilizadas
\usepackage{multirow}

% Cambia el tamaño de los captions
\usepackage[font={footnotesize}]{caption}

% Establece los enviroments para teoremas, ejemplos, definiciones, etc

\newtheorem{teo}{Teorema}
\newtheorem{cor}[teo]{Corolario}
\newtheorem{lem}[teo]{Lema}
\newtheorem{pro}[teo]{Proposición}
\newtheorem{pre}{Pregunta}

\theoremstyle{definition}
\newtheorem{defn}{Definición}
\newtheorem{ejem}{Ejemplo}
\newtheorem{ejer}{Ejercicio}
\newtheorem{notn}{Notación}
\newtheorem{nota}{Nota}
\newtheorem{prob}{Problema}

% Comandos para símbolos
\newcommand{\R}{\mathbb{R}}
\newcommand{\Z}{\mathbb{Z}}
\newcommand{\N}{\mathbb{N}}

% Cambia el nombre de varios comandos
\renewcommand{\contentsname}{Contenido}
\renewcommand*{\proofname}{Demostración}
\renewcommand{\figurename}{Fig.}

% Cambia los márgenes
\geometry{
    left=20mm,
    right=20mm,
}

% Resetea el contador de ecuaciones en cada sección
\newcounter{sec}
\setcounter{sec}{0}
\counterwithin*{equation}{sec}
\counterwithin*{ejer}{sec}

% Establecemos cómo será el encabezado y el pie de página
\fancyhf{}
\pagestyle{fancy}
\fancyhf{}
\fancyhead[L]{MA3712}
\fancyhead[C]{Eduardo José Gavazut Pinto}
\fancyhead[R]{13-10524}
\fancyfoot[L]{Sección 1}
\fancyfoot[R]{Profesor: Daniel Morales}
\fancyfoot[C]{\thepage\ de \pageref{LastPage}}
\renewcommand{\headrulewidth}{2pt} 
\renewcommand{\footrulewidth}{2pt}

% Añade color a los lados de los grafos tikz
\tikzset{
    EdgeStyle/.append style = {blue}
}

% Permite agregar etiquetas a los niveles de un árbol
\forestset{%
  label tree/.style={
    for tree={tier/.option=level},
    level label/.style={
      before typesetting nodes={
        for nodewalk={current,tempcounta/.option=level,group={root,tree breadth-first},ancestors}{if={>OR={level}{tempcounta}}{before drawing tree={label me=##1}}{}},
      }
    },
    before drawing tree={
      tikz+={\coordinate (a) at (current bounding box.east);},
    },
  },
  label me/.style={tikz+={\node [anchor=base north] at (.parent |- a) {#1};}},
}

% Establece el subrayado de color rojo
\definecolor{ferrari}{rgb}{1,0.17,0}
\setulcolor{ferrari}

% Reduce el espacio entre el título y el header
\setlength{\droptitle}{-5.5em}
\renewcommand\maketitlehookc{\vspace{-3ex}}

% Define el espaciado entre párrafos
\setlength{\parskip}{1.5em}

% Definimos nuestro título
\pretitle{\begin{flushleft}\LARGE\sffamily}
\title{
Ecuaciones Diferenciales 1\\
Universidad Simón Bolívar
}
\posttitle{\par\end{flushleft}\vskip 0.5em}
\preauthor{\begin{flushleft}\large\scshape}
\author{
Eduardo Gavazut \\
Carnet: 13-10524}
\postauthor{\par\end{flushleft}}
\predate{\begin{flushleft}\large\scshape}
\date{Enero - Marzo 2024}
\postdate{\par\end{flushleft}}

% Aquí empieza el documento
\pgfplotsset{compat=1.18}
\begin{document}

\maketitle
\thispagestyle{fancy}

% \section*{Clase 1}

\subsection*{Definiciones Básicas}

\begin{defn}
    Sea $x$ una variable independiente, $y$ una variable dependiente de $x$ y $D^{(i)}$ (con $i \in \N - \{0, +\infty\}$) el operador derivación. Una relación
    
    \[
    \Phi\big( x, y, Dy, D^{(2)}y, \dots \big) = 0
    \]

    es llamada una \ul{ecuación diferencial}.

    Una ecuación diferencial se dice que es \ul{ordinaria} si el operador derivada es el operador derivada en una variable.
\end{defn}

\begin{defn}
    El orden de una EDO es el orden de la mayor derivada presente en la ecuación.
\end{defn}

\begin{defn}
    Una EDO se dice \ul{lineal} si es lineal en términos de las variables $y, y', y'', \dots, y^{(n)}, \dots$.

    En otro caso, se dice que la EDO es \ul{no lineal}.
\end{defn}

\begin{defn}
    Una solución de una EDO es una función $y = y(x)$ tal que

    \[
    \Phi\big( x, y(x), y'(x), \dots \big) = 0
    \]
\end{defn}

\subsection*{EDO de Orden 1}

\begin{defn}
    Una EDO de orden 1 es una ecuación diferencial de tipo

    \[
    \Phi\big( x, y, y' \big) = 0
    \]

    Si la EDO es lineal, entonces tiene la forma

    \[
    a_1(x)\frac{dy}{dx} + a_0(x)y = f(x), \quad a_1 \neq 0
    \]
\end{defn}

Pasaremos ahora a resolver la EDO general lineal de primer orden mediante el método de los \ul{factores integrantes}:

Consideremos la ecuación

\begin{equation}\label{eq:met_integrantes1}
    \frac{dy}{dt} + p(t)y = g(t)
\end{equation}

donde $p$ y $g$ son funciones cualesquiera. Multipliquemos esta expresión por una función $\mu(t)$ y nos queda

\begin{equation}\label{eq:met_integrantes2}
    \mu(t)\frac{dy}{dt} + \mu(t)p(t)y = \mu(t)g(t)
\end{equation}

Ahora, si consideramos $\mu$ tal que satisface

\begin{equation}\label{eq:met_integrantes3}
    \frac{d\mu(t)}{dt} = p(t)\mu(t)
\end{equation}

vemos que el lado izquierdo de \refeq{eq:met_integrantes2} es la derivada del producto $\mu(t)y$.

Si además asumimos que $\mu(t)$ es positivo entonces nos queda

\[
    \frac{d\mu(t)/dt}{\mu(t)} = p(t)
\]

La derivada del logaritmo es conocida, entonces por el TFC esto implica que

\[
    \ln\mu(t) = \int p(t)dt + k
\]

Escogiendo $k = 0$ obtenemos

\[
    \mu(t) = \exp \Bigg( \int p(t)dt \Bigg)
\]

Así, la ecuación \refeq{eq:met_integrantes2} equivale a

\[
    \frac{d}{dt}(\mu(t)y) = \mu(t)g(t)
\]

Y se sigue que

\[
    \mu(t)y = \int \mu(s)g(s) + c
\]

Y la solución general de \refeq{eq:met_integrantes1} es

\[
    y = \displaystyle \frac{\int \mu(s)g(s) + c}{\mu(t)}
\]

donde $\mu$ es el factor integrante de la ecuación.

Para aplicar este método vemos que necesitamos dos integraciones: Una para obtener $\mu$ y otra para obtener $y$.
\begin{teo}
    Sean $f: A \subseteq \R^n \rightarrow \R^m$ con $A$ abierto, conexo y convexo, y $x_0 \in A$. Si $\frac{\partial f_i}{\partial x_j}(x_0)$ existen y además $\frac{\partial f_i}{\partial x_j} \in C(A)$, entonces $f$ es diferenciable.
\end{teo}

\begin{proof}
    Por razones de simplicidad, solamente se demostrará el caso para $f: A \subset \R^n \rightarrow \R$ (el caso general se realiza tomando coordenada a coordenada). Queremos demostrar que
    
    \[
    \lim_{h \to 0} \frac{\left| f(x + h) - f(x) - \nabla f(x) \cdot h \right|}{\normaeuc{h}} = 0
    \]
    
    Evaluemos primero lo siguiente
    
    \begin{align*}
        f(x + h) - f(x) &= f(x_1 + h_1, \dots, x_n + h_n) \pm f(x_1, \dots, x_n + h_n) \\
            &\pm f(x_1, x_2, \dots, x_n + h_n) \pm \dots \pm f(x_1, x_2, \dots, x_{n-1}, x_n + h_n) \\
            &- f(x_1, \dots, x_n)
    \end{align*}
    
    Aplicando ahora el \TVM, en cada coordenada, nos queda que
    
    \[
    f(x + h) - f(x) = \frac{\partial f}{\partial x_1}(y_1)h_1 + \frac{\partial f}{\partial x_2}(y_2)h_2 + \dots + \frac{\partial f}{\partial x_n}(y_n)h_n
    \]
    
    Entonces podemos concluir que
    
    \begin{gather*}
        \big| f(x + h) - f(x) - \nabla f(x) \cdot h \big| = \Bigg| \sum_{j=1}^n \bigg( \frac{\partial f}{\partial x_j} (y_j) - \frac{\partial f}{\partial x_j} (x) \bigg) h_j \Bigg| \\
            \leq \sum_{j=1}^n \Bigg| \bigg( \frac{\partial f}{\partial x_j} (y_j) - \frac{\partial f}{\partial x_j} (x) \bigg) \Bigg| |h_j|
    \end{gather*}
    
    \noindent como $|h_j| \leq \normaeuc{h}$ (para todo $j$), entonces lo anterior queda así
    
    \begin{equation}\label{eq:2.1.1}
        \leq \left(\sum_{j=1}^n \Bigg| \bigg( \frac{\partial f}{\partial x_j} (y_j) - \frac{\partial f}{\partial x_j} (x) \bigg) \Bigg| |h_j|\right) \leq \sum_{j=1}^n \Bigg| \bigg( \frac{\partial f}{\partial x_j} (y_j) - \frac{\partial f}{\partial x_j} (x) \bigg) \Bigg| \normaeuc{h}
    \end{equation}
    
    Ahora, observemos que para cada $j = 1, \dots, n$, el teorema del valor medio garantiza la existencia de un $c_j \in (x_j, x_j + h_j)$ tal que $y_j = (x_1, \dots, c_j, \dots, x_n)$. Si $h \to 0$, entonces $h_j \to 0$ para todo $j = 1, \dots, n$. Por lo tanto $y_j \to x_j$.
    
    Como todas las derivadas parciales son continuas por hipótesis, obtenemos que
    
    \[
    \lim_{h_j \to 0} \frac{\partial f}{\partial x_j} (y_j) = \frac{\partial f}{\partial x_j} (x_j)
    \]
    
    Y por \ref{eq:2.1.1}, tenemos que si hacemos a $x_j \to 0$, nos queda que
    
    \[
    \left| f(x + h) - f(x) \right| \leq \normaeuc{h} \cancelto{0}{\sum_{j=1}^n \Bigg| \bigg( \frac{\partial f}{\partial x_j} (y_j) - \frac{\partial f}{\partial x_j} (x) \bigg) \Bigg|}
    \]
    
    Y en conclusión,
    
    \[
    \lim_{h \to 0} \frac{\left| f(x + h) - f(x) - \nabla f(x) h \right|}{\normaeuc{h}} = 0
    \]
\end{proof}

\subsection{Regla de la Cadena}
\stepcounter{subsec}

\end{document}