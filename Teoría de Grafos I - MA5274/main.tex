\documentclass{tufte-handout}
\usepackage[utf8]{inputenc}
\usepackage{parskip}
\usepackage{amssymb, amsthm, amsmath, fdsymbol, mathtools, cancel, extarrows} % Varios paquetes para símbolos y fuentes
\usepackage{tikz, pgfplots} % Para dibujar
\usepackage{tkz-graph, tkz-berge} % Para dibujar grafos
\usepackage[linguistics]{forest} % Para dibujar árboles
\useforestlibrary{edges}
\usepackage{algorithm2e} % Para escribir pseudocódigo
\usepackage{titling} % Para estilizar el título
\usepackage{scrextend} % Añade márgenes para hacer bloques de texto
\usepackage{enumitem} % Para enumerar sin sangría
\usepackage{graphicx, subcaption} % Para colocar figuras e imágenes
\usepackage{lastpage}
\usepackage{fancyhdr} % Para hacer encabezados y pie de página más estilizados
\usepackage{color} % Para usar colores en el texto
\usepackage{soul} % Para subrayar con colores
\usepackage{soulutf8}
\usepackage{titling} % Cambia los parámetros del título
\usepackage{booktabs} % Para hacer tablas un poco más estilizadas
\usepackage{multirow}
\usepackage[font={footnotesize}]{caption} % Cambia el tamaño de los captions
\usepackage{subcaption} % Para referenciar subfiguras
\usepackage{hyperref}

% Establece el subrayado de color rojo
\definecolor{ferrari}{rgb}{1,0.17,0}
\setulcolor{ferrari}

% Agrega etiquetas con colores a las matrices
\usepackage{colortbl}
\usepackage{nicematrix}
\NiceMatrixOptions{
code-for-first-row = \color{red} ,
code-for-last-row = \color{red} ,
code-for-first-col = \color{red} ,
code-for-last-col = \color{red}
}

\tikzset{unode/.style = {
    circle, 
    draw=black, 
    thick,
    fill=black,
    inner sep=2.3pt,
    minimum size=2.3pt } }
\tikzset{uedge/.style = {
    draw=black, 
    very thick} }

% Establece los enviroments para teoremas, ejemplos, definiciones, etc
\newtheorem{teo}{Teorema}
\newtheorem{cor}[teo]{Corolario}
\newtheorem{lem}[teo]{Lema}
\newtheorem{pro}[teo]{Proposición}
\newtheorem{pre}{Pregunta}

\theoremstyle{definition}
\newtheorem{defn}{Definición}
\newtheorem{ejem}{Ejemplo}
\newtheorem{ejer}{Ejercicio}
\newtheorem{notn}{Notación}
\newtheorem{nota}{Nota}
\newtheorem{prob}{Problema}

% Comandos para símbolos
\newcommand{\Nastk}{\mathbb{N}^*}
\newcommand{\N}{\mathbb{N}}
\newcommand{\Z}{\mathbb{Z}}
\newcommand{\R}{\mathbb{R}}
\newcommand{\lngtd}[1]{\operatorname{long}(#1)}
\newcommand{\LC}{\nameref{teo:primer}}

% Cambia el nombre de varios comandos
\renewcommand{\contentsname}{Contenido}
\renewcommand*{\proofname}{Demostración}
\renewcommand{\figurename}{Fig.}

% Resetea el contador de ecuaciones en cada sección y/o subsección
\newcounter{sec}
\newcounter{subsec}
\setcounter{sec}{0}
\setcounter{subsec}{0}
\counterwithin*{equation}{sec}
\counterwithin*{equation}{subsec}

% Establece las notas de margen
\newcommand{\marginfootnote}[1]{\footnotemark\footnotetext{#1}}

% Establecemos cómo será el encabezado y el pie de página
\fancyhf{}
\pagestyle{fancy}
\fancyhf{}
\fancyhead[L]{MA-5274}
\fancyhead[C]{Eduardo José Gavazut Pinto}
\fancyhead[R]{13-10524}
\fancyfoot[L]{Sección 1}
\fancyfoot[R]{Profesor: Jesús Nieto}
\fancyfoot[C]{\thepage\ de \pageref{LastPage}}
\renewcommand{\headrulewidth}{2pt} 
\renewcommand{\footrulewidth}{2pt}

% Establece los entornos para los bloques de pseudocódigo
\RestyleAlgo{ruled}
\newenvironment{algoritmo}[1][htb]
  {\renewcommand{\algorithmcfname}{Algoritmo}% Update algorithm name
   \begin{algorithm}[#1]%
  }{\end{algorithm}}

\newcommand{\row}[1]{%
  \hbox{$\activatem\romannumeral\number\number#1 000 \unskip$}\vskip 0.5em plus 3em
}
\newcommand{\activatem}{%
  \begingroup\lccode`~=`m \lowercase{\endgroup\def~}{\bullet\hskip1em}%
  \mathcode`m="8000
}

% Define la geometría del margen
\geometry{
	left=13mm, % left margin
	textwidth=130mm, % main text block
	marginparsep=8mm, % gutter between main text block and margin notes
	marginparwidth=55mm % width of margin notes
}

% Permite agregar etiquetas a los niveles de un árbol
\forestset{%
  label tree/.style={
    for tree={tier/.option=level},
    level label/.style={
      before typesetting nodes={
        for nodewalk={current,tempcounta/.option=level,group={root,tree breadth-first},ancestors}{if={>OR={level}{tempcounta}}{before drawing tree={label me=##1}}{}},
      }
    },
    before drawing tree={
      tikz+={\coordinate (a) at (current bounding box.east);},
    },
  },
  label me/.style={tikz+={\node [anchor=base north] at (.parent |- a) {#1};}},
}

% Reduce el espacio entre el título y el header
\setlength{\droptitle}{-5.5em}
\renewcommand\maketitlehookc{\vspace{-3ex}}

% Define el espaciado entre párrafos
\setlength{\parskip}{1.5em}

% Definimos nuestro título
\pretitle{\begin{flushleft}\LARGE\sffamily}
\title{
Notas de Teoría de Grafos I, Septiembre-Diciembre 2022 \\
Universidad Simón Bolívar
}
\posttitle{\par\end{flushleft}\vskip 0.5em}
\preauthor{\begin{flushleft}\large\scshape}
\author{
Eduardo Gavazut \\
Carnet: 13-10524}
\postauthor{\par\end{flushleft}}
\predate{\begin{flushleft}\large\scshape}
\date{Septiembre-Diciembre 2022}
\postdate{\par\end{flushleft}}

% Aquí empieza el documento
\begin{document}

\maketitle
\thispagestyle{fancy}

\tableofcontents
\break
\section{Conceptos Fundamentales}

Este curso sigue los tres primeros capítulos de Introduction to Graph Theory de West. Esto comprende grafos, conteo, árboles, matching, teoremas de cubrimiento y una introducción a la factorización de grafos. Por lo tanto, es fundamental estudiar con el libro a la hora de revisar este curso y buscar ejercicios.

\subsection{Definiciones y teoremas básicos}

\begin{defn}
    Un \ul{grafo} es una tripleta $G = (V(G, E(G), f_G)$ en el cual $V(G)$ y $E(G)$ son conjuntos y $f_G$ es una función que asigna a cada $e \in E(G)$ un \textbf{par no ordenado} $\{x,y\} \subset V(G)$.
\end{defn}

\begin{prob}[El problema de los puentes de Könisberg]\label{prob:konisberg}

    \begin{marginfigure}
        \centering
        \includegraphics[scale=0.5]{img/konisberg.png}
        \caption{Parte del mapa de la ciudad de Könisberg. En verde están reflejados los puentes descritos en el problema.}
        \label{fig:konisberg}
    \end{marginfigure}
    
    La ciudad de Könisberg se encontraba ubicada sobre el río Pregel en Prusia. La ciudad ocupaba dos islas más dos áreas sobre ambas orillas. Estas regiones se encontraban unidas por $7$ puentes. La gente en la ciudad empezó a preguntarse si podían salir de sus hogares, cruzar cada puente exactamente una vez y regresar a casa.
    
    \begin{marginfigure}
        \centering
        \begin{tikzpicture}
            \SetGraphUnit{2}
            \Vertices{circle}{d,c,b,a}
            \Edge[label=$e_6$](c)(d)
            \Edge[label=$e_7$](c)(b)
            \Edge[label=$e_3$](a)(c)
            \tikzset{EdgeStyle/.append style = {bend left}}
            \Edge[label=$e_1$](a)(b)
            \Edge[label=$e_2$](b)(a)
            \Edge[label=$e_4$](a)(d)
            \Edge[label=$e_5$](d)(a)
        \end{tikzpicture}
        \caption{Grafo que representa la situación descrita en el problema \ref{prob:konisberg}.}
        \label{fig:konisberg-model}
    \end{marginfigure}
    
    Este problema puede ser analizado a través de un grafo: Etiquetamos las $4$ regiones de la figura \ref{fig:konisberg} y graficamos como en la figura \ref{fig:konisberg-model}. Vemos que en este caso,
    
    \begin{itemize}
        \item $V(G) = \{a, b, c, d\}$.
        \item $E(G) = \{e_1, e_2, e_3, e_4, e_5, e_6, e_7\}$.
        \item $f_G(e_1) = \{a,b\} = f_g(e_2)$, $f_G(e_3) = \{a,c\}$, $f_G(e_4) = \{a,d\} = f_G(e_5)$, $f_G(e_6) = \{d,c\}$ y $f_G(e_7) = \{b,c\}$.
    \end{itemize}
    
    Más adelante responderemos la pregunta que plantea el problema. Por ahora nos limitaremos a revisar más definiciones.
\end{prob}

\begin{defn}[Vocabulario]
    A los elementos de $v \in V(G)$ les llamaremos \ul{vértices} y a los de $e \in E(G)$ les llamaremos \ul{lados o arista}. Dados $v, w \in V(G)$ tales que $\{v, w\} = f_G(e)$ para algún $e \in E(G)$, entonces $v$ y $w$ son \ul{vecinos o adyacentes}.
    
    En un grafo es posible que un vértice sea vecino de sí mismo, y en este caso decimos que el lado correspondiente es un \ul{bucle}.
    
    \begin{marginfigure}
        \centering
        \begin{tikzpicture}
            \SetGraphUnit{2}
            \Vertex{a}
            \EA(a){b}
            \Edge(a)(b)
            \Loop[dist=2cm](a)
        \end{tikzpicture}
        \caption{Ejemplo de un grafo con bucle en el vértice $a$.}
    \end{marginfigure}
    
    Decimos que el grafo $G$ es simple si no tiene bucles y $f_G$ es inyectiva (es decir que no tiene lados paralelos como en el grafo de la figura \ref{fig:konisberg-model}). En este caso se identifica a $E(G)$ con el rango de $f_G$. Es decir, los lados son los pares no ordenados de vértices adyacentes.
\end{defn}

\begin{defn}
    Un \ul{clique} en un grafo es un conjunto de vértices que tienen la propiedad de que todos los vértices son vecinos entre sí. En general, un clique es un conjunto de vértices que son adyacentes dos a dos.
    
    Un conjunto de vértices es \ul{independiente} si sus elementos son no adyacentes dos a dos.
\end{defn}

\begin{prob}
    En todo grupo de seis personas, hay tres que se conocen o tres que son mutuamente desconocidos. Es decir, ¿es cierto que en todo grafo con $6$ vértices hay un clique con tres vértices o un conjunto independiente con tres vértices?
    
    \begin{marginfigure}
        \centering
        \begin{tikzpicture}
            \GraphInit[vstyle=Normal]
            \SetVertexNoLabel
            \grComplete[RA=2]{6}
            \end{tikzpicture}
        \caption{Cada vértice representa a una persona.}
        \label{fig:3conocidos}
    \end{marginfigure}
    
    Para analizar este problema, etiquetamos a las personas del grupo y las emparejamos como en la figura \ref{fig:3conocidos}. Vemos que para todo vértice $i$, $\{i,j\} \in E(G)$ para todo $j \neq i$. Ahora procedamos a pintar los lados del grafo de la siguiente manera: Para todo lado $\{i,j\}$, si $i$ conoce a $j$ lo pintamos de rojo, de lo contrario se elimina el lado.
    
    \begin{marginfigure}
        \centering
        \begin{tikzpicture}
            \SetGraphUnit{2}
            \Vertices{circle}{$v$,$v_1$,$v_2$,$v_3$,$v_4$,$v_5$}
            \Edges[color=red]($v$,$v_1$,$v$,$v_2$,$v$,$v_3$)
        \end{tikzpicture}
        \caption{Pintamos los lados de color rojo como lo describimos en el problema.}
        \label{fig:3conocidos-rojo}
    \end{marginfigure}
    
    Entonces, fijémonos en un vértice $v$. Para el resto de los $5$ vértices, cada lado correspondiente lo podemos pintar de rojo o no son adyacentes, y por el principio del casillero, hay al menos $3$ lados pintados de un solo color o que no son adyacentes. Supongamos que están pintados de rojo, y llamemos a los vértices $v_1$, $v_2$, $v_3$ como en la figura \ref{fig:3conocidos-rojo}. Si alguno de los lados $\{v_1, v_2\}$, $\{v_2, v_3\}$ ó $\{v_1, v_3\}$ está pintado de rojo, entonces tenemos un clique de tres vértices donde cada lado es de color rojo. Por el contrario, si los tres lados no son adyacentes entre sí, entonces tenemos un conjunto independiente.
    
    De cualquier manera, tenemos $3$ personas que se conocen (clique rojo) o tres personas que no se conocen (conjunto independientes).
\end{prob}
\section{Paseos, recorridos y ciclos}
\stepcounter{sec}

En la parte anterior definimos los caminos, pero dependiendo del problema, querremos modelarlo prestando atención en el movimiento dentro del grafo. Quizá nos interese fijarnos en los vértices recorridos o en los lados recorridos.

\subsection{Conexidad}
\stepcounter{subsec}

\begin{defn}
    Un \ul{paseo} $W$ es una lista $v_0, e_1, v_2, \dots, e_k, v_k$ de vértices y lados de un grafo $G$ tales que para $i = 1, \dots, k$ se tiene $f_G(e_k) = \{v_{j-1},v_j\}$. Vemos que $k$ es la cantidad de lados del paseo, y lo denotaremos por $k = \lngtd{W}$.
    
    Un \ul{recorrido} es un paseo sin lados repetidos, es decir que si $W$ es un paseo en $G$, entonces $\forall i,j$ tenemos que $i \neq j \implies e_i \neq e_j$, con $e_i, e_j \in W$.
    
    Diremos que un paseo es \ul{cerrado} si $v_0 = v_k$.
\end{defn}

Es inmediato suponer que existe una relación entre los paseos y los caminos (que definimos en la sección anterior). Veamos concretamente cómo se relacionan:

\begin{lem}
    Todo paseo de $u$ a $v$ contiene un camino de $u$ a $v$.
\end{lem}

\begin{proof}
    Pasaremos a realizar la demostración de este lema por inducción: Sea $W$ un camino con $\lngtd{W} = n$. Si $n = 0$, entonces el paseo no tiene lados, por lo tanto $W$ consiste de un único vértice y $u = v$. De esta forma $W$ contiene un camino de $u$ a $v$ de longitud $0$.
    
    \begin{marginfigure}
        \centering
        \begin{tikzpicture}
            \SetGraphUnit{1}
            \Vertex{u}
            \WE(u){v1}
            \WE(v1){v2}
            \WE(v2){v3}
            \WE(v3){w}
            \SO(w){v}
            \Edges(u,v1,v2,v3,w,v)
            \Loop[dist=1cm](w)
        \end{tikzpicture}
        \caption{Situación descrita en la demostración del lema. El bucle en $w$ representa los vértices entre $w$ y su repetición. Al igual que ocurre en $w$, esta situación se puede generalizar para cualquier vértice.}
    \end{marginfigure}
    
    Supongamos ahora que el teorema se cumple para $n < k$. Toca demostrar para $n = k$: Si $W$ no tiene vértices repetidos, entonces todos sus vértices y lados forman un camino de $u$ a $v$. Si $W$ tiene un vértice repetido $w$, entonces omitamos todos los lados y vértices que aparecen entre $w$ y su repetición, de esta forma obtenemos un paseo $W'$ contenido en $W$. Como $\lngtd{W'} < k$, podemos aplicar la H.I y hay un camino $P$ de $u$ a $V$ contenido en $W'$. Ya que $W' \subset W$, entonces el camino $P$ está contenido en $W$.
    
    De esta forma, hemos demostrado el lema por inducción.
\end{proof}

\begin{defn}
    Sean $G$ un grafo y $u, v \in G$. Diremos que $x$ está \ul{conectado} con $y$ si existe un camino de $u$ a $v$ (y de lo contrario, \ul{desconectado}). La \ul{relación de conexidad} sobre $V(G)$ consiste en los pares ordenados $(u,v)$ tales que $u$ está conectado a $v$. Las clases de equivalencia de esta relación son las componentes del grafo $G$, es decir todo subgrafo de $G$ tal que es conexo maximal.
\end{defn}

\begin{ejem}\label{ejem:cortes}
    El grafo de abajo tiene $4$ componentes, una de ellas siendo un vértice aislado. Los conjuntos de vértices de cada componente son $\{a,b\}$, $\{c,d,e,f,g\}$, $\{h\}$ e $\{i,j,k\}$, y estas son las clases de equivalencia respecto a la relación de conexidad.
    
    \begin{figure}
        \centering
        \begin{tikzpicture}
            \SetGraphUnit{1}
            \Vertex{a}
            \SO(a){b}
            \Edges(a,b)
            \Vertex[x=1 , y=0]{c}
            \EA(c){d}
            \SO(d){g}
            \EA(d){e}
            \EA(e){f}
            \Edges(c,g,d,c,d,e,g,e,f)
            \Vertex[x=3.5 , y=-1]{h}
            \Vertex[x=5 , y=-1]{i}
            \Vertex[x=5.5 , y=0]{j}
            \EA(j){k}
            \Edges(i,j,k)
        \end{tikzpicture}
        \caption{Grafo con $4$ componentes.}
    \end{figure}
\end{ejem}

Vemos del ejemplo anterior que eliminar un vértice o lado puede incrementar el número de componentes. Por ejemplo, si eliminamos el lado que incide en $i$ pasamos a tener $5$ componentes, lo mismo ocurre si eliminamos $j$ y los lados que inciden con él.

\begin{defn}
    Un \ul{lado-corte} ó \ul{vértice-corte} de un grafo es un lado o vértice que al retirarlo del grafo incrementa el número de componentes. Escribimos $G - e$ o $G - M$ para denotar al subgrafo de $G$ obtenido al retirar el lado $e$ o el conjunto de lados $M$. Escribimos $G - v$ ó $G - S$ para denotar al subgrafo obtenido al retirar el vértice $e$ o el conjunto de vértices $S$.
\end{defn}

Es importante resaltar que las componentes son disjuntas dos a dos, es decir que no existe un par de componentes que comparta un vértice. Añadir un lado que une a dos vértices en componentes distintas hace que se reduzca en $1$ el número de componentes. De esta forma, añadir un lado reduce el número de componentes en $0$ o en $1$, y eliminar un lado incrementa el número de componentes por $0$ o $1$.

\begin{pro}
    Todo grafo con $n$ vértices y $k$ lados tiene al menos $n-k$ componentes.
\end{pro}

\begin{proof}
    Un grafo con $n$ vértices sin lados tiene $n$ componentes. Sabemos por la discusión anterior que cada lado reduce el número en componentes en máximo $1$, así que cuando se han añadido $k$ lados, el número de componentes sigue siendo mayor o igual a $n-k$.
\end{proof}

\begin{ejem}
    Para el grafo presentado en el ejemplo \ref{ejem:cortes}, tenemos que
    
    \begin{itemize}
        \item Lados-corte: $ab$, $ef$, $ij$, $jk$.
        \item Vértices-corte: $e$, $j$.
    \end{itemize}
\end{ejem}

\begin{defn}
    Sean $G$ un grafo y $T \subseteq V(G)$. El \ul{grafo inducido} denotado por $G[T]$ está conformado por $T$, $f_G$ y todo lado $e \in E(G)$ que cumpla lo siguiente: para algún par no ordenado $\{u,v\}$ con $u,v \in V(T)$, se tiene que $\{u,v\} = f_G(e)$. Es decir, que el grafo inducido consiste también de todos los lados cuyos vértices están contenidos en $V(T)$.
\end{defn}

Para esta definición, es importante resaltar que un conjunto de vértices $S$ es un conjunto independiente sii el grafo inducido por $S$ no tiene lados.

\begin{ejem}
    Nuevamente, en el grafo del ejemplo \ref{ejem:cortes}, tenemos que $C_4$ y $P_5$ son subgrafos \textbf{NO} inducidos, y $P_4$ es un subgrafo inducido: Puede ser inducido por $\{c,g,e,f\}$ o por $\{c,d,e,f\}$.
\end{ejem}

\begin{teo}
    Sean $G$ un grafo y $e \in E(G)$, entonces $e$ es un lado-corte sii no pertenece a un ciclo.
\end{teo}

\begin{proof}
    Sea $e$ un lado de un grafo $G$ (con vértices $x,y$) y sea $H$ la componente que contiene a $e$. Como al eliminar $e$ ninguna otra componente se ve afectada, es suficiente probar que $H - e$ es conexo si y sólo si $e$ pertenece a un ciclo. Probemos ambas implicaciones:
    
    \begin{enumerate}
        \item[$\Leftarrow$] Supongamos que $e$ pertenece a un ciclo $C$ de $H$. Escojamos $u, v \in V(H)$. Como $H$ es conexo, entonces $H$ tiene un camino $P$ de $u$ a $v$. Si $P$ no contiene a $e$, entonces $P$ está en $H - e$, como esto es para todo $u, v \in V(H)$, $H - e$ es conexo. Si $P$ contiene a $e$ supongamos en primer lugar y sin pérdida de generalidad que $x$ está entre $u$ e $y$ en $P$. Como $H - e$ contiene un camino de $u$ a $x$ (el cual está en $P$), un camino de $x$ a $y$ (está en $C$) y un camino de $y$ a $v$ (está en $P$ nuevamente), por la transitividad de la relación de conexidad tenemos que eso implica que $H - e$ tiene un camino de $u$ a $v$, y esto se cumple para todo $u, v \in V(G)$.
        
        De esta forma, $u, v \in V(H)$ y $H - e$ es conexo.
        
        \item[$\Rightarrow$] Supongamos ahora que $H - e$ es conexo. Esto implica que existe un camino de $x$ a $y$, y si se agrega nuevamente el lado $e$, se tiene un ciclo. Por lo tanto $e$ pertenece a un ciclo.
    \end{enumerate}
    
    De esta manera queda demostrado.
\end{proof}
\section{Grafos y conteo}
\stepcounter{sec}

Algunas de las preguntas que ahora son relevantes y que contestaremos en esta sección son las siguientes:

\begin{itemize}
    \item ¿Cómo se relaciona la cantidad de lados de un grafo con la cantidad de vértices?
    \item ¿Cuántos subgrafos con características dadas, tiene un grafo?
    \item ¿Cuántas clases de isomorfismo tiene un grafo dado?
\end{itemize}

\begin{defn}
    Antes de hacer los análisis pertinentes, vale la pena recordar algunos conceptos y notación e introduciremos algunos más:
    
    \begin{enumerate}
        \item El \ul{grado} de $v \in V(G)$ es la cantidad de lados de $G$ incidentes con $v$ (si $v$ tiene bucles, estos contribuyen en $2$ al grado). Lo denotaremos por $d_G(v)$ ó $d(v)$.
        \item $\Delta(G)$ es el máximo de los grados en $G$.
        \item $\delta(G)$ es el mínimo de los grados en $G$.
        \item Si $\Delta(G) = \delta(G) = k$, decimos que $G$ es \ul{$k$-regular}.
        \item La \ul{vecindad} de $v \in V(G)$ es el conjunto $N_G(v)$ de los vecinos de $v$ en $G$.
        \item El \ul{orden} de $G$ es $n(G) = |V(G)|$.
        \item El \ul{tamaño} de $G$ es $e(G) = |E(G)|$.
        \item Dado $n \in \N$, denotamos $[n] = \{1, 2, \dots, n\}$.
    \end{enumerate}
\end{defn}

\subsection{Conteo y biyecciones}
\stepcounter{subsec}

Introduciremos esta subsección con un resultado bastante relevante: Si queremos calcular la suma de los grados de los vértices de $G$, cada lado se cuenta dos veces. Hemos demostrado el siguiente resultado:

\begin{teo}[Primer Teorema de la Teoría de Grafos]\label{teo:primer}
    Si $G$ es un grafo, entonces
    
    \[
    \sum_{v \in V(G)} d(v) = 2e(G)
    \]
\end{teo}

De este teorema se pueden sacar varios corolarios:

\begin{cor}
    En un grafo $G$, el promedio de los grados es
    
    \[
    \frac{2e(G)}{n(G)}
    \]
    
    Por lo tanto,
    
    \[
    \delta(G) \leq \frac{2e(G)}{n(G)} \leq \Delta(G)
    \]
\end{cor}

\begin{cor}
    $G$ tiene una cantidad par de vértices con grado impar.
\end{cor}

\begin{cor}
    Si $G$ es $k$-regular, entonces $e(G) = kn(G)/2$.
\end{cor}

\begin{cor}
    Si $n(G)$ es impar y $G$ es $k$-regular, entonces $k$ no puede ser impar.
\end{cor}

\begin{defn}
    El \ul{cubo $k$-dimensional} o \ul{hipercubo} $Q_k$, es el grafo simple cuyos vértices son las $k$-tuplas con entradas en $\{0,1\}$ y cuyos lados son las parejas de $k$-tuplas que difieren en \textbf{exactamente} una posición.
\end{defn}

\begin{figure}
    \centering
    \begin{tikzpicture}
        \SetGraphUnit{2}
        \SetVertexNoLabel
        \grCycle[rotation=45, prefix=a, RA=1]{4}
        \grCycle[x=3, y=1.5, rotation=45, prefix=b, RA=1]{4}
        \AssignVertexLabel{a}{000,100,110,010}
        \AssignVertexLabel{b}{001,101,111,011}
        \SetUpEdge[style={dashed}]
        \Edge(a0)(b0)
        \Edge(a1)(b1)
        \Edge(a2)(b2)
        \Edge(a3)(b3)
    \end{tikzpicture}
    \caption{Acá se puede apreciar una representación de $Q_3$}
    \label{fig:hipercubo}
\end{figure}

\begin{prob}[Estructura de los hipercubos]
    Con respecto a los hipercubos surjen muchísimas propiedades interesantes:
    
    \begin{itemize}
        \item Podemos establecer la \textit{paridad} de un vértice de $Q_k$ la siguiente manera: Si tiene una cantidad par de $1$s es par, y de lo contrario es impar. Cada lado de $Q_k$ incide en un vértice par y otro impar. De esta forma, los vértices pares son adyacentes únicamente a los impares, y viceversa. Por lo tanto ambos conjuntos de vértices forman un conjunto independiente y tenemos como resultado que $Q_k$ es bipartito.
        \item Cada posición en las $k$-tuplas se puede elegir de dos maneras, por lo tanto $n(Q_k) = 2^k$. Además, los vecinos de cada vértice se pueden obtener al cambiar una de las $k$ posiciones de la tupla, por lo tanto $Q_k$ es $k$-regular, y se tiene que $e(Q_k) = k2^{k-1}$.
        \item Vemos en la figura \ref{fig:hipercubo} que los lados no punteados son dos subgrafos de $Q_3$ isomorfos a $Q_2$, formados al mantener fijas la última coordenada de las tuplas. ¿De qué manera podemos generalizar este resultado?: Podemos formar un subcubo $j$-dimensional al fijar $k-j$ coordenadas y hacer que los valores en las $j$ sean escogidos a partir de una de las $2^j$ $j$-tuplas restantes. El subgrafo inducido por este conjunto de vértices es isomorfo a $Q_j$. Como hay $\binom{k}{j}$ maneras para escoger las $j$ coordenadas a variar, y $2^{k-j}$ maneras de escojer las coordenadas fijas y sus valores, en total tenemos $\binom{k}{j}2^{k-j}$ subcubos. Así, para cada $j \leq k$, $Q_k$ tiene $\binom{k}{j}2^{k-j}$ subgrafos isomorfos a $Q_j$.
    \end{itemize}
    
    % Maldita sea como me excitan los hipercubos
    Ahora, ¿cómo podemos construir un hipercubo $Q_k$?: Agregar $0$ a las tuplas de una copia de $Q_{k-1}$, agregar $1$ a las tuplas de otra copia de $Q_{k-1}$, y unir los vértices cuyas primeras $k-1$ coordenadas sean iguales. El resultado será $Q_k$.
\end{prob}

Vimos que un hipercubo es un grafo regular y bipartito, a continuación estableceremos una observación fundamental sobre esos grafos.

\begin{teo}
    Si $k > 0$, entonces un grafo $k$-regular y bipartito tiene la misma cantidad de vértices en cada partición.
\end{teo}

\begin{proof}
    Sea $G$ un grafo $k$-regular y bipartito. Supongamos que $\{A, B\}$ es la bipartición. Entonces cada lado de $G$ tiene un extremo en $A$, por lo tanto $e(G) = k|A|$. También cada lado de $G$ tiene un extremo en $B$, por lo tanto $e(G) = k|B|$. Esto implica que $|A| = |B|$.
\end{proof}

Otra técnica de conteo para conjuntos involucra establecer una biyección desde el conjunto a otro de un tamaño conocido. El siguiente problema utiliza este enfoque.

\begin{prob}
    ¿Cuántos $6$-ciclos tiene el grafo de Petersen?: Vimos anteriormente que el grafo de Petersen $P$ es $3$-regular. Por lo tanto tiene $10$ \textit{garras} (una por vértice). La figura \ref{fig:petersen} sugiere que a cada \textit{garra} le corresponde un $6$-ciclo. Si demostramos que este es el caso, tendremos que el grafo de Petersen tiene diez $6$-ciclos.
    
    \begin{marginfigure}
        \centering
        \begin{tikzpicture}
            \SetGraphUnit{1}
            \GraphInit[vstyle=Classic]
            \Vertex{w}
            \NO(w){u1}
            \SOEA(w){u2}
            \SOWE(w){u3}
            \Edge(w)(u1)
            \Edge(w)(u2)
            \Edge(w)(u3)
        \end{tikzpicture}
        \caption{Garra descrita en el problema.}
        \label{fig:garra}
    \end{marginfigure}
    
    Digamos que el ciclo $C$ es $(v1~v2~v3~v4~v5~v6)$. Como $P$ tiene cintura $5$, ningún $vj$ tiene más vecinos en $C$. Por lo tanto, cada $vj$ tiene exactamente un vecino fuera de $C$. Los vértices no adyacentes tienen exactamente un vecino en común $u1$, $u2$, $u3$, distintos entre sí ya que la cintura de $P$ es $5$. Solamente nos falta un vértice $w$ que sea vecino de $u1$, $u2$, $u3$. Por lo tanto esos vértices forman una garra con $w$.
    
    Veamos que una garra $G$ ocurre exactamente una vez quitando un $6$-ciclo: Sea $S$ el conjunto independiente en $G$. Cada $v \in S$ tiene sus otros vecinos fuera de $G$, y son distintos (ya lo discutimos). Entonces $P-G$ es $2$-regular y sus vértices son los seis a los que nos referimos en el párrafo anterior. Como la cintura de $P$ es $5$, entonces $P - G$ es un $6$-ciclo.
\end{prob}

Presentaremos una estrategia más de conteo: Los subgrafos de $G$ de la forma $G - v$ (con $v \in V(G)$) son útiles para hacer análisis, construcciones, demostraciones, etc. Recordemos que al retirar $v$, también estamos retirando los lados incidentes a este vértice. Por ejemplo, todos los subgrafos de la forma $G - v$ de $C_n$ son isomorfos a $P_n$.

\begin{pro}
    Sea $G$ un grafo simple con vértices $v_1, \dots, v_n$ y $n \geq 3$. Entonces
    
    \[
    e(G) = \dfrac{\sum e(G - v_i)}{n-2} \qquad \text{y} \qquad d_g(v_j) = \dfrac{\sum e(G - v_i)}{n-2} - e(G - v_j)
    \]
\end{pro}

\begin{proof}
    En primer lugar, un lado $e$ de $G$ está en $G - v_j$ sii $v_j$ no es extremo de $e$. Entonces $\sum e(G - v_i) = (n-2)e(G)$ ya que cada lado se cuenta $n-2$ veces.
    
    Además, para cada $j$ tenemos que $d_G(v_j)$ es el número de lados que se pierden al eliminar $v_j$ para obtener $G - v_j$. Entonces $d_G(v_j) = \frac{\sum e(G - v_i)}{n-2} - e(G - v_j)$.
    
    Y así, queda demostrado.
\end{proof}

El siguiente es un problema relacionado con subgrafos de la forma $G-v$:

\begin{prob}[Conjetura de Kelly, Ulam (1942)]
    Si $G$ es un grafo simple con orden al menos $3$, entonces $G$ está determinado de manera única por sus subgrafos de la forma $G - v$.
    
    Esta conjetura se conoce como la \textit{conjetura de la reconstrucción} ya que si tienes la lista de los subgrafos de la forma $G - e$ puedes reconstruir $G$ agregando vértices. En 2017 se dio una demostración.
\end{prob}

\subsection{Problemas extremos y optimización}
\stepcounter{subsec}

Un \textbf{problema extremo} pide el máximo o mínimo valor de una función definida sobre una clase de objetos. Por ejemplo, el máximo número de lados en un grafo simple con $n$ vértices es $\binom{n}{2}$.

Veamos un par de problemas más que derivaran en teoremas.

\begin{prob}
    ¿Cuál es el mínimo número de lados de un grafo conexo de orden $n$?:
    
    Vimos anteriormente que si $G$ tiene $n$ vértices y $k$ lados, entonces tiene al menos $n-k$ componentes\marginfootnote{Revisar la proposición \ref{pro:componentes}.}. Por lo que, si $k \leq n-2$ entonces $G$ no es conexo. Así que el mínimo buscado ha de ser mayor a $n - 2$.
    
    Más aún, la cota inferior es conseguida por el camino $P_n$.
\end{prob}

Así, hemos demostrado:

\begin{teo}
    El mínimo número de lados de un grafo conexo de orden $n$ es $n-1$.
\end{teo}

\begin{prob}
    ¿Cuál es el máximo valor de $\delta(G)$ para un grafo simple disconexo $G$?:
    
    Primero, sean $u, v \in V(G)$ no adyacentes. Ahora, por el teorema de inclusión y exclusión, la intersección de sus vecindades quedan de la siguiente manera:
    
    \[
    \left| N(u) \cap N(v) \right| = \left| N(u) \right| + \left| N(v) \right| - \left| N(u) \cup N(v) \right|
    \]
    
    Como $u, v$ no están en $N(u) \cup N(v)$, tenemos $\left| N(u) \cup N(v) \right| \leq n-2$. Por lo tanto, nos queda que
    
    \[
    \left| N(u) \cap N(v) \right| = \left| N(u) \right| + \left| N(v) \right| - n + 2
    \]
    
    Ahora nos preguntamos: ¿Cuál cota inferior será suficiente para que el lado derecho sea al menos $1$?
\end{prob}

\begin{teo}
    Si $G$ es un grafo simple de orden $n$ con $\delta(G) \geq (n-1)/2$ entonces $G$ es conexo.
\end{teo}

\begin{proof}
    Por la discusión anterior, tenemos que
    
    \[
    \left| N(u) \cap N(v) \right| = \left| N(u) \right| + \left| N(v) \right| - n + 2
    \]
    
    Como $\left| N(x) \right| \geq \delta(G) \geq (n-1)/2$, $\forall~x \in V(G)$ entonces lo anterior queda como
    
    \[
    \left| N(u) \cap N(v) \right| \geq 1
    \]
    
    Por lo que todo par $u, v \in V(G)$ no adyacentes está conectado. Por lo tanto $G$ es conexo.
\end{proof}

\begin{ejem}
    Sea $G$ no conexo, simple, de orden $n$ y con componentes isomorfas a la unión disjunta de $K_{\lfloor n/2 \rfloor}$ y $K_{\lceil n/2 \rceil}$. Como $\delta(G) = \lfloor n/2 \rfloor - 1$ y $G$ es no conexo, entonces $\delta(G)$ es efectivamente el máximo valor.
\end{ejem}

\begin{teo}
    El máximo valor de $\delta(G)$ entre los grafos simples, de orden $n$ y disconexos es $\lfloor n/2 \rfloor - 1$.
\end{teo}

\begin{notn}
    La unión disjunta de los grafos $H$ y $G$ también se conoce como \ul{suma} de $H$ y $G$ y se denota $H + G$.
\end{notn}

Otros problemas son los problemas de \textbf{optimización}, los cuales consisten en buscar objetos o cantidades extremas para un grafo dado en particular. Por ejemplo, el máximo tamaño de un subgrafo conexo, o de un conjunto de vértices independientes. Lo que sí suele darse en general para optimizar son algoritmos.

\begin{ejem}
    El siguiente algoritmo permite hallar un subgrafo bipartito \textit{grande} en el grafo sin bucles $G$:
    
    \begin{enumerate}
        \item Elige una bipartición $A$, $B$ para $G$.
        \item Llama $H$ al subgrafo de $G$ cuyos lados son los que tienen un extremo en $A$ y otro en $B$.
        \item Para cada $v \in V(G)$, si $d_H(v) < d_G(v)/2$, cambia a $v$ de conjunto bipartito.
    \end{enumerate}
    
    Ahora, se observa lo siguiente: Primero, $H$ es bipartito. Además el algoritmo termina porque $G$ es finito y en cada cambio se aumenta el tamaño de $H$. Por lo tanto, cuando el algoritmo termina, tennemos que $d_H(v) \geq d_G(v)/2$ para todo $v \in V(G)$. Sumando esto y aplicando la fórmula de la suma de los grados, nos queda que
    
    \[
    e(G) \geq e(G)/2
    \]
\end{ejem}

De esta forma, hemos demostrado el siguiente teorema:

\begin{teo}
    Si $G$ es un grafo finito sin bucles, entonces $G$ tiene un grafo bipartito con al menos $e(G)/2$ lados.
\end{teo}

\begin{nota}
    Este teorema no demuestra que el subgrafo bipartito tiene el máximo tamaño posible. De hecho, diferentes biparticiones iniciales pueden conducir a diferentes tamaños del subgrafo.
\end{nota}

\begin{prob}
    Dadas $n$ facciones en guerra, si ningun par de enemigos tiene un enemigo en común, ¿cuál es la cantidad máxima posible de pares de enemigos?:
    
    Esto equivale a determinar cuál es el máximo valor posible de $e(G)$ para un $G$ simple y de orden $n$ sin triángulos. Entonces sea $G$ como en la hipótesis, y consideremos $w \in V(G)$ con grado máximo $k$. Como no hay triángulos, no hay lados entre vecinos de $w$. Entonces
        
    \[
    \sum_{v \notin N(w)} d_G(v) \geq e(G)
    \]
    
    \noindent ya que cada \textit{no vecino} de $w$ cuenta al menos un extremo de un lado de $G$.
    
    Por otro lado, como cada vértice tiene grado a lo sumo $k$, entonces $e(G) \leq (n-k)k$. Pero $(n-k)k$ coincide con $e(K_{n-k,k})$. Al pasar un vértice del conjunto de tamaño $k$ al conjunto de tamaño $(n-k)$ se añaden $k-1$ lados y se retiran $n-k$, lo que da una ganancia neta de $2k-1-n$. Vemos entonces que ganamos lados cuando $2k > n + 1$ y perdemos cuando $2k < n + 1$, por lo que $e(K_{n-k,k})$ se maximiza cuando $k = \lfloor n/2 \rfloor$ o $k = \lceil n/2 \rceil$.
    
    De esta manera,
    
    \[
    e(G) \leq \lfloor n^2/4 \rfloor
    \]
\end{prob}

Como $K_{\lfloor n/2 \rfloor, \lceil n/2 \rceil}$ alcanza el máximo, hemos demostrado:

\begin{teo}[Mantel, 1972]
    La máxima cantidad de lados de un grafo simple de orden $n$ sin triángulos es $\lfloor n^2/4 \rfloor$.
\end{teo}

\subsection{Sucesiones Gráficas}
\stepcounter{subsec}

\begin{defn}
    La \ul{secuencia de grados} de un grafo $G$ es la lista de grados de sus vértices escrita en orden no-creciente: $d_1 \geq \dots \geq d_n$.
\end{defn}

Todo grafo $G$ tiene una secuencia de grados, ¿pero cuáles grafos aparecen al fijar una sucesión no-creciente de enteros no-negativos?

\begin{pro}
    Una sucesión no-creciente de enteros no negativos $d_1, \dots, d_n$ se corresponde con los grados de los vértices de un grafo $G$ sii $\sum d_i$ es par.
\end{pro}

\begin{proof}
    \begin{enumerate}
        \item[$\Leftarrow$] Supongamos que $\sum d_i$ es par. Entonces pasamos a construir un grafo $G$ con el conjunto de vértices $v_1, \dots, v_n$ donde $d(v_i) = d_i$, $\forall~i$. Como la suma es par, entonces la cantidad de vértices impares es par. Ahora emparejemos de forma abritraria a los vértices impares. Para cada pareja, asignemos un lado que tenga a cada vértice como uno de sus extremos. Para completar el grado de cada vértice, todavía nos hacen falta una cantidad par de lados por vértice. Esto lo logramos al colocar $\lfloor d_i/2 \rfloor$ bucles en el vértice $v_i$, $\forall~i$.
        
        \item[$\Rightarrow$] Supongamos que $G$ tiene la sucesión dada como su secuencia de grados. Entonces por \ref{teo:primer}, tenemos que $\sum d_i = 2e(G)$, el cual es un número par.
    \end{enumerate}
\end{proof}

Vemos que para esta demostración, hallar dicho $G$ es sencillo si se valen los bucles, pero si por ejemplo quisieramos construir un grafo con secuencia de grados $2, 0, 0$, esto es imposible sin bucles. De esta forma, si no tenemos bucles, la condición no es suficiente. Esta discusión nos da pie a la siguiente definición:

\begin{defn}
    Una \ul{secuencia gráfica} es una lista de números no-negativos tales que es la secuencia de grados de un grafo simple $G$. Un grafo simple con secuencia de grados $d$ \textit{realiza} $d$.
\end{defn}

\begin{ejem}
    \begin{marginfigure}
        \centering
        \begin{tikzpicture}
            \SetGraphUnit{1}
            \GraphInit[vstyle=Classic]
            \SetVertexNoLabel
            \Vertex{1}
            \Vertex[x=0, y=1]{2}
            \Vertex[x=1, y=0]{3}
            \Edge(1)(2)
        \end{tikzpicture}
        \caption{}
        \label{fig:sec1}
    \end{marginfigure}
    
    \begin{marginfigure}
        \centering
        \begin{tikzpicture}
            \SetGraphUnit{1}
            \GraphInit[vstyle=Classic]
            \SetVertexNoLabel
            \Vertex{1}
            \Vertex[x=0, y=1]{2}
            \Vertex[x=1, y=0]{3}
            \Vertex[x=1, y=1]{4}
            \Edge(1)(2)
            \SetUpEdge[style={dashed}]
            \Edge(3)(4)
            \Edge(2)(4)
        \end{tikzpicture}
        \caption{}
        \label{fig:sec2}
    \end{marginfigure}
    
    Las listas $2,2,1,1$ y $1,1,0$ son gráficas. El grafo $K_2 + K_1$ realiza $1,0,1$. Añadir un nuevo vértice adyacente a los grafos con grado $1$ y $0$ nos da como resultado un grafo con secuencia de grados $2,2,1,1$ como se muestra en \ref{fig:sec1} y \ref{fig:sec2}. De igual forma, si a un grafo que realiza $2,2,1,1$ tiene un vértice $w$ con vecinos de grados $2$ y $1$, tenemos como resultado un grafo con vértices $1,0,1$.
    
    De la misma manera, podemos experimentar con la secuencia $33333221$: Tendremos una realización con un vértice $w$ de grado $3$ con tres vértices de grado $3$. Esta existe si y sólo si $2223221$ es gráfica. Reordenando, probamos con $3222221$. Continuamos eliminando vértices y reordenando. Vemos que al insertar los vértices originales y sus lados respectivos, podemos llegar a una realización de la lista original. La situación descrita puede ser visualizada en la figura \ref{fig:sec3}.
    
    \begin{figure}
        \centering
        \begin{tikzpicture}[scale=0.9]
            \begin{scope}
                \SetGraphUnit{1}
                \GraphInit[vstyle=Classic]
                \SetVertexNoLabel
                \Vertex[x=0, y=1.5] {1}
                \Vertex[x=1, y=0]   {2}
                \Vertex[x=1, y=1.5] {3}
                \Vertex[x=1, y=3]   {4}
                \Vertex[x=2, y=0]   {5}
                \Vertex[x=2, y=1.5] {6}
                \Vertex[x=2, y=3]   {7}
                \Vertex[x=3, y=1.5] {8}
                \node[yshift=-0.4cm] at (0,1.5) {$v$};
                \node[xshift=0.4cm]  at (2,3)   {$u$};
                \node[yshift=-0.4cm] at (3,1.5) {$w$};
                \Edge(1)(2)
                \Edge(1)(3)
                \Edge(1)(4)
                \tikzset{EdgeStyle/.append style = {red}}
                \Edges[color=red](4,7,6)
                \tikzset{EdgeStyle/.append style = {blue}}
                \Edge[color=blue](3)(6)
                \Edge[color=blue](2)(5)
                \tikzset{EdgeStyle/.append style = {green}}
                \Edge[color=green](8)(7)
                \Edge[color=green](8)(6)
                \Edge[color=green](8)(2)
            \end{scope}
            \node at (1.5,-1) {$33333221$};
            \draw[<-, line width=2pt]   (3.5,1.5)   -- (4.5,1.5);
            \begin{scope}[xshift=5cm]
                \SetGraphUnit{1}
                \GraphInit[vstyle=Classic]
                \SetVertexNoLabel
                \Vertex[x=0, y=1.5] {1}
                \Vertex[x=1, y=0]   {2}
                \Vertex[x=1, y=1.5] {3}
                \Vertex[x=1, y=3]   {4}
                \Vertex[x=2, y=0]   {5}
                \Vertex[x=2, y=1.5] {6}
                \Vertex[x=2, y=3]   {7}
                \node[yshift=-0.4cm] at (0,1.5) {$v$};
                \node[xshift=0.4cm]  at (2,3)   {$u$};
                \Edge(1)(2)
                \Edge(1)(3)
                \Edge(1)(4)
                \tikzset{EdgeStyle/.append style = {red}}
                \Edges[color=red](4,7,6)
                \tikzset{EdgeStyle/.append style = {blue}}
                \Edge[color=blue](3)(6)
                \Edge[color=blue](2)(5)
            \end{scope}
            \node at (6,-1) {$3222221$};
            \draw[<-, line width=2pt]   (7.5,1.5)   -- (8.5,1.5);
            \begin{scope}[xshift=9cm]
                \SetGraphUnit{1}
                \GraphInit[vstyle=Classic]
                \SetVertexNoLabel
                \Vertex[x=0, y=0]   {2}
                \Vertex[x=0, y=1.5] {3}
                \Vertex[x=0, y=3]   {4}
                \Vertex[x=1, y=0]   {5}
                \Vertex[x=1, y=1.5] {6}
                \Vertex[x=1, y=3]   {7}
                \node[xshift=0.4cm]  at (1,3)   {$u$};
                \tikzset{EdgeStyle/.append style = {red}}
                \Edges[color=red](4,7,6)
                \tikzset{EdgeStyle/.append style = {blue}}
                \Edge[color=blue](3)(6)
                \Edge[color=blue](2)(5)
            \end{scope}
            \node at (9.5,-1) {$221111$};
            \draw[<-, line width=2pt]   (10.5,1.5)   -- (11.5,1.5);
            \begin{scope}[xshift=12cm]
                \SetGraphUnit{1}
                \GraphInit[vstyle=Classic]
                \SetVertexNoLabel
                \Vertex[x=0, y=0]   {2}
                \Vertex[x=0, y=1.5] {3}
                \Vertex[x=0, y=3]   {4}
                \Vertex[x=1, y=0]   {5}
                \Vertex[x=1, y=1.5] {6}
                \tikzset{EdgeStyle/.append style = {blue}}
                \Edge[color=blue](3)(6)
                \Edge[color=blue](2)(5)
            \end{scope}
            \node at (12.5,-1) {$11110$};
        \end{tikzpicture}
        \caption[]{Situación descrita en el ejemplo. Vemos que al eliminar vértices y reordenando, tenemos los grafos realizados por las secuencias.}
        \label{fig:sec3}
    \end{figure}
\end{ejem}

El siguiente teorema implica que este procedimiento recursivo efectivamente funciona.

\begin{teo}][Havel (1955), Hakimi (1962)]
    Para $n > 1$, una lista de enteros $d$ de tamaño $n$ es gráfica sii $d'$ es gráfica, donde $d'$ se obtiene al eliminar su elemento más grande $\Delta$ y sustrayendo $1$ de los siguientes $\Delta$ más grandes elementos. Además, la única sucesión gráfica de tamaño $1$ es $d_1 = 0$.
\end{teo}

\begin{proof}
    Para $n=1$, la demostración es trivial. Demostramos el si y sólo si en ambos sentidos:
    
    \begin{enumerate}
        \item[$\Leftarrow$] Sean $d$ con $d_1 \geq \dots \geq d_n$ y un grafo simple $G'$ con secuencia de grados $d'$. Añadimos un nuevo vértice a $G'$ adyacente a los vértices con grados $d_2 - 1, \dots, d_{\Delta+1}-1$. Estos $d_i$ son los $\Delta$ elementos más grandes de $d$ después de $\Delta$, por lo que el grafo resultante realiza $d$.
        
        \item[$\Rightarrow$] Ahora, consideremos un grafos imple $G$ que realiza a $d$. Queremos encontrar un grafo simple $G'$ que realice $d'$. Sea $w$ un vértice con grado $\Delta$ en $G$. Sea $S$ un conjunto de $\Delta$ vértices con el grado deseado $d_2, \dots, d_{\Delta + 1}$. Si $N(w) = S$, entonces eliminamos $w$ para obtener $G'$.
        
        Si esto no ocurre, entonces supongamos que existe un $x \in S - N(w)$ y $z \in N(w) - S$. Entonces $\Delta = d(x) \geq d(z)$. Así que hay un $y \in N(x) - N(z)$. Eliminamos $xy$ e $wz$ de $E(G)$ y en su lugar agregamos $wx$ e $zy$. De esta forma, preservamos el grado de los vértices en el grafo resultante $G^*$ con respecto a $G$. De ser necesario, repetimos este procedimiento las veces que sean necesarias hasta encontrarnos en la situación del párrafo anterior. Esto se puede hacer a lo sumo $\Delta$ veces (una por vecino de $w$).
    \end{enumerate}
    
    Y así, queda demostrado.
\end{proof}

\begin{defn}
    Lo que hicimos en el teorema anterior al intercambiar lados es lo que se conoce como un \ul{switch}. Más formalmente, un switch de un par de lados $xy$ e $zw$ en un grafo simple $G$ es el reemplazo de dichos lados por $xw$ e $yz$, dado que $xw$ e $yz$ no aparecían en el grafo original.
\end{defn}

Sobre los switches, se observa que

\begin{enumerate}
    \item Los switches preservan grados (y por ende, sucesiones gráficas).
    \item Si $H'$ se obtiene de $H$ mediante un switch, enotnces aplicar un switch a $H'$ con los mismos vértices produce $H$ nuevamente.
\end{enumerate}

Y esta definición nos inspira a estudiar el siguiente teorema:

\begin{teo}
    Sean $G,H$ grafos simples con $V(G) = V(H)$. $H$ se puede obtener a partir de $G$ aplicando switches de forma sucesiva sii para cada vértice $v$ se cumple $d_G(v) = d_H(v)$.
\end{teo}

\begin{proof}
    En primer lugar, como los switches preservan el grado de los vértices, la condición es necesaria.
    
    Ahora supongamos que $G$ y $H$ son grafos simples con $V(G) = V(H)$ y para cada vértice $v$ se cumple que $d_H(v) = d_G(v)$: Sea $n = V(G)$, para $n \leq 3$, sólo hay un grafo que realiza una sucesión de $3$ grados, por lo que al aplicar cualquier sucesión de switches, siempre vamos a obtener el mismo grafo.
    
    Procedamos a aplicar inducción en $n$ y supongamos que el resultado vale para $k < n$. Sea entonces $w$ un vértice con grado máximo $d_1$ y fijemos un conjunto $S$ de $d_1$ vértices (distintos de $w$) con los máximos grados. Como en el teorema anterior, aplicar switches lleva $G$ a otro grafo simple $G^*$ tal que $N_{G^*}(w) = S$. De forma análoga, esto lleva también a $H$ a $H^*$ tal que $N_{H^*}(w) = S$. Ahora, $G' = G^* - w$ y $H' = H^* - w$ tienen $n-1$ vpertices y para cada vértice $v$ se cumple que $d_{G'}(v) = d_{H'}(v)$. Por hipótesis inductiva, $H'$ se obtiene aplicando sucesivamente switches a $G'$. Entonces aplicar esos switches a $G^*$ resulta en $H^*$. De esta forma, aplicar switches se puede hacer $G \rightarrow G^* \rightarrow H^* \rightarrow H$.
    
    Y así, queda demostrado el teorema.
\end{proof}
\section{Grafos dirigidos}
\stepcounter{sec}

Hasta ahora hemos utilizado los grafos para modelar relaciones simétricas, pero en general este no tiene por qué ser el caso. Por ejemplo, en algunas redes se puede \textit{seguir} a alguien y éste no necesariamente te sigue. En estos casos decimos que la relación va en una \textit{dirección}. Esto se refleja gráficamente partiendo de la idea de un grafo: Solo hace galta agregar la dirección en los lados, lo que indica el orden de la relación.

\begin{defn}
    Un \ul{digrafo} o \ul{grafo dirigido}, es una tripleta $G = ( V(G), E(G), f_G)$ en el cual $V(G)$ y $E(G)$ son conjuntos (de vértices y lados) y $f_G: E(G) \rightarrow V(G) \times V(G)$.
    
    Si $(v,w) = f_G(e)$ para algún $e \in E(G)$, entonces decimos que $v$ es la \ul{cola} de $e$, y $w$ es la \ul{cabeza} de $e$. En este caso, escribimos $v \rightarrow w$ y también decimos que $v$ es \ul{predecesor} de $w$, y $w$ es \ul{sucesor} de $v$.
    
    Los lados cuyas imágenes tienen la forma $(v,v)$ son \ul{bucles} y si $f_G$ no es inyectiva, decimos que $G$ tiene lados \ul{múltiples}. Si $f_G$ es inyectiva, entonces decimos que $G$ es simple. Se observa que un digrafo simple $G$ puede tener bucles. Como antes, denotaremos $|V(G)| = n(G)$, $|E(G)| = e(G)$.
\end{defn}

\begin{ejem}
    \begin{marginfigure}
        % Falta hacer el digrafo
        \centering
        \includegraphics{img/digrafo.PNG}
        \caption{Ejemplo de digrafo.}
        \label{fig:digrafo}
    \end{marginfigure}
    
    Vemos en la figura \ref{fig:digrafo}~que $V(G) = [12]$, $E(G)$ está formado por las flechas y $f_G(e) = (t,h)$ si $h/t$ es primo. En este caso, el digrafo es simple.
\end{ejem}

\begin{ejem}
    \begin{marginfigure}
        % Falta hacer el digrafo
        \centering
        \includegraphics{img/markov.PNG}
        \caption{Cadena de Markov.}
        \label{fig:markov}
    \end{marginfigure}
    
    Los digrafos son útiles para modelar sistemas en los que los vértices representan estados, y los lados representan transición entre ellos. Un sistema cuyos estados cambian de forma aleatoria se llama \textbf{cadena de Markov}. Por ejemplo, en la figura \ref{fig:markov}~tenemos representados a los estados del tiempo con $G$ (bueno) y $B$ (malo). Las probabilidades son:
    
    \begin{enumerate}
        \item Que un día bueno siga siendo bueno: $0.8$.
        \item Que un día bueno se haga malo: $0.2$.
        \item Que un día malo siga siendo malo: $0.3$.
        \item Que un día malo se haga bueno: $0.7$.
    \end{enumerate}
\end{ejem}

\begin{defn}
    Dado un digrafo $D$, el \ul{grafo base} $G$ de $D$ se obtiene con los vértices de $D$ y considerando los lados sin tomar en cuenta el orden de los pares.
\end{defn}

\begin{defn}
    Un digrafo simple es un \ul{camino} si sus vértices se pueden ordenar de la forma $v_1, v_2, \dots, v_n$ con $v_i \rightarrow v_j$ si y sólo si $j = i+1$. En este caso decimos que se trata de un camino de $v_1$ a $v_n$.
    
    Un digrafo simple es un \ul{ciclo} si sus vértices se pueden ordenar de la forma $v_1, \dots, v_n$ con $v_i \rightarrow v_j$ si y sólo si $j = i+1$ ó $i=n$ y $j=1$.
    
    Otros conceptos visto para grafos son análogos para grafos:
    
    \begin{itemize}
        \item Subdigrafo.
        \item Isomorfismo (digrafos isomorfos, clases de isomorfismo).
        \item Descomposición de digrafos.
        \item Unión de digrafos.
    \end{itemize}
\end{defn}

Otros conceptos requieren modificaciones, o tienen algunas diferencias:

\begin{defn}
    Dado un digrafo $G$:
    
    \begin{itemize}
        \item Una \ul{matriz de incidencia} (si $G$ no tiene bucles) $M(G)$ está dada por $m_{ij} = 1$ si $x_i$ es la cola de $e_j$ y $m_{ij} = -1$ si $v_i$ es la cabeza de $e_j$.
        \item Una \ul{matriz de adyacencia} $A(G)$ está dada por $a_{ij} =$~cantidad de lados correspondientes a $v_i \rightarrow v_j$.
    \end{itemize}
    
    \begin{figure}
        \centering
        \includegraphics{img/matrices.PNG}
        \caption{}
        \label{fig:matrices}
    \end{figure}
\end{defn}

\begin{defn}
    Dados dos vértices $v,w$ de un digrafo, es posible que existe un camino de $v$ a $w$ y que no exista un camino de $w$ a $v$. Dado un digrafo $G$ con grafo base $G$, decimos que
    
    \begin{enumerate}
        \item $D$ es \ul{débilmente conexo} si $G$ es conexo.
        \item $D$ es \ul{conexo (o fuertemente conexo)} si para cada par ordenado de vértices $v,w$, existe un camino de $v$ a $w$ en $D$.
    \end{enumerate}
\end{defn}

\subsection{Grados y suceciones gráficas}
\stepcounter{subsec}

\begin{defn}
    Dados un digrafo $G$ y $v \in V(G)$,
    
    \begin{itemize}
        \item El \ul{grado de entrada} de $v$ en $G$ es la cantidad $d_G^-(v)$ de lados que tienen a $v$ como cabeza.
        \item El \ul{grado de salida} de $v$ en $G$ es la cantidad $d_G^+(v)$ de lados que tienen a $v$ como cola.
        \item El conjunto de \ul{predecesores} de $v$ es $N_G^-(v) \{w \in V(G) : w \rightarrow v\}$.
        \item El conjunto de \ul{sucesores} de $v$ es $N_G^+(v) \{w \in V(G) : v \rightarrow w\}$.
        \item $\delta^-(G) = \min \{d_G^-(w) : w \in V(G)\}$.
        \item $\Delta^-(G) = \max \{d_G^-(w) : w \in V(G)\}$.
        \item $\delta^+(G) = \min \{d_G^+(w) : w \in V(G)\}$.
        \item $\Delta^-(G) = \max \{d_G^+(w) : w \in V(G)\}$.
    \end{itemize}
\end{defn}

\begin{teo}
    En un digrafo $G$, tenemos que
    
    \[
    \sum_{v \in V(G)} d^+(v) = e(G) = \sum_{v \in V(G)} d^-(v)
    \]
\end{teo}

\begin{proof}
    Cada lado tiene exactamente una sola cola y una sola cabeza.
\end{proof}

Ahora, si quisiéramos definir sucesiciones (di)gráficas, tendrían que ser sucesiones de pares de números $(d_1^+,d_1^-), \dots, (d_n^+,d_n^-)$.

\begin{teo}
    Una lista de pares ordenados de enteros no negativos es realizable por un digrafo sii la suma de las primeras coordenadas es igual a la suma de las segundas coordenadas.
\end{teo}

\begin{proof}
    La condición es necesaria porque todo lado tiene una cola y una cabeza, contribuyendo cada uno exactamente uno a cada suma.
    
    Para ver la suficiencia, consideremos las parejas
    
    \[
    \{ (d_i^+, d_i^-) : 1 \leq i \leq n \}
    \]
    
    \noindent y los vértices $v_1, \dots, v_n$. Sea ahora $\sum_{j=1}^n d_j^- = \sum_{i=1}^n d_i^+ = m$. Consideremos los $n$ vértices y construyamos $m$ pares ordenados de la siguiente manera:
    
    Para cada $i$, se coloca $i$ en las primeras $d_i^+$ coordenadas. Para cada $j$, se coloca $-j$ en las primeras $d_j^-$ coordenadas. Para cada $(i,-j)$ agregamos un lado de $v_i$ a $v_j$. El digrafo resultante realiza $(d_1^+, d_1^-), \dots, (d_n^+, d_n^-)$.
    
    Así, queda demostrado el teorema.
\end{proof}

Para este teorema permitimos lados múltiples, ¿cómo podemos estudiar una versión análoga para digrafos simples?. Nos valdremos de una técnica que hace uso de los grafos bipartitos:

\begin{defn}
    El \ul{split} de un digrafo $D$ es un grafo bipartito $G$ cuyos conjuntos bipartitos $V^+$, $V^-$ son copias de $V(D)$. Para cada vértice $x \in V(D)$, tenemos un vértice $x^+ \in V^+$ y un vértice $V^- \in V^-$. Para cada lado $u,v \in E(D)$, hay un lados con extremos $u^+, v^- \in G$.
\end{defn}

\begin{figure}
    \centering
    \includegraphics{img/split.PNG}
    \caption{Ejemplo de split.}
    \label{fig:split}
\end{figure}

En la figura \ref{fig:split}~se observa que los grados en $G$ corresponden a los grados de salida y entrada en $D$.

Más aún, dado un grafo bipartito $G$ con bipartición $X$, $Y$ tal que $|X| = |Y| = n$, se puede revertir el proceso para obtener un digrafo $D$ tal que $G$ sea su split. Aquí se ve claro por qué permitimos bucles en los digrafos simples.

Entonces, dada $(d_1^+,d_1^-), \dots, (d_n^+,d_n^-)$ tal que la suma de las primeras coordenadas es igual a la suma de las segundas, existe un digrafo simple que la realiza si y sólo si existe un grafo $G$, bipartito con particiones $X$ e $Y$ tal que $|X| = |Y| = n$, donde los $d_j^+$ son grados en $X$ y los $d_j^-$ son los grados en $Y$.

\subsection{El digrafo de De Bruijin}
\stepcounter{subsec}

Es sencillo dar los conceptos de \textit{paseo} y \textit{recorrido} para digrafos. A partir de allí se puede definir de inmediato \textit{recorridos eulerianos}, \textit{circuitos eulerianos} y \textit{digrafos eulerianos}. De hecho, para obtener una caracterización de los digrafos eulerianos en función de los grados de los vértices, basta con un par de resultados sencillos.

\begin{lem}
    Si $G$ es un digrafo tal que $\delta^+(G) \geq 1$ ó $\delta^-(G) \geq 1$, entonces $G$ tiene un ciclo.
\end{lem}

\begin{proof}
    Supongamos que $G$ es un digrafo para el cual $\delta^+(G) \geq 1$. Sea $P$ un camino maximal en $G$. Si $u$ es el último vértice de $P$, $u$ tiene un sucesor en $P$. De esta manera, obtenemos un ciclo en $G$. Para demostrar cuando $\delta^-(G) \geq 1$, el razonamiento es totalmente análogo.
\end{proof}

\begin{teo}
    Un digrafo $G$ es euleriano sii para cada $v \in V(G)$ vale $d^+(G) = d^-(G)$ y el grado base de $G$ a lo sumo tiene una componente no trivial.
\end{teo}

\begin{proof}
    Queda como ejercicio. Se demuestra usando inducción y el lema anterior.
\end{proof}

Ahora, pasemos a observar la siguiente sucesión binaria:

\[
0000111101100101
\]

Esta cumple con lo siguiente:

\begin{enumerate}
    \item Cada bloque de $4$ dígitos consecutivos es distinto.
    \item Si esto se ve de forma cíclica, es aparente que están todos los $4$-bloques posibles.
\end{enumerate}

Esto es un ejemplo de lo que se conoce como \textbf{ciclo de De Bruijin}. En general, un ciclo de De Bruijin es una cadena de longitud $2^n$ tal que, vista de forma cíclica, contiene todos los $n$-bloques binarios distintos. Dado $n$, ¿existe un ciclo de De Bruijin para ese $n$? Si es así, ¿cómo se puede construir?

Volvamos al ejemplo que dimos anteriormente ($n=4$). Usaremos un digrafo $D_4$ construído de la siguiente manera:

\begin{enumerate}
    \item Los vértices de $D_4$ son cadenas binarias de $3$ dígitos.
    \item $u \rightarrow v$ si los últimos $2$ dígitos de $u$ coinciden con los primeros de $v$.
    \item Etiquetamos el lado $u \rightarrow v$ con el último dígito de $v$.
    \item $D_4$ es euleriano.
    \item Las etiquetas de los lados de cualquier circuito euleriano en $D_4$ forman un ciclo de De Bruijin.
\end{enumerate}

\begin{figure}
    \centering
    \includegraphics{img/de-bruijin.PNG}
    \caption{Ciclo de De Bruijin $D_4$.}
    \label{fig:de-bruijin}
\end{figure}

Vemos que en general, dado $n$ tenemos que los vértices de $D_n$ son las cadenas binarias con $n-1$ dígitos, si $u \rightarrow v$ los últimos $n-2$ dígitos coinciden con los primeros $n-2$ de $v$, y se etiqueta al lado $u \rightarrow v$ con el último dígito de $v$.

Esto nos motiva a el siguiente teorema:

\begin{teo}
    Para cada $n$, $D_n$ es euleriano y las etiquetas de cada ciclo euleriano en $D_n$ definen un diclo de De Bruijin.
\end{teo}

\begin{proof}
    Sea $D_n$. Entonces al agregar $0$ o $1$ al final de un vértice y eliminar el primero, obtenemos un sucesor, por lo que $d^+(v) = 2$ para todo $v \in V(D_n)$, y cada vértice es sucesor de otro que comienza en $1$ y otro que comienza en $0$, por lo tanto $d^-(v) = 2$. Entonces $d^+(v) = d^-(v) = 2$ para cada $v$. Más aún, dado un vértice $b_1b_2 \dots b_{n-1}$ podemos llegar a él desde cualquier otro vértice, siguiente los lados etiquetados con $b_1, b_2, \dots$. Por lo tanto $D_n$ es euleriano.
    
    Ahora, para llegar a un vértice $v = b_1b_2 \dots b_{n-1}$, debe haber sido a través de lados con etiquetas $b_1, \dots, b_{n-1}$. Si desde $v$ se usa un lado etiquetado con $b_n$, se obtiene la cadena $b_1b_2 \dots b_n$. Como los vértices son cadenas distintas, tenemos todas las cadenas binarias (distintas) de longitud $n$ usando un ciclo euleriano.
\end{proof}

\subsection{Orientaciones y torneos}
\stepcounter{subsec}

Para introducir el concepto de torneos, presentamos el siguiente problema:

\begin{prob}
    En varios deposrtes se hacen torneos tipo \textit{round robin}: cada equipo se enfrenta a otro sólo una vez y no hay empates. Si son $n$ equipos, ¿cuántos resultados son posibles en un torneo así?
    
    Podemos modelar el resultado con un digrafo:
    
    \begin{itemize}
        \item Cada equipo es un vértice.
        \item Todos los vértices son vecinos dos a dos.
        \item Si $x$ derrota a $y$ entonces escribimos $x \rightarrow y$.
        \item En cada juego hay dos posibilidades $x \rightarrow y$ ó $y \rightarrow x$.
    \end{itemize}
    
    \begin{marginfigure}
        \centering
        \begin{tikzpicture}
            \GraphInit[vstyle=Classic]
            \SetVertexNoLabel
            \grComplete[RA=2]{6}
            \tikzset{EdgeStyle/.append style = {red}}
            \Edge[style={->}, color=red](a0)(a1)
            \Edge[style={->}, color=red](a0)(a2)
            \Edge[style={->}, color=red](a0)(a4)
            \Edge[style={->}, color=red](a5)(a0)
            \Edge[style={->}, color=red](a3)(a0)
            \Edge[style={->}, color=red](a2)(a1)
            \Edge[style={->}, color=red](a1)(a3)
            \Edge[style={->}, color=red](a1)(a4)
            \Edge[style={->}, color=red](a5)(a1)
            \Edge[style={->}, color=red](a3)(a2)
            \Edge[style={->}, color=red](a2)(a4)
            \Edge[style={->}, color=red](a5)(a2)
            \Edge[style={->}, color=red](a4)(a5)
            \Edge[style={->}, color=red](a5)(a3)
            \Edge[style={->}, color=red](a3)(a4)
        \end{tikzpicture}
        \caption{Torneo de 6 vértices.}
        \label{fig:torneo}
    \end{marginfigure}
    
    Observémos las características del digrafo \ref{fig:torneo}~, no hay lados múltiples, no hay bucles y cada vértice es vecino de todos los demás.
\end{prob}

Todo esto nos permite establecer las siguientes definiciones:

\begin{defn}
    Una \ul{orientación} de un grafo $G$ es un digrafo $D$ obtenido a partir de $G$ al escoger una orientación $x \rightarrow y$ ó $y \rightarrow x$ para cada lado $xy \in E(G)$. Un \ul{grafo orientado} es una orientación de un grafo simple sin bucles. Un \ul{torneo} es una orientación de un grafo completo.
\end{defn}

\begin{prob}
    ¿En general, cuántos digrafos con $n$ vértices hay?: En primer lugar, hay $n^2$ pares ordenados de vértices (permitiendo bucles). Un digrafo simple permite bucles pero usa cada par ordenado a lo sumo una vez como lado, por lo tanto existen $2^{n^2}$ conjuntos de lados posibles.
    
    Por otro lado, modifiquemos el análisis para las orientaciones de un grafo simple con $n$ vértices: Hay $\binom{n}{2}$ pares no ordenados de vértices, y para cada par no ordenado de vértices $\{x,y\}$ hay tres posibilidades, son no adyacentes, $x \rightarrow y$ ó $y \rightarrow x$. Entonces hay en total $3^{\binom{n}{2}}$ orientaciones. Para torneos, sólo hay dos posibilidades, $x \rightarrow y$ ó $y \rightarrow x$. Por lo tanto hay $2^{\binom{n}{2}}$ torneos con $n$ vértices.
\end{prob}

En un torneo con $n$ vérticces es muy probable que más de un equipo tenga grado de salida máximo. ¿Cómo decidimos el campeón de un torneo? Una posibilidad es buscar un equipo $x$ con la propiedad:

\begin{enumerate}
    \item Dado un equipo $x$, $x$ derrota a $z$ o $x$ derrota a algún $y$ que derrota a $z$.
\end{enumerate}

Más formalmente, definimos:

\begin{defn}
    En un digrafo, un \ul{rey} es un vértice desde el cual cualquier otro vértice puede ser alcanzado mediante un camino de longitud a lo sumo $2$.
\end{defn}

Ahora, vale la pena preguntarse: Dado un torneo $T$, ¿existe un rey en $T$?

\begin{teo}[Landau, 1953]
    Todo torneo infinito tiene un rey.
\end{teo}

\begin{proof}
    Sea $x$ un vértice en un torneo $T$. Si $x$ no es un rey, entonces algún vértice $y$ no puede alcanzarse desde $x$ con un camino de a lo sumo longitud $2$. Por lo tanto ningún sucesor de $x$ es un predecesor de $y$. Como $T$ es una orientación de un clique, todo sucesor de $x$ debe ser sucesor de $y$, y además $y \rightarrow x$. Por lo tanto $d^+(y) > d^+(x)$.
    
    Si $y$ no es un rey, entonces repetimos el argumento para encontrar $z$ con un grado de salida más grande. Como $T$ es finito, este procedimiento termina. Y este termina solamente si hemos encontrado un rey.
\end{proof}

\end{document}