\section{Grafos y conteo}
\stepcounter{sec}

Algunas de las preguntas que ahora son relevantes y que contestaremos en esta sección son las siguientes:

\begin{itemize}
    \item ¿Cómo se relaciona la cantidad de lados de un grafo con la cantidad de vértices?
    \item ¿Cuántos subgrafos con características dadas, tiene un grafo?
    \item ¿Cuántas clases de isomorfismo tiene un grafo dado?
\end{itemize}

\begin{defn}
    Antes de hacer los análisis pertinentes, vale la pena recordar algunos conceptos y notación e introduciremos algunos más:
    
    \begin{enumerate}
        \item El \ul{grado} de $v \in V(G)$ es la cantidad de lados de $G$ incidentes con $v$ (si $v$ tiene bucles, estos contribuyen en $2$ al grado). Lo denotaremos por $d_G(v)$ ó $d(v)$.
        \item $\Delta(G)$ es el máximo de los grados en $G$.
        \item $\delta(G)$ es el mínimo de los grados en $G$.
        \item Si $\Delta(G) = \delta(G) = k$, decimos que $G$ es \ul{$k$-regular}.
        \item La \ul{vecindad} de $v \in V(G)$ es el conjunto $N_G(v)$ de los vecinos de $v$ en $G$.
        \item El \ul{orden} de $G$ es $n(G) = |V(G)|$.
        \item El \ul{tamaño} de $G$ es $e(G) = |E(G)|$.
        \item Dado $n \in \N$, denotamos $[n] = \{1, 2, \dots, n\}$.
    \end{enumerate}
\end{defn}

\subsection{Conteo y biyecciones}
\stepcounter{subsec}

Introduciremos esta subsección con un resultado bastante relevante: Si queremos calcular la suma de los grados de los vértices de $G$, cada lado se cuenta dos veces. Hemos demostrado el siguiente resultado:

\begin{teo}[Primer Teorema de la Teoría de Grafos]\label{teo:primer}
    Si $G$ es un grafo, entonces
    
    \[
    \sum_{v \in V(G)} d(v) = 2e(G)
    \]
\end{teo}

De este teorema se pueden sacar varios corolarios:

\begin{cor}
    En un grafo $G$, el promedio de los grados es
    
    \[
    \frac{2e(G)}{n(G)}
    \]
    
    Por lo tanto,
    
    \[
    \delta(G) \leq \frac{2e(G)}{n(G)} \leq \Delta(G)
    \]
\end{cor}

\begin{cor}
    $G$ tiene una cantidad par de vértices con grado impar.
\end{cor}

\begin{cor}
    Si $G$ es $k$-regular, entonces $e(G) = kn(G)/2$.
\end{cor}

\begin{cor}
    Si $n(G)$ es impar y $G$ es $k$-regular, entonces $k$ no puede ser impar.
\end{cor}

\begin{defn}
    El \ul{cubo $k$-dimensional} o \ul{hipercubo} $Q_k$, es el grafo simple cuyos vértices son las $k$-tuplas con entradas en $\{0,1\}$ y cuyos lados son las parejas de $k$-tuplas que difieren en \textbf{exactamente} una posición.
\end{defn}

\begin{figure}
    \centering
    \begin{tikzpicture}
        \SetGraphUnit{2}
        \SetVertexNoLabel
        \grCycle[rotation=45, prefix=a, RA=1]{4}
        \grCycle[x=3, y=1.5, rotation=45, prefix=b, RA=1]{4}
        \AssignVertexLabel{a}{000,100,110,010}
        \AssignVertexLabel{b}{001,101,111,011}
        \SetUpEdge[style={dashed}]
        \Edge(a0)(b0)
        \Edge(a1)(b1)
        \Edge(a2)(b2)
        \Edge(a3)(b3)
    \end{tikzpicture}
    \caption{Acá se puede apreciar una representación de $Q_3$}
    \label{fig:hipercubo}
\end{figure}

\begin{prob}[Estructura de los hipercubos]
    Con respecto a los hipercubos surjen muchísimas propiedades interesantes:
    
    \begin{itemize}
        \item Podemos establecer la \textit{paridad} de un vértice de $Q_k$ la siguiente manera: Si tiene una cantidad par de $1$s es par, y de lo contrario es impar. Cada lado de $Q_k$ incide en un vértice par y otro impar. De esta forma, los vértices pares son adyacentes únicamente a los impares, y viceversa. Por lo tanto ambos conjuntos de vértices forman un conjunto independiente y tenemos como resultado que $Q_k$ es bipartito.
        \item Cada posición en las $k$-tuplas se puede elegir de dos maneras, por lo tanto $n(Q_k) = 2^k$. Además, los vecinos de cada vértice se pueden obtener al cambiar una de las $k$ posiciones de la tupla, por lo tanto $Q_k$ es $k$-regular, y se tiene que $e(Q_k) = k2^{k-1}$.
        \item Vemos en la figura \ref{fig:hipercubo} que los lados no punteados son dos subgrafos de $Q_3$ isomorfos a $Q_2$, formados al mantener fijas la última coordenada de las tuplas. ¿De qué manera podemos generalizar este resultado?: Podemos formar un subcubo $j$-dimensional al fijar $k-j$ coordenadas y hacer que los valores en las $j$ sean escogidos a partir de una de las $2^j$ $j$-tuplas restantes. El subgrafo inducido por este conjunto de vértices es isomorfo a $Q_j$. Como hay $\binom{k}{j}$ maneras para escoger las $j$ coordenadas a variar, y $2^{k-j}$ maneras de escojer las coordenadas fijas y sus valores, en total tenemos $\binom{k}{j}2^{k-j}$ subcubos. Así, para cada $j \leq k$, $Q_k$ tiene $\binom{k}{j}2^{k-j}$ subgrafos isomorfos a $Q_j$.
    \end{itemize}
    
    % Maldita sea como me excitan los hipercubos
    Ahora, ¿cómo podemos construir un hipercubo $Q_k$?: Agregar $0$ a las tuplas de una copia de $Q_{k-1}$, agregar $1$ a las tuplas de otra copia de $Q_{k-1}$, y unir los vértices cuyas primeras $k-1$ coordenadas sean iguales. El resultado será $Q_k$.
\end{prob}

Vimos que un hipercubo es un grafo regular y bipartito, a continuación estableceremos una observación fundamental sobre esos grafos.

\begin{teo}
    Si $k > 0$, entonces un grafo $k$-regular y bipartito tiene la misma cantidad de vértices en cada partición.
\end{teo}

\begin{proof}
    Sea $G$ un grafo $k$-regular y bipartito. Supongamos que $\{A, B\}$ es la bipartición. Entonces cada lado de $G$ tiene un extremo en $A$, por lo tanto $e(G) = k|A|$. También cada lado de $G$ tiene un extremo en $B$, por lo tanto $e(G) = k|B|$. Esto implica que $|A| = |B|$.
\end{proof}

Otra técnica de conteo para conjuntos involucra establecer una biyección desde el conjunto a otro de un tamaño conocido. El siguiente problema utiliza este enfoque.

\begin{prob}
    ¿Cuántos $6$-ciclos tiene el grafo de Petersen?: Vimos anteriormente que el grafo de Petersen $P$ es $3$-regular. Por lo tanto tiene $10$ \textit{garras} (una por vértice). La figura \ref{fig:petersen} sugiere que a cada \textit{garra} le corresponde un $6$-ciclo. Si demostramos que este es el caso, tendremos que el grafo de Petersen tiene diez $6$-ciclos.
    
    \begin{marginfigure}
        \centering
        \begin{tikzpicture}
            \SetGraphUnit{1}
            \GraphInit[vstyle=Classic]
            \Vertex{w}
            \NO(w){u1}
            \SOEA(w){u2}
            \SOWE(w){u3}
            \Edge(w)(u1)
            \Edge(w)(u2)
            \Edge(w)(u3)
        \end{tikzpicture}
        \caption{Garra descrita en el problema.}
        \label{fig:garra}
    \end{marginfigure}
    
    Digamos que el ciclo $C$ es $(v1~v2~v3~v4~v5~v6)$. Como $P$ tiene cintura $5$, ningún $vj$ tiene más vecinos en $C$. Por lo tanto, cada $vj$ tiene exactamente un vecino fuera de $C$. Los vértices no adyacentes tienen exactamente un vecino en común $u1$, $u2$, $u3$, distintos entre sí ya que la cintura de $P$ es $5$. Solamente nos falta un vértice $w$ que sea vecino de $u1$, $u2$, $u3$. Por lo tanto esos vértices forman una garra con $w$.
    
    Veamos que una garra $G$ ocurre exactamente una vez quitando un $6$-ciclo: Sea $S$ el conjunto independiente en $G$. Cada $v \in S$ tiene sus otros vecinos fuera de $G$, y son distintos (ya lo discutimos). Entonces $P-G$ es $2$-regular y sus vértices son los seis a los que nos referimos en el párrafo anterior. Como la cintura de $P$ es $5$, entonces $P - G$ es un $6$-ciclo.
\end{prob}

Presentaremos una estrategia más de conteo: Los subgrafos de $G$ de la forma $G - v$ (con $v \in V(G)$) son útiles para hacer análisis, construcciones, demostraciones, etc. Recordemos que al retirar $v$, también estamos retirando los lados incidentes a este vértice. Por ejemplo, todos los subgrafos de la forma $G - v$ de $C_n$ son isomorfos a $P_n$.

\begin{pro}
    Sea $G$ un grafo simple con vértices $v_1, \dots, v_n$ y $n \geq 3$. Entonces
    
    \[
    e(G) = \dfrac{\sum e(G - v_i)}{n-2} \qquad \text{y} \qquad d_g(v_j) = \dfrac{\sum e(G - v_i)}{n-2} - e(G - v_j)
    \]
\end{pro}

\begin{proof}
    En primer lugar, un lado $e$ de $G$ está en $G - v_j$ sii $v_j$ no es extremo de $e$. Entonces $\sum e(G - v_i) = (n-2)e(G)$ ya que cada lado se cuenta $n-2$ veces.
    
    Además, para cada $j$ tenemos que $d_G(v_j)$ es el número de lados que se pierden al eliminar $v_j$ para obtener $G - v_j$. Entonces $d_G(v_j) = \frac{\sum e(G - v_i)}{n-2} - e(G - v_j)$.
    
    Y así, queda demostrado.
\end{proof}

El siguiente es un problema relacionado con subgrafos de la forma $G-v$:

\begin{prob}[Conjetura de Kelly, Ulam (1942)]
    Si $G$ es un grafo simple con orden al menos $3$, entonces $G$ está determinado de manera única por sus subgrafos de la forma $G - v$.
    
    Esta conjetura se conoce como la \textit{conjetura de la reconstrucción} ya que si tienes la lista de los subgrafos de la forma $G - e$ puedes reconstruir $G$ agregando vértices. En 2017 se dio una demostración.
\end{prob}

\subsection{Problemas extremos y optimización}
\stepcounter{subsec}

Un \textbf{problema extremo} pide el máximo o mínimo valor de una función definida sobre una clase de objetos. Por ejemplo, el máximo número de lados en un grafo simple con $n$ vértices es $\binom{n}{2}$.

Veamos un par de problemas más que derivaran en teoremas.

\begin{prob}
    ¿Cuál es el mínimo número de lados de un grafo conexo de orden $n$?:
    
    Vimos anteriormente que si $G$ tiene $n$ vértices y $k$ lados, entonces tiene al menos $n-k$ componentes\marginfootnote{Revisar la proposición \ref{pro:componentes}.}. Por lo que, si $k \leq n-2$ entonces $G$ no es conexo. Así que el mínimo buscado ha de ser mayor a $n - 2$.
    
    Más aún, la cota inferior es conseguida por el camino $P_n$.
\end{prob}

Así, hemos demostrado:

\begin{teo}
    El mínimo número de lados de un grafo conexo de orden $n$ es $n-1$.
\end{teo}

\begin{prob}
    ¿Cuál es el máximo valor de $\delta(G)$ para un grafo simple disconexo $G$?:
    
    Primero, sean $u, v \in V(G)$ no adyacentes. Ahora, por el teorema de inclusión y exclusión, la intersección de sus vecindades quedan de la siguiente manera:
    
    \[
    \left| N(u) \cap N(v) \right| = \left| N(u) \right| + \left| N(v) \right| - \left| N(u) \cup N(v) \right|
    \]
    
    Como $u, v$ no están en $N(u) \cup N(v)$, tenemos $\left| N(u) \cup N(v) \right| \leq n-2$. Por lo tanto, nos queda que
    
    \[
    \left| N(u) \cap N(v) \right| = \left| N(u) \right| + \left| N(v) \right| - n + 2
    \]
    
    Ahora nos preguntamos: ¿Cuál cota inferior será suficiente para que el lado derecho sea al menos $1$?
\end{prob}

\begin{teo}
    Si $G$ es un grafo simple de orden $n$ con $\delta(G) \geq (n-1)/2$ entonces $G$ es conexo.
\end{teo}

\begin{proof}
    Por la discusión anterior, tenemos que
    
    \[
    \left| N(u) \cap N(v) \right| = \left| N(u) \right| + \left| N(v) \right| - n + 2
    \]
    
    Como $\left| N(x) \right| \geq \delta(G) \geq (n-1)/2$, $\forall~x \in V(G)$ entonces lo anterior queda como
    
    \[
    \left| N(u) \cap N(v) \right| \geq 1
    \]
    
    Por lo que todo par $u, v \in V(G)$ no adyacentes está conectado. Por lo tanto $G$ es conexo.
\end{proof}

\begin{ejem}
    Sea $G$ no conexo, simple, de orden $n$ y con componentes isomorfas a la unión disjunta de $K_{\lfloor n/2 \rfloor}$ y $K_{\lceil n/2 \rceil}$. Como $\delta(G) = \lfloor n/2 \rfloor - 1$ y $G$ es no conexo, entonces $\delta(G)$ es efectivamente el máximo valor.
\end{ejem}

\begin{teo}
    El máximo valor de $\delta(G)$ entre los grafos simples, de orden $n$ y disconexos es $\lfloor n/2 \rfloor - 1$.
\end{teo}

\begin{notn}
    La unión disjunta de los grafos $H$ y $G$ también se conoce como \ul{suma} de $H$ y $G$ y se denota $H + G$.
\end{notn}

Otros problemas son los problemas de \textbf{optimización}, los cuales consisten en buscar objetos o cantidades extremas para un grafo dado en particular. Por ejemplo, el máximo tamaño de un subgrafo conexo, o de un conjunto de vértices independientes. Lo que sí suele darse en general para optimizar son algoritmos.

\begin{ejem}
    El siguiente algoritmo permite hallar un subgrafo bipartito \textit{grande} en el grafo sin bucles $G$:
    
    \begin{enumerate}
        \item Elige una bipartición $A$, $B$ para $G$.
        \item Llama $H$ al subgrafo de $G$ cuyos lados son los que tienen un extremo en $A$ y otro en $B$.
        \item Para cada $v \in V(G)$, si $d_H(v) < d_G(v)/2$, cambia a $v$ de conjunto bipartito.
    \end{enumerate}
    
    Ahora, se observa lo siguiente: Primero, $H$ es bipartito. Además el algoritmo termina porque $G$ es finito y en cada cambio se aumenta el tamaño de $H$. Por lo tanto, cuando el algoritmo termina, tennemos que $d_H(v) \geq d_G(v)/2$ para todo $v \in V(G)$. Sumando esto y aplicando la fórmula de la suma de los grados, nos queda que
    
    \[
    e(G) \geq e(G)/2
    \]
\end{ejem}

De esta forma, hemos demostrado el siguiente teorema:

\begin{teo}
    Si $G$ es un grafo finito sin bucles, entonces $G$ tiene un grafo bipartito con al menos $e(G)/2$ lados.
\end{teo}

\begin{nota}
    Este teorema no demuestra que el subgrafo bipartito tiene el máximo tamaño posible. De hecho, diferentes biparticiones iniciales pueden conducir a diferentes tamaños del subgrafo.
\end{nota}

\begin{prob}
    Dadas $n$ facciones en guerra, si ningun par de enemigos tiene un enemigo en común, ¿cuál es la cantidad máxima posible de pares de enemigos?:
    
    Esto equivale a determinar cuál es el máximo valor posible de $e(G)$ para un $G$ simple y de orden $n$ sin triángulos. Entonces sea $G$ como en la hipótesis, y consideremos $w \in V(G)$ con grado máximo $k$. Como no hay triángulos, no hay lados entre vecinos de $w$. Entonces
        
    \[
    \sum_{v \notin N(w)} d_G(v) \geq e(G)
    \]
    
    \noindent ya que cada \textit{no vecino} de $w$ cuenta al menos un extremo de un lado de $G$.
    
    Por otro lado, como cada vértice tiene grado a lo sumo $k$, entonces $e(G) \leq (n-k)k$. Pero $(n-k)k$ coincide con $e(K_{n-k,k})$. Al pasar un vértice del conjunto de tamaño $k$ al conjunto de tamaño $(n-k)$ se añaden $k-1$ lados y se retiran $n-k$, lo que da una ganancia neta de $2k-1-n$. Vemos entonces que ganamos lados cuando $2k > n + 1$ y perdemos cuando $2k < n + 1$, por lo que $e(K_{n-k,k})$ se maximiza cuando $k = \lfloor n/2 \rfloor$ o $k = \lceil n/2 \rceil$.
    
    De esta manera,
    
    \[
    e(G) \leq \lfloor n^2/4 \rfloor
    \]
\end{prob}

Como $K_{\lfloor n/2 \rfloor, \lceil n/2 \rceil}$ alcanza el máximo, hemos demostrado:

\begin{teo}[Mantel, 1972]
    La máxima cantidad de lados de un grafo simple de orden $n$ sin triángulos es $\lfloor n^2/4 \rfloor$.
\end{teo}

\subsection{Sucesiones Gráficas}
\stepcounter{subsec}

\begin{defn}
    La \ul{secuencia de grados} de un grafo $G$ es la lista de grados de sus vértices escrita en orden no-creciente: $d_1 \geq \dots \geq d_n$.
\end{defn}

Todo grafo $G$ tiene una secuencia de grados, ¿pero cuáles grafos aparecen al fijar una sucesión no-creciente de enteros no-negativos?

\begin{pro}
    Una sucesión no-creciente de enteros no negativos $d_1, \dots, d_n$ se corresponde con los grados de los vértices de un grafo $G$ sii $\sum d_i$ es par.
\end{pro}

\begin{proof}
    \begin{enumerate}
        \item[$\Leftarrow$] Supongamos que $\sum d_i$ es par. Entonces pasamos a construir un grafo $G$ con el conjunto de vértices $v_1, \dots, v_n$ donde $d(v_i) = d_i$, $\forall~i$. Como la suma es par, entonces la cantidad de vértices impares es par. Ahora emparejemos de forma abritraria a los vértices impares. Para cada pareja, asignemos un lado que tenga a cada vértice como uno de sus extremos. Para completar el grado de cada vértice, todavía nos hacen falta una cantidad par de lados por vértice. Esto lo logramos al colocar $\lfloor d_i/2 \rfloor$ bucles en el vértice $v_i$, $\forall~i$.
        
        \item[$\Rightarrow$] Supongamos que $G$ tiene la sucesión dada como su secuencia de grados. Entonces por \ref{teo:primer}, tenemos que $\sum d_i = 2e(G)$, el cual es un número par.
    \end{enumerate}
\end{proof}

Vemos que para esta demostración, hallar dicho $G$ es sencillo si se valen los bucles, pero si por ejemplo quisieramos construir un grafo con secuencia de grados $2, 0, 0$, esto es imposible sin bucles. De esta forma, si no tenemos bucles, la condición no es suficiente. Esta discusión nos da pie a la siguiente definición:

\begin{defn}
    Una \ul{secuencia gráfica} es una lista de números no-negativos tales que es la secuencia de grados de un grafo simple $G$. Un grafo simple con secuencia de grados $d$ \textit{realiza} $d$.
\end{defn}

\begin{ejem}
    \begin{marginfigure}
        \centering
        \begin{tikzpicture}
            \SetGraphUnit{1}
            \GraphInit[vstyle=Classic]
            \SetVertexNoLabel
            \Vertex{1}
            \Vertex[x=0, y=1]{2}
            \Vertex[x=1, y=0]{3}
            \Edge(1)(2)
        \end{tikzpicture}
        \caption{}
        \label{fig:sec1}
    \end{marginfigure}
    
    \begin{marginfigure}
        \centering
        \begin{tikzpicture}
            \SetGraphUnit{1}
            \GraphInit[vstyle=Classic]
            \SetVertexNoLabel
            \Vertex{1}
            \Vertex[x=0, y=1]{2}
            \Vertex[x=1, y=0]{3}
            \Vertex[x=1, y=1]{4}
            \Edge(1)(2)
            \SetUpEdge[style={dashed}]
            \Edge(3)(4)
            \Edge(2)(4)
        \end{tikzpicture}
        \caption{}
        \label{fig:sec2}
    \end{marginfigure}
    
    Las listas $2,2,1,1$ y $1,1,0$ son gráficas. El grafo $K_2 + K_1$ realiza $1,0,1$. Añadir un nuevo vértice adyacente a los grafos con grado $1$ y $0$ nos da como resultado un grafo con secuencia de grados $2,2,1,1$ como se muestra en \ref{fig:sec1} y \ref{fig:sec2}. De igual forma, si a un grafo que realiza $2,2,1,1$ tiene un vértice $w$ con vecinos de grados $2$ y $1$, tenemos como resultado un grafo con vértices $1,0,1$.
    
    De la misma manera, podemos experimentar con la secuencia $33333221$: Tendremos una realización con un vértice $w$ de grado $3$ con tres vértices de grado $3$. Esta existe si y sólo si $2223221$ es gráfica. Reordenando, probamos con $3222221$. Continuamos eliminando vértices y reordenando. Vemos que al insertar los vértices originales y sus lados respectivos, podemos llegar a una realización de la lista original. La situación descrita puede ser visualizada en la figura \ref{fig:sec3}.
    
    \begin{figure}
        \centering
        \begin{tikzpicture}[scale=0.9]
            \begin{scope}
                \SetGraphUnit{1}
                \GraphInit[vstyle=Classic]
                \SetVertexNoLabel
                \Vertex[x=0, y=1.5] {1}
                \Vertex[x=1, y=0]   {2}
                \Vertex[x=1, y=1.5] {3}
                \Vertex[x=1, y=3]   {4}
                \Vertex[x=2, y=0]   {5}
                \Vertex[x=2, y=1.5] {6}
                \Vertex[x=2, y=3]   {7}
                \Vertex[x=3, y=1.5] {8}
                \node[yshift=-0.4cm] at (0,1.5) {$v$};
                \node[xshift=0.4cm]  at (2,3)   {$u$};
                \node[yshift=-0.4cm] at (3,1.5) {$w$};
                \Edge(1)(2)
                \Edge(1)(3)
                \Edge(1)(4)
                \tikzset{EdgeStyle/.append style = {red}}
                \Edges[color=red](4,7,6)
                \tikzset{EdgeStyle/.append style = {blue}}
                \Edge[color=blue](3)(6)
                \Edge[color=blue](2)(5)
                \tikzset{EdgeStyle/.append style = {green}}
                \Edge[color=green](8)(7)
                \Edge[color=green](8)(6)
                \Edge[color=green](8)(2)
            \end{scope}
            \node at (1.5,-1) {$33333221$};
            \draw[<-, line width=2pt]   (3.5,1.5)   -- (4.5,1.5);
            \begin{scope}[xshift=5cm]
                \SetGraphUnit{1}
                \GraphInit[vstyle=Classic]
                \SetVertexNoLabel
                \Vertex[x=0, y=1.5] {1}
                \Vertex[x=1, y=0]   {2}
                \Vertex[x=1, y=1.5] {3}
                \Vertex[x=1, y=3]   {4}
                \Vertex[x=2, y=0]   {5}
                \Vertex[x=2, y=1.5] {6}
                \Vertex[x=2, y=3]   {7}
                \node[yshift=-0.4cm] at (0,1.5) {$v$};
                \node[xshift=0.4cm]  at (2,3)   {$u$};
                \Edge(1)(2)
                \Edge(1)(3)
                \Edge(1)(4)
                \tikzset{EdgeStyle/.append style = {red}}
                \Edges[color=red](4,7,6)
                \tikzset{EdgeStyle/.append style = {blue}}
                \Edge[color=blue](3)(6)
                \Edge[color=blue](2)(5)
            \end{scope}
            \node at (6,-1) {$3222221$};
            \draw[<-, line width=2pt]   (7.5,1.5)   -- (8.5,1.5);
            \begin{scope}[xshift=9cm]
                \SetGraphUnit{1}
                \GraphInit[vstyle=Classic]
                \SetVertexNoLabel
                \Vertex[x=0, y=0]   {2}
                \Vertex[x=0, y=1.5] {3}
                \Vertex[x=0, y=3]   {4}
                \Vertex[x=1, y=0]   {5}
                \Vertex[x=1, y=1.5] {6}
                \Vertex[x=1, y=3]   {7}
                \node[xshift=0.4cm]  at (1,3)   {$u$};
                \tikzset{EdgeStyle/.append style = {red}}
                \Edges[color=red](4,7,6)
                \tikzset{EdgeStyle/.append style = {blue}}
                \Edge[color=blue](3)(6)
                \Edge[color=blue](2)(5)
            \end{scope}
            \node at (9.5,-1) {$221111$};
            \draw[<-, line width=2pt]   (10.5,1.5)   -- (11.5,1.5);
            \begin{scope}[xshift=12cm]
                \SetGraphUnit{1}
                \GraphInit[vstyle=Classic]
                \SetVertexNoLabel
                \Vertex[x=0, y=0]   {2}
                \Vertex[x=0, y=1.5] {3}
                \Vertex[x=0, y=3]   {4}
                \Vertex[x=1, y=0]   {5}
                \Vertex[x=1, y=1.5] {6}
                \tikzset{EdgeStyle/.append style = {blue}}
                \Edge[color=blue](3)(6)
                \Edge[color=blue](2)(5)
            \end{scope}
            \node at (12.5,-1) {$11110$};
        \end{tikzpicture}
        \caption[]{Situación descrita en el ejemplo. Vemos que al eliminar vértices y reordenando, tenemos los grafos realizados por las secuencias.}
        \label{fig:sec3}
    \end{figure}
\end{ejem}

El siguiente teorema implica que este procedimiento recursivo efectivamente funciona.

\begin{teo}][Havel (1955), Hakimi (1962)]
    Para $n > 1$, una lista de enteros $d$ de tamaño $n$ es gráfica sii $d'$ es gráfica, donde $d'$ se obtiene al eliminar su elemento más grande $\Delta$ y sustrayendo $1$ de los siguientes $\Delta$ más grandes elementos. Además, la única sucesión gráfica de tamaño $1$ es $d_1 = 0$.
\end{teo}

\begin{proof}
    Para $n=1$, la demostración es trivial. Demostramos el si y sólo si en ambos sentidos:
    
    \begin{enumerate}
        \item[$\Leftarrow$] Sean $d$ con $d_1 \geq \dots \geq d_n$ y un grafo simple $G'$ con secuencia de grados $d'$. Añadimos un nuevo vértice a $G'$ adyacente a los vértices con grados $d_2 - 1, \dots, d_{\Delta+1}-1$. Estos $d_i$ son los $\Delta$ elementos más grandes de $d$ después de $\Delta$, por lo que el grafo resultante realiza $d$.
        
        \item[$\Rightarrow$] Ahora, consideremos un grafos imple $G$ que realiza a $d$. Queremos encontrar un grafo simple $G'$ que realice $d'$. Sea $w$ un vértice con grado $\Delta$ en $G$. Sea $S$ un conjunto de $\Delta$ vértices con el grado deseado $d_2, \dots, d_{\Delta + 1}$. Si $N(w) = S$, entonces eliminamos $w$ para obtener $G'$.
        
        Si esto no ocurre, entonces supongamos que existe un $x \in S - N(w)$ y $z \in N(w) - S$. Entonces $\Delta = d(x) \geq d(z)$. Así que hay un $y \in N(x) - N(z)$. Eliminamos $xy$ e $wz$ de $E(G)$ y en su lugar agregamos $wx$ e $zy$. De esta forma, preservamos el grado de los vértices en el grafo resultante $G^*$ con respecto a $G$. De ser necesario, repetimos este procedimiento las veces que sean necesarias hasta encontrarnos en la situación del párrafo anterior. Esto se puede hacer a lo sumo $\Delta$ veces (una por vecino de $w$).
    \end{enumerate}
    
    Y así, queda demostrado.
\end{proof}

\begin{defn}
    Lo que hicimos en el teorema anterior al intercambiar lados es lo que se conoce como un \ul{switch}. Más formalmente, un switch de un par de lados $xy$ e $zw$ en un grafo simple $G$ es el reemplazo de dichos lados por $xw$ e $yz$, dado que $xw$ e $yz$ no aparecían en el grafo original.
\end{defn}

Sobre los switches, se observa que

\begin{enumerate}
    \item Los switches preservan grados (y por ende, sucesiones gráficas).
    \item Si $H'$ se obtiene de $H$ mediante un switch, enotnces aplicar un switch a $H'$ con los mismos vértices produce $H$ nuevamente.
\end{enumerate}

Y esta definición nos inspira a estudiar el siguiente teorema:

\begin{teo}
    Sean $G,H$ grafos simples con $V(G) = V(H)$. $H$ se puede obtener a partir de $G$ aplicando switches de forma sucesiva sii para cada vértice $v$ se cumple $d_G(v) = d_H(v)$.
\end{teo}

\begin{proof}
    En primer lugar, como los switches preservan el grado de los vértices, la condición es necesaria.
    
    Ahora supongamos que $G$ y $H$ son grafos simples con $V(G) = V(H)$ y para cada vértice $v$ se cumple que $d_H(v) = d_G(v)$: Sea $n = V(G)$, para $n \leq 3$, sólo hay un grafo que realiza una sucesión de $3$ grados, por lo que al aplicar cualquier sucesión de switches, siempre vamos a obtener el mismo grafo.
    
    Procedamos a aplicar inducción en $n$ y supongamos que el resultado vale para $k < n$. Sea entonces $w$ un vértice con grado máximo $d_1$ y fijemos un conjunto $S$ de $d_1$ vértices (distintos de $w$) con los máximos grados. Como en el teorema anterior, aplicar switches lleva $G$ a otro grafo simple $G^*$ tal que $N_{G^*}(w) = S$. De forma análoga, esto lleva también a $H$ a $H^*$ tal que $N_{H^*}(w) = S$. Ahora, $G' = G^* - w$ y $H' = H^* - w$ tienen $n-1$ vpertices y para cada vértice $v$ se cumple que $d_{G'}(v) = d_{H'}(v)$. Por hipótesis inductiva, $H'$ se obtiene aplicando sucesivamente switches a $G'$. Entonces aplicar esos switches a $G^*$ resulta en $H^*$. De esta forma, aplicar switches se puede hacer $G \rightarrow G^* \rightarrow H^* \rightarrow H$.
    
    Y así, queda demostrado el teorema.
\end{proof}