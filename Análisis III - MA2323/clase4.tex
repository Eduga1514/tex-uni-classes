\begin{teo}[Teorema de la función implícita, generalización de la versión 3]
    Sea $F: \R^{n+m} \rightarrow \R^m$ diferenciable y de clase $C^2$, donde $F = (F_1, \dots, F_m)$ y a su vez $F_i(x_1, \dots, x_n, z_1, \dots, z_m)$. Consideremos $p_0 = (x_0, z_0) \in \R^{n+m}$. Consideremos también para cada $i = 1, \dots m$, $\Delta$ de la siguiente manera
    
    \[
    \Delta =
    \renewcommand\arraystretch{2}
    \begin{vmatrix}
        \frac{\partial F_1}{\partial z_1} & \dots & \frac{\partial F_1}{\partial z_m} \\
        \vdots & \ddots & \vdots \\
        \frac{\partial F_m}{\partial z_1} & \dots & \frac{\partial F_m}{\partial z_m}
    \end{vmatrix}
    \]
    
    \noindent donde cada derivada está evaluada en el punto $p_0$.
    
    Si tenemos que $F(p_0) = 0$ y $\Delta(p_0) \neq 0$, entonces en un entorno de $p_0$ tendremos que el sistema
    
    \[
    \begin{cases}
        F_1(x_1, \dots, x_n, z_1, \dots, z_m) = 0 \\
        \vdots \\
        F_m(x_1, \dots, x_n, z_1, \dots, z_m) = 0
    \end{cases}
    \]
    
    \noindent define de forma implícita a las funciones $z_i = f_i(x_1, \dots, x_n)$ con $i = 1, \dots, m$.
    
    Además, tendremos para $k = 1, \dots, n$ y $j = 1, \dots, m$
    
    \[
    \dfrac{\partial z_j}{\partial x_k} = -\dfrac{\dfrac{\partial(F_1, \dots, F_m)}{\partial(z_1, \dots, z_{j-1}, x_k, \dots, z_m)}}{\dfrac{\partial(F_1, \dots, F_m)}{\partial(z_1, \dots, z_m)}}
    \]
\end{teo}

A continuación no demostraremos este último teorema porque los cálculos se hacen muy extensos, pero si discutiremos algunas aplicaciones del T.F.I.:

\begin{ejem}
    Sean una función $F: \R^3 \rightarrow \R$ y $p_o = (x_0, y_0, z_0)$ tal que $F(p_0) = 0$. Sean también $\partial_xF$, $\partial_yF$ y $\partial_zF$ continuas en un entorno de $p_0$. Entonces $F(x,y,z) = 0$ define implícitamente $z = f(x,y)$. Más aún, se puede decir que
    
    \[
    \frac{\partial z}{\partial x}(p_0) = - \frac{\partial_xF(p_0)}{\partial_zF(p_0)}
    \qquad
    \frac{\partial z}{\partial y}(p_0) = - \frac{\partial_yF(p_0)}{\partial_zF(p_0)}
    \]
    
    \noindent si $\partial_zF(p_0) \neq 0$.
    
    Entonces, si calculamos la ecuación del plano tangente, esta nos queda como
    
    \begin{align*}
        z - z_0 &= \frac{\partial f}{\partial x}(x_0,y_0)(x-x_0) + \frac{\partial f}{\partial y}(x_0,y_0)(y-y_0) \\
            &= \partial_zF(p_0)(z-z_0) + \partial_yF(p_0)(y-y_0) + \partial_xF(p_0)(x-x_0) = 0
    \end{align*}
\end{ejem}

\begin{ejem}
    También es muy utilizado en el estudio de las ecuaciones diferenciales ordinarias exactas. Este tema lo abordaremos más adelante.
\end{ejem}

\subsection{Teorema de la función inversa}
\stepcounter{subsec}

Este teorema al igual que el teorema de la función implícita es central en el curso de análisis. Se irá desarrollando poco a poco porque la demostración es bien extensa.

Primero, veremos un lema que además de ser muy útil para la demostración de este teorema, también se utiliza en la materia de ecuaciones diferenciales.

\begin{lem}[Lema de contracción]
    Sean $M \subseteq \R^n$ cerrado y $F: M \rightarrow M$ una función y $K \in (0,1)$ tales que
    
    \[
    \normaeuc{F(x) - F(y)} \leq K\normaeuc{x - y} \quad \forall x,y \in M
    \]
    
    Entonces $F$ tiene un $x_0 \in M$ tal que $F(x_0) = x_0$. Luego nos queda que la sucesión $\sucinf{F^n(x_0)}{n}$ es de Cauchy y converge a $x_0$.
\end{lem}

\begin{proof}
    Primero tenemos que por hipótesis, para cada $x \in M$ fijo y arbitrario
    
    \[
    \normaeuc{F^2(x) - F(x)} = \normaeuc{F(F(x)) - F(x)} \leq K\normaeuc{F(x) - x}
    \]
    
    Ahora, por inducción podemos decir que
    
    \[
    \normaeuc{F^{n+1}(x) - F^n(x)} \leq K\normaeuc{F^n(x) - F^{n-1}(x)} \leq K^n\normaeuc{F(x) - x}
    \]
    
    En lo particular esto nos dice que $\sucinf{F^n(x_0)}{n}$ es una sucesión acotada, ya que
    
    \begin{align*}
        \normaeuc{F^n(x) - x} &\leq \normaeuc{F^n(x) - F^{n-1}(x)} + \normaeuc{F^{n-1}(x) - F^{n-2}(x)} + \dots \normaeuc{F(x) - x} \\
            &\leq \LaTeXunderbrace{(K^{n-1} + K^{n-2} + \dots + K)}_{\text{converge a $\frac{1}{1-K}$}}\normaeuc{F(x) - x}
    \end{align*}
    
    Nuevamente por inducción tenemos que para $m,k \in \N$
    
    \[
    \normaeuc{F^{m+k}(x) + F^m(x)} \leq K^m\normaeuc{F^k(x) + x}
    \]
    
    Ahora, como el término $F^k(x) - x$ está acotado, entonces existe $N_0 \in \N$ tal que $n,m \geq N_0$ con $n = m+k$, entonces
    
    \[
    \text{si $m,m \geq N_0$} \quad \implies \quad \normaeuc{F^{m+k}(x) + F^m(x)} < \varepsilon
    \]
    
    \noindent esto lo podemos hacer porque $\limtoinfty{m}{k^m} = 0$ ya que $K \in (0,1)$.
    
    Por lo tanto, $\sucinf{F^n(x_0)}{n}$ es de Cauchy.
    
    Ahora sea $x_0 \in M$ tal que $x_0 = \limtoinfty{n}{F^n(x)}$. Entonces dado $\varepsilon > 0$, existe $N_1$ tal que
    
    \[
    \text{si $n \geq N_1$} \quad \implies \quad \normaeuc{x_0 - F^n(x)} < \varepsilon
    \]
    
    Por lo tanto, si $n \geq N_1$
    
    \[
    \normaeuc{F(x_0) - F^{n+1}(x)} \leq K\normaeuc{x_0 - F^n(x)} < K\varepsilon
    \]
    
    De esta manera, $\limtoinfty{n}{F^n(x)} = F(x_0)$. En consecuencia $F(x_0) = x_0$.
    
    Lo único que queda es ver que $x_0$ es único: Supongamos que existe $x_1 \in M$ tal que $F(x_1) = x_1$. Consideremos ahora lo siguiente
    
    \begin{align*}
        \normaeuc{x_0 - x_1} &= \normaeuc{F(x_0) - F(x_1)} \leq K\normaeuc{x_0 - x_1} \\
        &\implies \normaeuc{x_0 - x_1} < K\normaeuc{x_0 - x_1}
    \end{align*}
    
    \noindent con $K \in (0,1)$. Esto es una contradicción ya que $\normaeuc{x_0 - x_1} > K\normaeuc{x_0 - x_1}$. Luego no existe dicho $x_1$ y concluimos que $x_0$ es único.
    
    Y así queda demostrado el lema de contracción, que próximamente utilizaremos para la demostración del lema de la función implícita.
\end{proof}