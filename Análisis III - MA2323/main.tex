\documentclass{tufte-handout}
\usepackage[utf8]{inputenc}
\usepackage{parskip} % Varios paquetes para símbolos y fuentes
\usepackage{amsmath}
\usepackage{amsthm}
\usepackage{mathtools}
\usepackage{physics}
\usepackage{tikz, pgfplots}
\usetikzlibrary{decorations.pathreplacing, calligraphy, arrows.meta}
\usepackage{titling} % Para estilizar el título
\usepackage{scrextend} % Añade márgenes para hacer bloques de texto
\usepackage{enumitem} % Para enumerar sin sangría
\usepackage{graphicx} % Maneja las imágenes
\usepackage{float}
\usepackage{wrapfig, subcaption}
\usepackage{lastpage}
\usepackage{fancyhdr} % Para hacer encabezados y pie de página más estilizados
\usepackage{color} % Para usar colores en el texto
\usepackage{soul} % Para subrayar con colores
\usepackage{soulutf8}
\usepackage{cancel} % Para tachar expresiones
\usepackage{titling} % Cambia los parámetros del título
\usepackage{chngcntr}

\theoremstyle{plain}
\newtheorem{teo}{Teorema}
\newtheorem{cor}[teo]{Corolario}
\newtheorem{lem}[teo]{Lema}
\newtheorem{pro}[teo]{Proposición}
\newtheorem{pre}{Pregunta}
\newtheorem{prop}{Propiedad}

\theoremstyle{definition}
\newtheorem{defn}{Definición}
\newtheorem{ejem}{Ejemplo}
\newtheorem{ejer}{Ejercicio}
\newtheorem{nota}{Notación}
\newtheorem{aco}{Acotación}

\renewcommand{\contentsname}{Contenido}
\renewcommand*{\proofname}{Demostración}
\renewcommand{\figurename}{Fig.}
\newcommand{\marginfootnote}[1]{\footnotemark\footnotetext{#1}}

\newcommand{\N}{\mathbb{N}}
\newcommand{\R}{\mathbb{R}}
\newcommand{\Oc}{\mathcal{O}}
\newcommand{\Pa}{\mathcal{P}}
\newcommand{\F}{\mathcal{F}}
\newcommand{\Rint}{R\big([a,b]\big)}
\newcommand{\sucinf}[2]{\left({#1}_{#2}\right)_{#2=1}^{\infty}}
\newcommand{\norma}[1]{\left\| #1 \right\|_{\infty}}
\newcommand{\normaeuc}[1]{\left\| #1 \right\|_2}
\newcommand{\normap}[2]{\left\| #1 \right\|_{#2}}
\newcommand{\limtoinfty}[2]{\lim_{{#1} \to \infty} #2}
\newcommand{\sumtoinfty}[2]{\sum_{#1}^{\infty} #2}
\newcommand{\TVM}{\hyperref[teo:1.1.1]{\textbf{T.V.M}}}
\newcommand{\TFIi}{\nameref{teo:3.1.1}}
\newcommand{\TFIii}{\nameref{teo:3.1.2}}
\newcommand{\TFIiii}{\nameref{teo:3.1.3}}
\newcommand{\TINV}{\nameref{teo:inversa}}
\newcommand{\LC}{\nameref{lem:4.1.1}}
\newcommand{\Taylor}{\nameref{teo:taylor}}
\newcommand{\Lagrange}{\nameref{teo:lagrange}}

% Establece el directorio para las imagenes
\graphicspath{ {img/} }

% Resetea el contador de ecuaciones en cada sección y/o subsección
\newcounter{sec}
\newcounter{subsec}
\setcounter{sec}{0}
\setcounter{subsec}{0}
\counterwithin*{equation}{sec}
\counterwithin*{equation}{subsec}

% Establecemos cómo será el encabezado y el pie de página
\fancyhf{}
\pagestyle{fancy}
\fancyhf{}
\fancyhead[L]{MA-2323}
\fancyhead[C]{Eduardo José Gavazut Pinto}
\fancyhead[R]{13-10524}
\fancyfoot[L]{Sección 1}
\fancyfoot[R]{Profesor: Iris López}
\fancyfoot[C]{\thepage\ de \pageref{LastPage}}
\renewcommand{\headrulewidth}{2pt} 
\renewcommand{\footrulewidth}{2pt}

\geometry{
	left=13mm, % left margin
	textwidth=130mm, % main text block
	marginparsep=8mm, % gutter between main text block and margin notes
	marginparwidth=55mm % width of margin notes
}

% Establece el subrayado de color rojo
\definecolor{ferrari}{rgb}{1,0.17,0}
\setulcolor{ferrari}

% Reduce el espacio entre el título y el header
\setlength{\droptitle}{-5.5em}
\renewcommand\maketitlehookc{\vspace{-3ex}}

% Define el espaciado entre párrafos
\setlength{\parskip}{1.5em}

% Definimos nuestro título
\pretitle{\begin{flushleft}\LARGE\sffamily}
\title{
Notas de Análisis III \\
Universidad Simón Bolívar
}
\posttitle{\par\end{flushleft}\vskip 0.5em}
\preauthor{\begin{flushleft}\large\scshape}
\author{
Eduardo Gavazut \\
Carnet: 13-10524}
\postauthor{\par\end{flushleft}}
\predate{\begin{flushleft}\large\scshape}
\date{Septiembre-Diciembre 2022}
\postdate{\par\end{flushleft}}

% Aquí empieza el documento
\begin{document}

\maketitle
\thispagestyle{fancy}

\tableofcontents
\break
\section*{Clase 1}

\subsection*{Definiciones Básicas}

\begin{defn}
    Sea $x$ una variable independiente, $y$ una variable dependiente de $x$ y $D^{(i)}$ (con $i \in \N - \{0, +\infty\}$) el operador derivación. Una relación
    
    \[
    \Phi\big( x, y, Dy, D^{(2)}y, \dots \big) = 0
    \]

    es llamada una \ul{ecuación diferencial}.

    Una ecuación diferencial se dice que es \ul{ordinaria} si el operador derivada es el operador derivada en una variable.
\end{defn}

\begin{defn}
    El orden de una EDO es el orden de la mayor derivada presente en la ecuación.
\end{defn}

\begin{defn}
    Una EDO se dice \ul{lineal} si es lineal en términos de las variables $y, y', y'', \dots, y^{(n)}, \dots$.

    En otro caso, se dice que la EDO es \ul{no lineal}.
\end{defn}

\begin{defn}
    Una solución de una EDO es una función $y = y(x)$ tal que

    \[
    \Phi\big( x, y(x), y'(x), \dots \big) = 0
    \]
\end{defn}

\subsection*{EDO de Orden 1}

\begin{defn}
    Una EDO de orden 1 es una ecuación diferencial de tipo

    \[
    \Phi\big( x, y, y' \big) = 0
    \]

    Si la EDO es lineal, entonces tiene la forma

    \[
    a_1(x)\frac{dy}{dx} + a_0(x)y = f(x), \quad a_1 \neq 0
    \]
\end{defn}

Pasaremos ahora a resolver la EDO general lineal de primer orden mediante el método de los \ul{factores integrantes}:

Consideremos la ecuación

\begin{equation}\label{eq:met_integrantes1}
    \frac{dy}{dt} + p(t)y = g(t)
\end{equation}

donde $p$ y $g$ son funciones cualesquiera. Multipliquemos esta expresión por una función $\mu(t)$ y nos queda

\begin{equation}\label{eq:met_integrantes2}
    \mu(t)\frac{dy}{dt} + \mu(t)p(t)y = \mu(t)g(t)
\end{equation}

Ahora, si consideramos $\mu$ tal que satisface

\begin{equation}\label{eq:met_integrantes3}
    \frac{d\mu(t)}{dt} = p(t)\mu(t)
\end{equation}

vemos que el lado izquierdo de \refeq{eq:met_integrantes2} es la derivada del producto $\mu(t)y$.

Si además asumimos que $\mu(t)$ es positivo entonces nos queda

\[
    \frac{d\mu(t)/dt}{\mu(t)} = p(t)
\]

La derivada del logaritmo es conocida, entonces por el TFC esto implica que

\[
    \ln\mu(t) = \int p(t)dt + k
\]

Escogiendo $k = 0$ obtenemos

\[
    \mu(t) = \exp \Bigg( \int p(t)dt \Bigg)
\]

Así, la ecuación \refeq{eq:met_integrantes2} equivale a

\[
    \frac{d}{dt}(\mu(t)y) = \mu(t)g(t)
\]

Y se sigue que

\[
    \mu(t)y = \int \mu(s)g(s) + c
\]

Y la solución general de \refeq{eq:met_integrantes1} es

\[
    y = \displaystyle \frac{\int \mu(s)g(s) + c}{\mu(t)}
\]

donde $\mu$ es el factor integrante de la ecuación.

Para aplicar este método vemos que necesitamos dos integraciones: Una para obtener $\mu$ y otra para obtener $y$.
\begin{teo}
    Sean $f: A \subseteq \R^n \rightarrow \R^m$ con $A$ abierto, conexo y convexo, y $x_0 \in A$. Si $\frac{\partial f_i}{\partial x_j}(x_0)$ existen y además $\frac{\partial f_i}{\partial x_j} \in C(A)$, entonces $f$ es diferenciable.
\end{teo}

\begin{proof}
    Por razones de simplicidad, solamente se demostrará el caso para $f: A \subset \R^n \rightarrow \R$ (el caso general se realiza tomando coordenada a coordenada). Queremos demostrar que
    
    \[
    \lim_{h \to 0} \frac{\left| f(x + h) - f(x) - \nabla f(x) \cdot h \right|}{\normaeuc{h}} = 0
    \]
    
    Evaluemos primero lo siguiente
    
    \begin{align*}
        f(x + h) - f(x) &= f(x_1 + h_1, \dots, x_n + h_n) \pm f(x_1, \dots, x_n + h_n) \\
            &\pm f(x_1, x_2, \dots, x_n + h_n) \pm \dots \pm f(x_1, x_2, \dots, x_{n-1}, x_n + h_n) \\
            &- f(x_1, \dots, x_n)
    \end{align*}
    
    Aplicando ahora el \TVM, en cada coordenada, nos queda que
    
    \[
    f(x + h) - f(x) = \frac{\partial f}{\partial x_1}(y_1)h_1 + \frac{\partial f}{\partial x_2}(y_2)h_2 + \dots + \frac{\partial f}{\partial x_n}(y_n)h_n
    \]
    
    Entonces podemos concluir que
    
    \begin{gather*}
        \big| f(x + h) - f(x) - \nabla f(x) \cdot h \big| = \Bigg| \sum_{j=1}^n \bigg( \frac{\partial f}{\partial x_j} (y_j) - \frac{\partial f}{\partial x_j} (x) \bigg) h_j \Bigg| \\
            \leq \sum_{j=1}^n \Bigg| \bigg( \frac{\partial f}{\partial x_j} (y_j) - \frac{\partial f}{\partial x_j} (x) \bigg) \Bigg| |h_j|
    \end{gather*}
    
    \noindent como $|h_j| \leq \normaeuc{h}$ (para todo $j$), entonces lo anterior queda así
    
    \begin{equation}\label{eq:2.1.1}
        \leq \left(\sum_{j=1}^n \Bigg| \bigg( \frac{\partial f}{\partial x_j} (y_j) - \frac{\partial f}{\partial x_j} (x) \bigg) \Bigg| |h_j|\right) \leq \sum_{j=1}^n \Bigg| \bigg( \frac{\partial f}{\partial x_j} (y_j) - \frac{\partial f}{\partial x_j} (x) \bigg) \Bigg| \normaeuc{h}
    \end{equation}
    
    Ahora, observemos que para cada $j = 1, \dots, n$, el teorema del valor medio garantiza la existencia de un $c_j \in (x_j, x_j + h_j)$ tal que $y_j = (x_1, \dots, c_j, \dots, x_n)$. Si $h \to 0$, entonces $h_j \to 0$ para todo $j = 1, \dots, n$. Por lo tanto $y_j \to x_j$.
    
    Como todas las derivadas parciales son continuas por hipótesis, obtenemos que
    
    \[
    \lim_{h_j \to 0} \frac{\partial f}{\partial x_j} (y_j) = \frac{\partial f}{\partial x_j} (x_j)
    \]
    
    Y por \ref{eq:2.1.1}, tenemos que si hacemos a $x_j \to 0$, nos queda que
    
    \[
    \left| f(x + h) - f(x) \right| \leq \normaeuc{h} \cancelto{0}{\sum_{j=1}^n \Bigg| \bigg( \frac{\partial f}{\partial x_j} (y_j) - \frac{\partial f}{\partial x_j} (x) \bigg) \Bigg|}
    \]
    
    Y en conclusión,
    
    \[
    \lim_{h \to 0} \frac{\left| f(x + h) - f(x) - \nabla f(x) h \right|}{\normaeuc{h}} = 0
    \]
\end{proof}

\subsection{Regla de la Cadena}
\stepcounter{subsec}
Lo que hemos hecho es probar un caso particular de un resultado que se le debe a F.P Ramsey. Enunciemos el teorema de Ramsey de la siguiente manera:

\begin{teo}[Ramsey (TR)]\label{teo:TR}
    Para todo $n \in \N$ y todo $k \in \N$
    
    \[
    \omega \rightarrow (\omega)_r^n
    \]
\end{teo}

\begin{proof}
    Procedamos igual que en la demostración anterior con un argumento inductivo: El caso $r=1$ es trivial, así que pasemos a demostrar el caso $r=2$ con inducción sobre $n$.
        
    Si $n=1$, nuevamente es trivial; si $n=2$, ya hemos hecho la demostración en \ref{pre:ramsay1}. Supongamos entonces que el teorema vale para $n$ y probemos para $\omega \rightarrow (\omega)_2^{n+1}$. Como $\N$ es un conjunto infinito numerable, estudiar sus coloraciones vale para cualquier conjunto infinito numerable, así que por conveniencia nos quedaremos estudiando $\N$. Luego, tendremos $f: \N \rightarrow 2$, y nuestro objetivo será conseguir un conjunto homogéneo para $f$.
    
    \begin{marginfigure}
        \centering
        \begin{forest}
            [$0$, for tree={grow=90}, red
                [$1$, red
                    [\vdots
                        [$n-1$, red
                            [$X_1$
                                [$Y_3$
                                    [\vdots]
                                ]
                                [$Y_2$
                                    [\vdots]
                                ]
                            ]
                            [$X_0$
                                [$Y_1$
                                    [\vdots]
                                ]
                                [$Y_0$
                                    [\vdots]
                                ]
                            ]
                        ]
                    ]
                ]
            ]
        \end{forest}
        \caption{Representación de la construcción del árbol construído en la demostración del teorema de Ramsey.}
        \label{fig:ramseyfig1}
    \end{marginfigure}
    
    Entonces, construyamos un árbol de la siguiente manera: Sabemos que los primero $n$ elementos tienen el mismo color, es decir $f(n-1) = i$, donde $i = 0$ o $i = 1$. Definamos entonces dos particiones de $\N$, $X_0$ y $X_1$ tales que
    
    \[
    t \in X_i \iff f\left( n \cup \{t\} \right) = i, \quad i \in \{0,1\}
    \]
      
    \noindent ahora, partiremos ambos conjuntos $X_0$ y $X_1$ de la siguiente manera: Sea $t_0 \in X_0$ el menor elemento de $X_0$ y consideremos $C_{t_0} = \pred(t_0) \cup \{t_0\}$, definiremos dos conjuntos $Y_0$ y $Y_1$ tales que
    
    \[
    t \in Y_i \iff f\left( x \cup \{t\} \right) = i, \quad \forall x \in C_{t_0}^{[n]}
    \]
    
    Supongamos que en este árbol que hemos estado construyendo, tenemos en el nivel $m$ a un elemento $s$, ¿cómo es la partición del conjunto al que pertenece?. Pues consideremos $C_{s} = \pred(t_0) \cup \{s\}$ y sean entonces $Z_0$, $Z_1$ dichas particiones definidas de esta manera:
    
    \[
    t \in Z_i \iff f\left( x \cup \{t\} \right) = i, \quad \forall x \in C_{s}^{[n]}
    \]
    
    Los sucesores inmediatos de $s$ son elegidos de tal forma que tomamos el menor de cada $Z_i$. Entonces cada elemento del nivel $m$ tiene a lo sumo $2^{\binom{m}{n}}$ sucesores inmediatos, ya que para cada elemento del nivel $m$, $\left| C_s^{[n]} \right| = \binom{m}{n}$ y tenemos 2 colores.
    
    Para cada uno de los pares de particiones que hemos construído, puede ocurrir que uno de ellos sea vacío, pero no que los dos sea vacío. Esto es así por la cardinalidad de $\N$.
    
    Como resultado, hemos construído inductivamente un árbol $T$ infinito donde cada nivel es finito, entonces por el teorema \ref{teo:arboles1}, existe una rama $R \subset T$ infinita. Ahora, sea $x \in R^{[n]}$, para todo $s > n$, $t > s$, por construcción tenemos que
    
    \[
    f\left(x \cup \{s\}\right) = f\left(x \cup \{t\}\right)
    \]
    
    Con esto, podemos definir una nueva partición $g: R^{[n]} \rightarrow 2$, donde
    
    \begin{gather*}
        g(x) = 0, \quad \text{si} \quad f(x) = 0 \\
        g(x) = 1, \quad \text{si} \quad f(x) = 1
    \end{gather*}
    
    Luego, recordemos que por hipótesis inductiva, $\omega \rightarrow (\omega)_2^{n}$ y $g$ cumple las condiciones para la hipótesis inductiva. De esta manera, hay un $H \subset R$ infinito tal que $H^{[n]}$ es monocromático para $g$, y por lo tanto monocromático para $f$.
    
    De esta forma, hemos demostrado el caso $r=2$. Supongamos ahora que el teorema es válido para $r \leq k$, es decir que $\omega \rightarrow (\omega)_r^n$ es válido para todo $n$. Sea ahora $f: \N^{[n]} \rightarrow k+1$, podemos definir otra partición auxiliar $G: \N^{[n]} \rightarrow 2$ definida como
    
    \begin{gather*}
        G(x) = 0, \quad \text{si} \quad f(x) = 0 \\
        G(x) = 1, \quad \text{si} \quad f(x) = 1
    \end{gather*}
    
    \noindent entonces, por hipótesis inductiva, $G$ tiene un conjunto homogéneo $H$ infinito. Si $G\left( H^{[n]} \right) = 0$, $H$ es homogéneo para $f$. Si $G\left( H^{[n]} \right) = 1$, entonces $f|H^{[n]}$ es una partición de $k$ partes, y por hipótesis inductiva existe un conjunto $H' \subseteq H$ tal que es infinito y homogéneo. Este conjunto $H'$ también es homogéneo para $f$.
\end{proof}

Con esto demostrado, podemos pasar a demostrar una consecuencia de carácter finita del teorema de Ramsey:

\begin{teo}[Teorema de Ramsey finito (TFR)]\label{teo:TRF}
    Dados números enteros positivos $n, r$ y $m$, existe un entero positivo $N$ tal que
    
    \[
    N \rightarrow (m)_r^n
    \]
\end{teo}

\marginnote{La versión que se usará durante el transcurso del curso es la verión infinita del \TR}.

\begin{proof}
    Supongamos que el teorema no se cumple, es decir que para cada $N \in \N$, existe una coloración $f_N: N^{[n]} \rightarrow k$ tal que $\forall x \in N^{[m]}$, $x^{[n]}$ \textbf{no} es homogéneo para $f_N$.
    
    Con esto, definamos una función $f: \N^{[n+1]} \rightarrow k$ tal que para cualquier sucesión de números naturales $\{a_0, a_1, \dots, a_n\}$ tenemos
    
    \[
    f\left( \{a_0, a_1, \dots, a_n\} \right) = f_{a_n} \left( \{a_0, a_1, \dots, a_{n-1}\} \right)
    \]
    
    \noindent es decir, $f$ asigna a una sucesión creciente de $n+1$ elementos, la coloración será la que le asigne $f_{a_n}$ a los primeros $n$ elementos de esa lista.
    
    Luego, el \hyperref[teo:TR]{TR} nos dice que existe un $H \subset \N$ infinito tal que $H$ es homogéneo para $f$, con $H = \{h_0, h_1, \dots \}$. De esta forma, un conjunto $h$ de $m$ elementos tal que $h \subset H$ es también homogéneo para $f_{h_m}$. Al hacer la restricción $f_N|f_{h_m}$, tenemos que $h$ es homogéneo. Esto es una contradicción, ya que habíamos supuesto que no hay homogéneos de tamaño $m$ para $f_N$.
    
    Por lo tanto, el teorema de Ramsey finito es cierto.
\end{proof}
% Aquí empieza la clase 4

\begin{teo}[Teorema de la función implícita, generalización de la versión 3]
    Sea $F: \R^{n+m} \rightarrow \R^m$ diferenciable y de clase $C^2$, donde $F = (F_1, \dots, F_m)$ y a su vez $F_i(x_1, \dots, x_n, z_1, \dots, z_m)$. Consideremos $p_0 = (x_0, z_0) \in \R^{n+m}$. Consideremos también para cada $i = 1, \dots m$, $\Delta$ de la siguiente manera
    
    \[
    \Delta =
    \renewcommand\arraystretch{2}
    \begin{vmatrix}
        \frac{\partial F_1}{\partial z_1} & \dots & \frac{\partial F_1}{\partial z_m} \\
        \vdots & \ddots & \vdots \\
        \frac{\partial F_m}{\partial z_1} & \dots & \frac{\partial F_m}{\partial z_m}
    \end{vmatrix}
    \]
    
    \noindent donde cada derivada está evaluada en el punto $p_0$.
    
    Si tenemos que $F(p_0) = 0$ y $\Delta(p_0) \neq 0$, entonces en un entorno de $p_0$ tendremos que el sistema
    
    \[
    \begin{cases}
        F_1(x_1, \dots, x_n, z_1, \dots, z_m) = 0 \\
        \vdots \\
        F_m(x_1, \dots, x_n, z_1, \dots, z_m) = 0
    \end{cases}
    \]
    
    \noindent define de forma implícita a las funciones $z_i = f_i(x_1, \dots, x_n)$ con $i = 1, \dots, m$.
    
    Además, tendremos para $k = 1, \dots, n$ y $j = 1, \dots, m$
    
    \[
    \dfrac{\partial z_j}{\partial x_k} = -\dfrac{\dfrac{\partial(F_1, \dots, F_m)}{\partial(z_1, \dots, z_{j-1}, x_k, \dots, z_m)}}{\dfrac{\partial(F_1, \dots, F_m)}{\partial(z_1, \dots, z_m)}}
    \]
\end{teo}

A continuación no demostraremos este último teorema porque los cálculos se hacen muy extensos, pero si discutiremos algunas aplicaciones del T.F.I.:

\begin{ejem}
    Sean una función $F: \R^3 \rightarrow \R$ y $p_o = (x_0, y_0, z_0)$ tal que $F(p_0) = 0$. Sean también $\partial_xF$, $\partial_yF$ y $\partial_zF$ continuas en un entorno de $p_0$. Entonces $F(x,y,z) = 0$ define implícitamente $z = f(x,y)$. Más aún, se puede decir que
    
    \[
    \frac{\partial z}{\partial x}(p_0) = - \frac{\partial_xF(p_0)}{\partial_zF(p_0)}
    \qquad
    \frac{\partial z}{\partial y}(p_0) = - \frac{\partial_yF(p_0)}{\partial_zF(p_0)}
    \]
    
    \noindent si $\partial_zF(p_0) \neq 0$.
    
    Entonces, si calculamos la ecuación del plano tangente, esta nos queda como
    
    \begin{align*}
        z - z_0 &= \frac{\partial f}{\partial x}(x_0,y_0)(x-x_0) + \frac{\partial f}{\partial y}(x_0,y_0)(y-y_0) \\
            &= \partial_zF(p_0)(z-z_0) + \partial_yF(p_0)(y-y_0) + \partial_xF(p_0)(x-x_0) = 0
    \end{align*}
\end{ejem}

\begin{ejem}
    También es muy utilizado en el estudio de las ecuaciones diferenciales ordinarias exactas. Este tema lo abordaremos más adelante.
\end{ejem}

\subsection{Teorema de la función inversa}
\stepcounter{subsec}

Este teorema al igual que el teorema de la función implícita es central en el curso de análisis. Se irá desarrollando poco a poco porque la demostración es bien extensa.

Primero, veremos un lema que además de ser muy útil para la demostración de este teorema, también se utiliza en la materia de ecuaciones diferenciales.

\begin{lem}[Lema de contracción]\label{lem:4.1.1}
    Sean $M \subseteq \R^n$ cerrado y $F: M \rightarrow M$ una función y $K \in (0,1)$ tales que
    
    \[
    \normaeuc{F(x) - F(y)} \leq K\normaeuc{x - y} \quad \forall x,y \in M
    \]
    
    Entonces $F$ tiene un $x_0 \in M$ tal que $F(x_0) = x_0$. Luego nos queda que la sucesión $\sucinf{F^n(x_0)}{n}$ es de Cauchy y converge a $x_0$.
\end{lem}

\begin{proof}
    Primero tenemos que por hipótesis, para cada $x \in M$ fijo y arbitrario
    
    \[
    \normaeuc{F^2(x) - F(x)} = \normaeuc{F(F(x)) - F(x)} \leq K\normaeuc{F(x) - x}
    \]
    
    Ahora, por inducción podemos decir que
    
    \[
    \normaeuc{F^{n+1}(x) - F^n(x)} \leq K\normaeuc{F^n(x) - F^{n-1}(x)} \leq K^n\normaeuc{F(x) - x}
    \]
    
    En lo particular esto nos dice que $\sucinf{F^n(x_0)}{n}$ es una sucesión acotada, ya que
    
    \begin{align*}
        \normaeuc{F^n(x) - x} &\leq \normaeuc{F^n(x) - F^{n-1}(x)} + \normaeuc{F^{n-1}(x) - F^{n-2}(x)} + \dots \normaeuc{F(x) - x} \\
            &\leq \LaTeXunderbrace{(K^{n-1} + K^{n-2} + \dots + K)}_{\text{converge a $\frac{1}{1-K}$}}\normaeuc{F(x) - x}
    \end{align*}
    
    Nuevamente por inducción tenemos que para $m,k \in \N$
    
    \[
    \normaeuc{F^{m+k}(x) + F^m(x)} \leq K^m\normaeuc{F^k(x) + x}
    \]
    
    Ahora, como el término $F^k(x) - x$ está acotado, entonces existe $N_0 \in \N$ tal que $n,m \geq N_0$ con $n = m+k$, entonces
    
    \[
    \text{si $m,m \geq N_0$} \quad \implies \quad \normaeuc{F^{m+k}(x) + F^m(x)} < \varepsilon
    \]
    
    \noindent esto lo podemos hacer porque $\limtoinfty{m}{k^m} = 0$ ya que $K \in (0,1)$.
    
    Por lo tanto, $\sucinf{F^n(x_0)}{n}$ es de Cauchy.
    
    Ahora sea $x_0 \in M$ tal que $x_0 = \limtoinfty{n}{F^n(x)}$. Entonces dado $\varepsilon > 0$, existe $N_1$ tal que
    
    \[
    \text{si $n \geq N_1$} \quad \implies \quad \normaeuc{x_0 - F^n(x)} < \varepsilon
    \]
    
    Por lo tanto, si $n \geq N_1$
    
    \[
    \normaeuc{F(x_0) - F^{n+1}(x)} \leq K\normaeuc{x_0 - F^n(x)} < K\varepsilon
    \]
    
    De esta manera, $\limtoinfty{n}{F^n(x)} = F(x_0)$. En consecuencia $F(x_0) = x_0$.
    
    Lo único que queda es ver que $x_0$ es único: Supongamos que existe $x_1 \in M$ tal que $F(x_1) = x_1$. Consideremos ahora lo siguiente
    
    \begin{align*}
        \normaeuc{x_0 - x_1} &= \normaeuc{F(x_0) - F(x_1)} \leq K\normaeuc{x_0 - x_1} \\
        &\implies \normaeuc{x_0 - x_1} < K\normaeuc{x_0 - x_1}
    \end{align*}
    
    \noindent con $K \in (0,1)$. Esto es una contradicción ya que $\normaeuc{x_0 - x_1} > K\normaeuc{x_0 - x_1}$. Luego no existe dicho $x_1$ y concluimos que $x_0$ es único.
    
    Y así queda demostrado el lema de contracción, que próximamente utilizaremos para la demostración del lema de la función implícita.
\end{proof}

Ahora podemos presentar el teorema de la función inversa. Recordar que de este teorema existen muchas versiones. En este curso nos centraremos en una sola.

\begin{teo}[Teorema de la función inversa]\label{teo:inversa}
    Sean $A \subset \R^n$ abierto, conexo y convexo. Sea $p_0 \in A$, $F: A \rightarrow \R^n$ de clase $C^1(A)$. Supongamos que $JF(p_0) \neq 0$.
    
    Entonces $F$ es $C^1$ invertible localmente en un entorno de $p_0$ y si $G$ es su inversa local, tenemos que $y = G(x)$ y $JG(y) = [JF(x)]^{-1}$.
\end{teo}

\begin{figure}
    \centering
    \includegraphics{img/funcion-inversa.PNG}
    \caption{Situación descrita en el teorema de la función inversa.}
\end{figure}

\begin{proof}
    Primero, denotaremos
    
    \[
    \lambda = JF(p_0)
    \]
    
    \noindent así, $\lambda$ define una transformación de $\R^n \rightarrow \R^n$ lineal, continua e invertible. 
    
    Sea ahora $\lambda^{-1} \cdot F(x)$\marginfootnote{No confundir, esto se refiere a la composición.}, derivando nos queda que
    
    \[
    \left( \lambda^{-1} \cdot F(x) \right)' = \lambda^{-1}JF(x)
    \]
    
    \noindent y si $x = p_0$ entonces
    
    \[
    \lambda^{-1}JF(p_0) = \lambda^{-1}\lambda = id
    \]
    
    Entonces, si obtenemos que $\lambda^{-1} \cdot F$ es localmente invervtible, concuimos que $F$ también lo es (ya que $F = \lambda\lambda^{-1}F$). Esto reduce el teorema al caso en que $JF(p_0) = id$.
    
    Sea ahora $q_0 = F(p_0)$ y $H(x) = F(x + p_0) - q_0$ (vemos que $H(0) = 0$). Por lo tanto, $H$ está definida en un abierto que contiene al valor $0$, ya que $F$ está definida en un abierto que contiene a $p_0$. Entonces bastará demostrar que $H$ es invertible (localmente) alrededor de $0$.
    
    Bajo todas estas consideraciones, bastará demostrar el teorema bajo tres simplificaciones relevantes:
    
    \[
    p_0 = 0, \qquad F(0) = 0, \qquad JF(0) = id
    \]
    
    % Aquí empieza la clase 5
    Sea $W(x) = x - F(x)$ donde $JW(x) = I - JF(x)$. Vemos que $JW(0) = 0$. Por continuidad de $F$, existe un valor $r > 0$ tal que
    
    \begin{equation}\label{eq:5.1.1}
        \text{si $\normaeuc{x} < 2r$} \quad \implies \quad \normap{JW(x)}{} \leq \frac{1}{2}
    \end{equation}
    
    \noindent donde $\normap{JW(x)}{} = \sup_{\normaeuc{u} \leq 1} \normaeuc{JW(x) \cdot u}$.
    
    Consideremos ahora $\psi(t) = W(c(t))$, donde $c(t) = tx + (1-t)0$ con $t \in [0,1]$ (en $c(0) = 0$ y en $c(1) = x$). Con esta parametrización, podemos aplicar \TVM~y existe $t_0 \in (0,1)$ tal que
    
    \begin{equation}\label{eq:5.1.2}
        \psi'(t_0) = \psi(1) - \psi(0)
    \end{equation}
    
    Ahora, por un lado tenemos que
    
    \[
    \psi'(t) = JW\left(c(t)\right)x \implies \psi'(t_0) = JW\left(c(t_0)\right)x
    \]
    
    Entonces $\psi(1) = W(x)$ y $\psi(0) = W(0) = 0$. Y por \ref{eq:5.1.1} y \ref{eq:5.1.2} tenemos que
    
    \begin{align*}
        \normaeuc{\psi(t_0)} &\leq \normaeuc{x}\normap{JW\Big( c(t_0) \Big)}{} \\
            &\leq \frac{\normaeuc{x}}{2}
    \end{align*}
    
    Esto nos permite concluir que $\normaeuc{W(x)} \leq \normaeuc{x}/2$. ¿Qué nos está diciendo esto? Que $W$ está definida de esta manera
    
    \[
    W: \overline{B(0,r)} \rightarrow \overline{B(0,r/2)} \quad \text{y} \quad W\left( B(0,r) \right) = B(0, r/2)
    \]
    
    Con la función así definida, ahora queremos asegurar que para dado $y \in \overline{B(0, r/2)}$, existe un único $x \in \overline{B(0,r)}$ tal que $F(x) = y$. Sea ahora $W_y(x) = y + W(x)$. Entonces aplicando nuevamente el \TVM~tendremos lo siguiente:
    
    \begin{align*}
        \normaeuc{W_y(x_1) - W_y(x_2)} &= \normaeuc{y + W(x_1) - y - W(x_2)} = \normaeuc{W(x_1) - W(x_2)} \\
            &\leq \frac{1}{2}\normaeuc{x_1 - x_2}
    \end{align*}
    
    \noindent con $x_1, x_2 \in \overline{B(0,r)}$. Al igual que antes, esto lo logramos con la ayuda de una función auxiliar $\psi(t) = W\left( c(t) \right)$ donde $c(t) = (1-t)x_1 + tx_2$ (con $t \in (0,1)$).
    
    De esta forma, estamos concluyendo que
    
    \[
    \normaeuc{W_y(x_1) - w_y(x_2)} \leq \frac{1}{2} \normaeuc{x_1 - x_2}, \qquad \forall x_1, x_2 \in \overline{B(0,r)}
    \]
    
    Y aplicando el \LC~, se sigue que $W_y$ tiene un único punto fijo $x$ tal que $W_y(x) = x$. Entonces esto implica que
    
    \begin{align*}
        x = y + W(x) \quad &\implies \quad x = y + x - F(x) \\
            &\implies \quad F(x) = y
    \end{align*}
    
    \noindent para un único valor $x \in \overline{B(0,r/2)}$.
    
    Ahora demostremos la otra parte del teorema. Consideremos el conjunto abierto
    
    \[
    U_1 = \{ x \in B(0,r) : \normaeuc{F(x)} < r/2\}
    \]
    
    Sea $V_1 = F(U_1)$. Así, $F: U \rightarrow V_1$ es inyectiva y podemos construir $G$ tal que $G: V_1 \rightarrow U_1$. Queremos ahora asegurar que $V_1$ es abierto y $G \in C^1$: Sea $x_1$ tal que $y_1 = F(x_1)$ con $\normaeuc{y_1} < r/2$. Sea también $y \in \R^n$ tal que $\normaeuc{y} < r/2$. Entonces hay un único $x \in \overline{B(0,r)}$ tal que $F(x) = y$ y expresando $x = x - F(x) - F(x)$, tendremos que
    
    \begin{align*}
        x - x_1 &= x - F(x) + F(x) - x_1 + F(x_1) - F(x_1) \\
            &= F(x) - F(x_1) + W(x) - W(x_1)
    \end{align*}
    
    Y tomando la norma, nos queda por desigualdad triangular lo siguiente
    
    \begin{align*}
        \normaeuc{x - x_1} &\leq \normaeuc{F(x) - F(x_1)} + \normaeuc{W(x) - W(x_1)} \\
            &\leq \normaeuc{F(x) - F(x_1)} + \frac{\normaeuc{x - x_1}}{2}
    \end{align*}
    
    De aquí obtenemos la siguiente desigualdad
    
    \begin{equation}\label{eq:5.1.3}
        \frac{\normaeuc{x-x_1}}{2} \leq \normaeuc{F(x_1) - F(x_2)}
    \end{equation}
    
    Así, si $y, y_1$ son cercanos entonces $x, x_1$ son también son cercanos. Por lo tanto, tenemos que por un lado
    
    \[
    \normaeuc{x} < r \wedge \normaeuc{F(x)} < \frac{r}{2} \quad \implies \quad x \in U_1 \text{~y por lo tanto $y \in V_1$}
    \]
    
    Esto nos dice que $B(0, r/2) \subset V_1$ y esto implica que $V_1$ es abierto en $\R^n$.
    
    Por otro lado, la desigualdad \ref{eq:5.1.3} implica que
    
    \[
    \frac{\normaeuc{G(y) - G(y_1)}}{2} \leq \normaeuc{y_1 - y}
    \]
    
    Y ahora se puede decir lo siguiente: Dado $\varepsilon > 0$, existe $\delta > 0$ tal que si $\delta > \varepsilon / 2$ entonces concluimos que $G$ es continua.
    
    Para demostrar que $G$ es diferenciable, recordemos que como $JF(x_1)$ es invertible, entonces podemos expresar
    
    \[
    F(x) - F(x_1) = JF(x_1)(x-x_1) + \Phi(x-x_1)(x-x_1)
    \]
    
    \noindent donde $\Phi$ es tal que $\lim_{x \to x_1} \Phi(x-x_1) = 0$. Esto se deduce de la definición de diferenciabilidad.
    
    Ahora, podemos calcular el siguiente factor
    
    \begin{align*}
        G(y) - G(y_1) &- \Big( JF(x_1) \Big)^{-1}(y-y_1) \\
            &= (x-x_1) - \Big( JF(x_1) \Big)^{-1}\Big( F(x) - F(x_1) \Big) \\
            &= (x-x_1) - \Big( JF(x_1) \Big)^{-1}\Big( JF(x_1)(x-x_1) + \Phi(x-x_1)(x-x_1) \Big) \\
            &= \Big( JF(x_1) \Big)^{-1}\Phi(x-x_1)(x-x_1) \\
            &= \Big( JF(x_1) \Big)^{-1}\Big( \Phi\big( G(y) - G(y_1) \big)(y-y_1) \Big)
    \end{align*}
    
    Luego,
    
    \[
    \normaeuc{G(y) - G(y_1) - [JF(x_1)]^{-1}(y-y_1)} \leq \normap{JF^{-1}(x_1)}{}\normaeuc{x-x_1}\normaeuc{\Phi(x-x_1)}
    \]
    
    Y por la desigualdad \ref{eq:5.1.3}, lo anterior nos queda como
    
    \[
    \normaeuc{G(y) - G(y_1) - [JF(x_1)]^{-1}(y-y_1)} \leq 2C\normaeuc{y_1 - y}\normaeuc{\Big( G(y_1) - G(y) \Big)}
    \]
    
    Como $G$ es continua, $y \to y_1$ entonces $G(y) \to G(y_1)$ y $\lim_{x \to x_1} \Phi(x-x_1) = 0$ entonces podemos acotar el factor $\normaeuc{\Big( G(y_1) - G(y) \Big)}$ por otro tan pequeño como queramos. Luego dado $\varepsilon > 0$, $\exists \delta_0 > 0$ tal que
    
    \[
    \text{si $\normaeuc{y - y_1} < \delta_0$} \quad \implies \quad \normaeuc{\Big( G(y_1) - G(y) \Big)} < \frac{\varepsilon}{2C}
    \]
    
    Por lo tanto, si $\normaeuc{y - y_1} < \delta_0$ entonces
    
    \[
    \dfrac{\normaeuc{G(y) - G(y_1) - [JF(x_1)]^{-1}(y-y_1)}}{\normaeuc{y - y_1}} \leq \varepsilon
    \]
    
    En conclusión estamos diciendo que $G$ es diferenciable y además $JG(y_1) = [JF(x_1)]^{-1}$.
    
    Y así queda demostrado el teorema de la función inversa.
\end{proof}
\section{Teorema Fundamental del Cálculo y sus aplicaciones}

Ahora, desarrollaremos los resultados para establecer el Teorema Fundamental del Cálculo (TFC). Este teorema fue establecido por Leibniz y es el resultado que nos permite relacionar la teoría de derivadas con la teoría de integración. Este es el teorema central de la teoría de integración.

\begin{lem}
    Sea $f: [a,b] \rightarrow \R$ y $c \in (a,b)$. Si $f$ es integrable en $[a,c]$ y $[c,b]$ entonces $f$ es integrable en $[a,b]$.
\end{lem}

\begin{proof}
    Como la función es integrable en $[a,c]$, dado $\varepsilon > 0$, entonces existe una partición $P_1$ tal que
    
    \[
    U(f, P_1) - L(f, P_1) < \varepsilon/2
    \]
    
    \noindent análogamente, como $f$ es integrable en $[c,b]$, dado $\varepsilon > 0$, existe una partición $P_2$ tal que
    
    \[
    U(f, P_2) - L(f, P_2) < \varepsilon/2
    \]
    
    Ahora, definamos una partición $\Pe = P_1 \cup P_2$, sabemos que esta partición abarca al intervalo $[a,b]$ en su totalidad. Ahora, por \ref{eq:supamasb}, tenemos que
    
    \begin{gather*}
        U(f, \Pe) = U(f, P_1) + U(f, P_2) \\
        L(f, \Pe) = L(f, P_1) + L(f, P_2)
    \end{gather*}
    
    Así, tenemos que
    
    \[
    U(f, \Pe) - L(f, \Pe) = U(f, P_1) - L(f, P_1) + U(f, P_2) - L(f, P_2) < \varepsilon
    \]
    
    \noindent en consecuencia, la función $f$ será integrable en $[a,b]$.
\end{proof}

\begin{teo}
    Sea $f: [a,b] \rightarrow \R$ tal que $f \in \Rint$ y continua en $[a,b]$. Si $F(x) = \int_a^x f(t)dt$ para cada $x \in [a,b]$, entonces $F$\marginfootnote{Establecer esta función nos lleva a la siguiente definición:
    
    \begin{defn}
        Esta función $F$ definida de esta manera se conoce como la \ul{antiderivada} de $f$.
    \end{defn}} es continua en $[a,b]$.
\end{teo}

\begin{proof}
    Sea $\varepsilon > 0$, y los puntos $x_0, x$ en $[a,b]$ ($x > x_0$). Ahora, si queremos ver que $F$ es continua, para dicho $\varepsilon$ hemos de encontrar un $\delta$ tal que se satisfaga lo siguiente:
    
    \[
    \text{si} \quad |x - x_0| < \delta, \qquad \text{entonces} \quad |F(x) - F(x_0)| < \varepsilon
    \]
    
    Ahora, por el lema que acabamos de demostrar, y el teorema \ref{teo:riemod} tenemos
    
    \begin{align*}
        |F(x) - F(x_0)| &= \left| \int_a^x f(t)dt - \int_a^{x_0} f(t)dt \right| = \left| \int_a^x f(t)dt - \int_a^{x_0} f(t)dt \right| \\
        &= \left| \int_a^{x_0} f(t)dt + \int_{x_0}^x f(t)dt - \int_a^{x_0} f(t)dt \right| = \left| \int_{x_0}^x f(t)dt \right| \\
        &\leq \int_{x_0} |f(t)|dt
    \end{align*}
    
    \noindent como $f$ es continua y acotada en $[a,b]$, será acotada en $[x_0, x]$. Por lo tanto existe $M > 0$ tal que
    
    \[
    |F(x) - F(x_0)| \leq M\int_{x_0}^xdt \quad \text{con} \quad \sup_{x\in[a,b]} |f(x)|
    \]
    
    \noindent pero $M\int_{x_0}^xdt = M(x-x_0)$. Entonces, al escoger $\delta = \varepsilon/M$, tendremos que
    
    \[
    |F(x) - F(x_0)| \leq M(x-x_0) \leq M\delta = \epsilon
    \]
    
    De esta manera, queda demostrado.
\end{proof}

\begin{teo}[Primer Teorema Fundamental del Cálculo]\label{teo:1TFC}
    Sea una función $f: [a,b] \rightarrow \R$ con $f \in \Rint$ y continua. Entonces
    
    \[
    F(x) = \intab f(x)dx \quad \text{es derivable}
    \]
    
    Mas aún, $F'(x) = f(x)$.
\end{teo}

\begin{proof}
    En un principio, por definición,
    
    \[
    F'(x) = \lim_{x \to x_0} \frac{F(x) - F(x_0)}{x_0}
    \]
    
    \noindent es decir, que dado $\delta > 0$, queremos hallar un $\varepsilon > 0$ tal que
    
    \[
    \text{si} \quad |x - x_0| < \delta \quad \text{entonces} \left| \frac{F(x) - F(x_0)}{x-x_0} - f(x_0) \right| < \varepsilon
    \]
    
    \noindent entonces, estimemos cuánto da este último factor
    
    \[
    \left| \frac{F(x) - F(x_0)}{x-x_0} - f(x_0) \right| = \left| \frac{F(x) - F(x_0) - f(x_0)(x-x_0)}{x-x_0} \right|
    \]
    
    Por la definición de $F$, y el lema que acabamos de demostrar, tenemos
    
    \[
    \left|\dfrac{F(x) - F(x_0) - f(x_0)(x-x_0)}{x-x_0}\right| = \dfrac{\left| \int_{x_0}^x f(t)dt - \int_{x_0}^x f(x_0)dt  \right|}{|x-x_0|} \leq \dfrac{\int_{x_0}^x |f(t)dt - f(x_0)|}{|x-x_0|}
    \]
    
    Ahora, por la continuidad de $f$, dado $\epsilon > 0$, existe un $\delta_f > 0$ tal que si $|x-x_0| < \delta_f$ entonces $|f(x)-f(x_0)| < \epsilon$. Entonces
    
    \[
    \dfrac{\int_{x_0}^x |f(t)dt - f(x_0)|}{|x-x_0|} \leq \frac{\epsilon}{|x-x_0|}\int_{x_0}^xdt = \epsilon
    \]
    
    Por lo tanto, concluímos que si $|x-x_0| < \delta_f$, tenemos que
    
    \[
    \left| \frac{F(x) - F(x_0)}{x-x_0} - f(x_0) \right| < \varepsilon
    \]
    
    De esta manera, basta fijar $\delta = \delta_f$ para concluir que $F'(x_0) = f(x)$. Y así queda demostrado el teorema.
\end{proof}

\begin{teo}[Segundo Teorema Fundamental del Cálculo]\label{teo:2TFC}
    Sean $f \in C[a,b]$, $F$ tal que $F'(x) = f(x)$. Entonces
    
    \[
    \intab f(x)dx = F(b) - F(a)
    \]
\end{teo}

\begin{proof}
    Sea $G(x) = \int_a^x f(t)dt$, y por el 1TFC, obtenemos que $G'(x) = f(x)$. Pero por otro lado, también tenemos que $F'(x) = f(x)$. Como $G'(x) = F'(x)$ entonces
    
    \[
    G(x) - F(x) = k, \quad \text{con } k \in \R
    \]
    
    \noindent pero sabemos a qué equivale $G$, luego
    
    \[
    \int_a^x f(t)dt - F(x) = k \implies \cancelto{0}{\int_a^af(t)dt} - F(a) = k \implies -F(a) = k
    \]
    
    De esta manera,
    
    \begin{align*}
        \int_a^x f(t)&dt - F(x) = -F(a) \implies \intab f(t)dt - F(b) = -F(a) \\
        &\implies \intab f(t)dt = F(a) - F(b)
    \end{align*}
    
    \noindent y así queda demostrado el teorema.
\end{proof}

La primera aplicación importante que veremos de estos teoremas es una bastante usada en los cursos de cálculo:

\begin{teo}[Cambio de Variable]
    Sean $I_1, I_2 \subset \R$, y sean $f: I_1 \rightarrow I_2$ tal que $f \in C^1(I_1)$\marginfootnote{Aquí estamos manejando la siguiente notación:
    
    \begin{nota}
        Sea $f$ una función, e $I$ un intervalo cualquiera. Decir que $f \in C^1(I_1)$ equivale a pedir que la función sea continua, derivable y que su derivada sea continua sobre el intervalo $I$.
    \end{nota}}, $g: I_2 \rightarrow \R$ tal que $g \in C(I_2)$. Entonces
    
    \[
    \intab g\left(f(t)\right)f'(t)dt = \int_{f(a)}^{f(b)} g(u)du
    \]
\end{teo}

\begin{proof}
    Sea $G$ derivable en $I_2$ tal que $G'(x) = g(x)$. Luego, por 1TFC tenemos que $G(x) = \int_a^x g(u)du$, y por el 2TFC, podemos decir que
    
    \begin{equation}\label{eq:cl6.1}
        G(f(b))-G(f(a)) = \int_{f(a)}^{f(b)} g(u)du \quad \text{porque $G'(x) = g(x)$ para cada $x$}
    \end{equation}
    
    Por otro lado, la regla de la cadena nos dice que
    
    \[
    \left[ G(f(t)) \right]' = G'(f(t))f'(t)
    \]
    
    \noindent entonces, aplicando nuevamente el 2TFC,
    
    \begin{equation}\label{eq:cl6.2}
        \intab \left[ G(f(t)) \right]'dt = G(f(b)) - G(f(a))
    \end{equation}
    
    Luego, por \ref{eq:cl6.1} y \ref{eq:cl6.2} tenemos que
    
    \[
    \int_{f(a)}^{f(b)} g(u)du = \intab \left[ G(f(t)) \right]'dt = \intab g(f(t))f'(t)dt
    \]
    
    De esta forma, queda demostrado.
\end{proof}

\begin{teo}[Integración por partes]
    Sean $f, g \in C^1[a,b]$. Entonces
    
    \[
    \intab f(t)g'(t)dt = f(b)g(b) - f(a)g(a) - \intab f'(t)g(t)dt
    \]
\end{teo}

\begin{proof}
    Sabemos que
    
    \[
    [f(t)g(t)]' = f'(t)g(t) + g'(t)f(t)
    \]
    
    También sabemos por el 2TFC que
    
    \[
    \intab (f(t)g(t))'dt = f(b)g(b) - f(a)g(a)
    \]
    
    Por otro lado,
    
    \[
    \intab (f(t)g(t))'dt = \intab f'(t)g(t)dt + \intab g'(t)f(t)dt
    \]
    
    De esta forma, despejando y sustituyendo nos queda
    
    \begin{gather*}
        \intab g'(t)f(t)dt = \intab (f(t)g(t))'dt - \intab f'(t)g(t)dt \\
        \implies \intab g'(t)f(t)dt = f(b)g(b) - f(a)g(a) - \intab f'(t)g(t)dt
    \end{gather*}
    
    Así, queda demostrado el teorema.
\end{proof}
\section{Clase 7}
\subsection{Introducción a Ecuaciones diferenciales de Orden Superior}

Una EDO de orden $n$ es una ecuación de la forma

\[
    P_0(t)\frac{d^ny}{dt^n} + \dots + P_{n-1}(t)\frac{dy}{dt} + P_n(t)y = G(t)
\]

Donde las funciones $P_0, \dots, P_n, G$  son reales y continuas en algún intervalo $I: \alpha < t < \beta$ y $P_0 \neq 0$ en este intervalo. Dividir esta ecuación por $P_0(t)$ da como resultado

\begin{equation}\label{eq:edon1}
    L[y] = \frac{d^ny}{dt^n} + \dots + p_{n-1}(t)\frac{dy}{dt} + p_n(t)y = g(t)
\end{equation}

Donde $L[y]$ es un operador diferencial lineal\footnote{Un \ul{operador diferencial lineal} es una notación que ahorra escribir las ecuaciones diferenciales una y otra vez. Se define de esta manera: Para cualquier función $\phi$ que es $n$ diferenciable en un intervalo $I$, el operador diferencial $L$ es una ecuación:

\[
    L[\phi] = \phi'' +  p\phi + q\phi
\]

El valor de $L$ en un punto $t$ es

\[
    L[\phi](t) = \phi''(t) +  p(t)\phi(t) + q(t)\phi(t)
\]}

\begin{teo}
    Sean $p_0, p_1, \dots, p_n \in C(I \subset \R)$ y $x_0 \in I$. Si se tienen $y_0, y_0', \dots, y_0^{(n-1)} \in \R$ dados, entonces

    \[
        \sum_{i=0}^{n} p_i(x) y^{(n-i)} = f(x)
    \]

    \noindent admite una única solución en $I$ que satisface

    \[
        y(x_0) = y_0, \quad y'(x_0) = y_0', \quad \dots, \quad y^{(n-1)} = y_0^{(n-1)} \quad \footnotemark
    \]\footnotetext{Estas serán las $n$ conciciones iniciales de la ecuación.}
\end{teo}

Al igual que la demostración de la existencia y unicidad de la solución para EDOs de primer orden, esta demostración es bastante larga y requiere de muchos resultados previos.

\begin{defn}
    Diremos que \refeq{eq:edon1} es \ul{homogénea} si $f(x) \equiv 0$. En otro caso diremos que es \ul{no homogénea}.
\end{defn}

\begin{pro}
    Sean $y_1, \dots, y_n$ soluciones de \refeq{eq:edon1} con $f(x) \equiv 0$. Entonces

    \begin{equation}\label{eq:combinacionl}
        \sum_{i=1}^{n} c_i y_i(x) = y(x)
    \end{equation}

    \noindent es también solución de dicha EDO.
\end{pro}

\begin{proof}
    \textbf{(!!!!!!)} Leer la demostración de Daniel subida al classroom, hay un par de cosas medio chungas y que vale la pena preguntar.

    Cada $y_i(x)$ es solución de \refeq{eq:edon1}, con $f(x) = 0$. Entonces

    \begin{equation*}
        \begin{aligned}
            L[y_1] = \frac{d^ny_1}{dt^n} + \dots + p_{n-1}(t)\frac{dy_1}{dt} + p_n(t)y_1 &= 0 \\
            L[y_2] = \frac{d^ny_1}{dt^n} + \dots + p_{n-1}(t)\frac{dy_2}{dt} + p_n(t)y_2 &= 0 \\
            &\vdots \\
            L[y_n] = \frac{d^ny_n}{dt^n} + \dots + p_{n-1}(t)\frac{dy_n}{dt} + p_n(t)y_n &= 0
        \end{aligned}
    \end{equation*}

    Ahora, sea una combinación lineal de estas soluciones:

    \[
        \phi = c_1y_1 + c_2y_2 + \dots + c_ny_n
    \]

    Luego,

    \[
        L[y] = L[c_1y_1 + c_2y_2 + \dots c_ny_n] = c_1L[y_1] + c_2L[y_2] + \dots + c_nL[y_n]\footnotemark
    \]\footnotetext{Esto es así ya que los operadores diferenciales son lineales.}

    Como sabemos que $L[y_i] = 0$ (para $i=1,\dots,n$) entonces nos queda que

    \[
        L[y] = 0
    \]

    Y esto implica que $y$ es también solución.
\end{proof}

Ahora vale la pena preguntarse: ¿Toda solución de \refeq{eq:edon1} se puede escribir como una combinación lineal de $y_1, \dots, y_n$? A continuación veremos que seremos capaces de encontrar $c_1, \dots, c_n$ de tal forma que \refeq{eq:combinacionl} satisface las ecuaciones

\begin{equation*}
    \begin{aligned}
        c_1y_1(t_0) + \dots + c_ny_n(t_0) &= y_0 \\
        c_1y_1'(t_0) + \dots + c_ny_n'(t_0) &= y_0' \\
        &\vdots \\
        c_1y_1^{(n-1)}(t_0) + \dots + c_ny_n^{(n-1)}(t_0) &= y_0^{(n-1)} \\
    \end{aligned}
\end{equation*}

\begin{defn}
    El \ul{Wronskiano} de las funciones $\{y_1, \dots, y_n\} \subset C^{(n-1)}(\R)$ es el determinante

    \[
        W(y_1, \dots, y_n) =
        \begin{vmatrix}
            y_1  & y_2  & \dots & y_n \\
            y_1' & y_2' & \dots &' y_n \\
            \vdots&     &     \\
            y_1^{(n-1)} & y_2^{(n-1)} & \dots & y_n^{(n-1)} \\
        \end{vmatrix}
    \]
\end{defn}

\begin{teo}
    Sean $\{y_1, \dots, y_n\}$ soluciones de

    \begin{equation}\label{eq:homog}
        y^{(n)} + p_1(x)y^{(n-1)} + \dots + p_ny = 0
    \end{equation}

    \noindent en $I \subset \R$. Entonces $\{y_1, \dots, y_n\}$ es un conjunto L.I en $I$ sii $W(y_1, \dots, y_n) \neq 0$ para todo $x \in I$.
\end{teo}

\begin{teo}
    Si $\{y_1, \dots, y_n\}$ es un conjunto L.I de soluciones de \refeq{eq:homog}, entonces cualquier otra solución de \refeq{eq:homog} puede ser escrita como

    \[
        y(x) = c_1y_1(x) + \dots + c_ny_n(x)
    \]

    \noindent donde $c_1, \dots, c_n$ son constantes.
\end{teo}

\subsection{Solución de ecuación de orden 2 homogénea}

Si $p_0, p_1, p_2$ son constantes entonces

\[
    p_0y'' + p_1y' + p_2y = 0 \implies y'' + \tilde{p_1}y' + \tilde{p_2}y = 0
\]

\noindent y si $y(x) = e^{rx}$ es solución, entonces

\[
    p(r) = \LaTeXoverbrace{(r^2 + \tilde{p_1}r + \tilde{p_2})e^{rx}}^{\mathclap{\text{Ecuación característica de la EDO}}} = 0
\]

Varias consideraciones:

\begin{enumerate}
    \item Si $p(r) = 0$ posee dos raíces distintas y reales, $y(x) = c_1e^{r_1x} + c_2e^{r_2x}$.
    \item Si $p(r) = 0$ posee una raíz doble real, $y(x) = c_1e^{r_1x} + c_2xe^{r_2x}$.
    \item Si $p(r) = 0$ posee raíces complejas $r_1 = \alpha + i\beta$, $r_2 = \overline{r_1}$, entonces $y(x) = c_1e^{\alpha x}\cos(\beta x) + c_2e^{\alpha x}\sin(\beta x)$.
\end{enumerate}

\textbf{TAREA: Realizar el problema 4.1.20 del Boyce y realizar los problemas resueltos de la sección 3.1 y los problemas 1 al 8.}
\section{Clase 8}
\subsection{Resolución de Ecuaciones no homogéneas: Método de Variación de Parámetros}

Antes de pasar a explicar este método, hacen falta un par de teoremas\footnote{La demostración de estos dos teoremas puede encontrarse en el capítulo 3 del Boyce, y son necesarios para los resultados que se establecerán a continuación.} previos:

\begin{teo}
    Si $Y_1$, $Y_2$ son soluciones de la ecuación no homogénea \refeq{eq:edon1}, entonces la diferencia $Y_1 - Y_2$ es una solución a la ecuación homogénea asociada. Más aún, si $y_1$, $y_2$ conforman un conjunto fundamental de soluciones\footnote{Es decir, que $y_1(x)$ y $y_2(x)$ son soluciones a la ecuación y su Wronskiano es distinto de cero.} Entonces

    \[
        Y_1(x) - Y_2(x) = c_1y_1(x) + c_2y_2(x)
    \]

    \noindent con $c_1, c_2$ constantes.
\end{teo}

\begin{teo}
    La solución general de \ref{eq:edon1} puede escribirse en la forma

    \[
        y = \phi(x) = c_1y_1(x) + c_2y_2(x) + Y
    \]

    \noindent donde $y_1, y_2$ conforman un conjunto fundamental de soluciones de la ecuación homogénea asociada, $c_1, c_2$ son constantes, e $Y$ es una solución en específico de \ref{eq:edon1}.
\end{teo}

Recordemos que una ecuación diferencial no homogénea es aquella de la forma \refeq{eq:edon1}. La idea del método de Variación de Parámetros, es encontrar primero una solución a la ecuación homogénea asociada

\[
    y_c(x) = c_1y_1 + c_2y_2 = 0
\]

La idea básica es reemplazar las constantes $c_1$ y $c_2$ por funciones $u_1(x)$, $u_2(x)$ respectivamente, y hallar estas funciones de tal forma que la expresión

\[
    y = u_1y_1 + u_2y_2
\]

Es una solución de \ref{eq:edon1}.

\textbf{(!!!!) Toda esta charla que viene de $y'$ y $y''$ no la entendí muy bien, y tampoco entiendo muy bien qué es lo que se concluye. PREGUNTAR.}

Desarrollando esta idea, derivemos $y$ para hallar $u_1$ y $u_2$ en el proceso:

\begin{equation*}
    \begin{aligned}
        y' &= u_1'y_1 + u_2'y_2 + u_2y_2' + u_1y_1' \\
        y'' &= u_1''y_1 + u_1'y_1' + u_2''y_2 + u_2'y_2' + u_2'y_2' + u_2y_2'' + u_1'y_1' + u_1y_1''
    \end{aligned}
\end{equation*}

Como $y_1$, $y_2$ son soluciones de la ecuación homogénea asociada, Entonces

\[
    u_1'y_1 + u_2'y_2 = 0
\]

Y esto implica que

\begin{equation*}
    \begin{aligned}
        y' &= u_2y_2' + u_1y_1' \\
        y'' &=  u_1'y_1' + u_2'y_2' + u_2y_2'' + u_1y_1''
    \end{aligned}
\end{equation*}

Luego, gracias a los teoremas visto al inicio de la sección, sabemos que

\[
    f(x) = \LaTeXunderbrace{u_1(y_1'' + py_1' + qy_1) + u_2(y_2'' + py_2' + qy_2)}_{\text{Solución a la ecuación homogénea asociada}} + \LaTeXoverbrace{u_1'y_1' + u_2'y_2'}^{\mathclap{\text{Solución específica de la ecuación no homogénea}}}\footnotemark
\]\footnotetext{Como $y_1, y_2$ conforman un conjunto fundamental de soluciones, entonces también $y_1', y_2'$ también conforman uno.}

Como $W(y_1, y_2) \neq 0$, todo esto nos da como resultado el sistema

\begin{equation}
    \begin{cases*}
        u_1'y_1 + u_2'y_2 = 0 \\
        u_1'y_1' + u_2'y_2' = f(x)
    \end{cases*}
\end{equation}

De donde

\[
    u_1' = -\frac{y_2f(x)}{W(y_1, y_2)} \qquad u_2' = \frac{y_1f(x)}{W(y_1, y_2)}
\]

Todo esto nos quiere decir que para resolver EDO de este tipo, basta con seguir los siguientes pasos:

\begin{enumerate}
    \item Calcular $y_c$.
    \item Calcular $W(y_1, y_2)$.
    \item Aplicar $u_1', u_2'$ y calcular sus integrales.
\end{enumerate}

\end{document}