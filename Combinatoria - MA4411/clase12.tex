\subsection{Teorema min-max}

A continuación describiremos un método para aumentar el valor de un flujo dado suponiendo que el flujo no tiene el máximo valor posible. Este método nos llevará a probar el teorema min-max (mínimo corte, máximo flujo).

Tomemos como ejemplo nuevamente la red \ref{fig:red} con el flujo descrito más adelante. En el camino dirigido $s, a, t$ vemos que ni el arco $(s,a)$ ni $(a,t)$ tienen flujos que tomen valores máximos para sus capacidades, por lo que podemos incrementar el flujo para ambos arcos hasta que la capacidad de uno de ellos sea alcanzada. Definimos entonces $f_1(s,a) = 4$ y $f_1(a,t) = 3$ y ahora $(a,t)$ está saturado. Definimos ahora $f_1(x,y)$ para el resto de los arcos y obtendremos un nuevo flujo

\begin{tabular}{c|ccccccccc}
    $(x,y)$  & $(s,a)$ & $(s,b)$ & $(s,c)$ & $(a,d)$ & $(b,d)$ & $(c,d)$ & $(a,t)$ & $(c,t)$ & $(d,t)$ \\
    $f(x,y)$ & $4$ & $2$ & $3$ & $1$ & $2$ & $1$ & $3$ & $2$ & $4$
\end{tabular}

\noindent con valor $\val(f_1) = \val(f) + 1 = f$.

Pero esto se puede mejorar: Sea $G$ el grafo base del grafo dirigido $D$ de la red y consideremos el camino $s, a, d, c, t$  en $G$. Para la red, el arco $(c,d)$ va en sentido contrario, entonces tendríamos el siguiente camino junto a sus flujos y capacidades en la red:

\[
s \xlongrightarrow[5]{4} a \xlongrightarrow[6]{1} d \xlongleftarrow[7]{1} c \xlongrightarrow[5]{2} t
\]

Como el flujo en $(c,d)$ está en dirección contraria al resto, podemos reducir su flujo en 1 e incrementar el flujo en el resto de los arcos por la misma cantidad sin violar la regla de conservación. En este sentido, obtenemos un nuevo flujo $f_2$ tal que sus valores en el camino descrito son

\[
s \xlongrightarrow[5]{5} a \xlongrightarrow[6]{2} d \xlongleftarrow[7]{0} c \xlongrightarrow[5]{3} t
\]

\noindent y el resto de los arcos tienen el mismo flujo que en $f_1$.

En este flujo ya no podemos seguir incrementando debido a que $(s,a)$ está saturado y el flujo en $(c,d)$ no puede ser menor que cero. Ahora $\val(f_2) = \val(f_1) + 1 = 10$. Sea el corte $S = \{s,b\}$, $T=\{a,c,d,t\}$, este corte tiene capacidad $c(S,T)=10$. Entonces por el teorema \ref{teo:cotavalor}, $f_2$ es un flujo máximo (en el sentido que tiene el máximo valor posbile).

En general, dado un flujo $f$ en una red con digrafo $D$ y grafo base $G$, decimos que un camino en $G$

\[
s=x_1,x_2,\dots,x_{k-1},x_k=t
\]

\noindent es un \ul{camino que aumenta} $f$ si para $i \leq k-1$ se cumple que

\begin{gather*}
    (x_i, x_{i+1}) \in A(D) \implies f(x_i,x_{i+1}) < c(x_i,x_{i+1}) \qquad \text{arcos directos} \\
    (x_{i+1}, x_i) \in A(D) \implies f(x_i,x_{i+1}) > 0 \qquad \text{arcos inversos}
\end{gather*}

En otras palabras, los arcos directos no están usados a su máxima capacidad y los arcos indirectos cargan algún flujo en sentido contrario. Para estos caminos podemos aumentar el flujo de los arcos directos y reducir el de los indirectos por la misma cantidad sin violar la regla de conservación.

\begin{teo}\label{teo:minmax}
    El valor máximo de un flujo de $s$ a $t$ en una red es igual a la capacidad mínima de un corte que separa $s$ y $t$.
\end{teo}

\begin{proof}
    Primero, a un camino como el anterior en el que $x_k \neq t$ lo llamaremos \ul{camino que aumenta $f$ parcial}. Si hay un camino que aumenta $f$ se puede proceder de la siguiente forma: Llamemos $\alpha$ al mínimo entre las cantidades

    \begin{gather*}
        c(x_i, x_{i+1}) - f(x_i, x_{i+1}) \quad \text{si} \quad (x_i, x_{i+1}) \in A(D) \\
        f(x_{i+1}, x_i) \quad \text{si} \quad (x_{i+1}, x_i) \in A(D)
    \end{gather*}
    
    Sumemos $\alpha$ al flujo en los arcos directos y restemos $\alpha$ en los inversos. El flujo resultante $f*$ satisface $\val(f*) = \val(f) + \alpha$.
    
    Por otro lado, siendo $f$ un flujo con valor máximo, definimos $S = \{x \in V(D): \text{existe camino parcial que aumenta $f$ de $s$ a $x$}\}$, $T = V(D) - S$. Como $f$ tiene flujo máximo, entonces $t \in T$. Entonces $(S,T)$ es un corte que separa $s$ y $t$.
    
    Consideremos ahora $(x,y) \in A(D)$ tal que $x \in S$, $y \in T$. Entonces $f(x,y)=c(x,y)$, ya que $f$ es un flujo con valor máximo, entonces no podemos aumentar su flujo. De la misma manera consideremos $(u,v) \in A(D)$ tal que $u \in T$, $v \in S$. Entonces $f(u,v) = 0$. Ahora
    
    \[
    \val(f) = \sum_{\substack{x\in S \\ y \in T}} f(x,y) - \cancelto{0}{\sum_{\substack{u\in T \\ v \in S}} f(u,v)} = c(S,T)
    \]
    
    De esta forma $\val(f) = c(S,T)$.
    
    Ahora, sea $(S',T')$ otro corte que separa $s$ y $t$, entonces tenemos
    
    \[
    c(S',T') \geq \val(f) = c(S,T)
    \]
    
    \noindent por lo tanto $c(S,T)$ es la mínima capacidad.
\end{proof}