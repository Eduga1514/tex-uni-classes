\section{Fórmulas generadoras}

Hemos visto ya que los problemas de conteo pueden resolverse utilizando unas fórmulas muy sencillas. Por ejemplo, la cantidad de maneras de distribuir $n$ objetos indistinguibles en $m$ casillas distinguibles es

\[
\binom{m}{n}
\]

Este número está relacionado con un objeto algebráico: es el coeficiente de $x^n$ en el desarrollo de $(x+1)^m$. En este sentido podría decirse que $(x+1)^m$ \textit{genera} los números

\[
\binom{m}{0}, \quad \binom{m}{1}, \quad \dots, \quad \binom{m}{m}
\]

Esta idea sirve para estudiar muchísimos problemas de conteo y, más generalmente, para facilitar el cálculo de los términos de algunas sucesiones definidas recursivamente, por ejemplo, la sucesión de Fibonacci.

Revisemos el siguiente problema:

\begin{ejer}
    El profesor Severo diseñó el examen final de Matemáticas I de la siguiente manera: Una larga lista de preguntas está dividida en tres grupos según el tema. Los estudiantes deben responder 7 preguntas: no más de dos preguntas del grupo 1, al menos dos del grupo 2 y al menos tres preguntas del grupo 3. ¿De cuántas maneras se pueden elegir las siente preguntas?
\end{ejer}

\begin{proof}[Respuesta]
    Podemos hacer una tabla con todas las posibilidades ya que los números son pequeños:
    
    \begin{center}
        \begin{tabular}{ccc}
            \text{Grupo 1} & \text{Grupo 2} & \text{Grupo 3} \\ \toprule
            $0$ & $2$ & $5$ \\
            $0$ & $3$ & $4$ \\
            $0$ & $4$ & $3$ \\
            $1$ & $2$ & $4$ \\
            $1$ & $3$ & $3$ \\
            $2$ & $2$ & $3$
        \end{tabular}
    \end{center}
    
    ¿Qué se puede apreciar en esta tabla?: En cada fila hay tres números que suman 7. Es decir, soluciones de $m_1 + m_2 + m_3 = 7$, pero sujetas a las condiciones $0 \leq m_1 \leq 2$, $2 \leq m_2 \leq 4$ y $3 \leq m_3 \leq 5$. Analicemos el producto
    
    \[
    p(x) = (1 + x + x^2)(x^2 + x^3 + x^4)(x^3 + x^4 + x^5)
    \]
    
    Al desarrollar el coeficiente de este prodcuto, el coeficiente de $x^7$ es la solución buscada.
\end{proof}

\subsection{Sumas formales infinitas}

Si en lugar de utilizar polinomios, consideramos \textit{sumas formales infinitas}, podemos aprovechar algunas características algebráicas, por ejemplo, sabemos de los curso de cálculo y análisis que la serie geométrica congerve a $1/1-x$ cuando $a = 1$ y $r = x$, entonces

\[
\frac{1}{1-x} = 1 + x^2 + x^3 + x^4 + \dots \implies 1 = (1-x)(1 + x^2 + x^3 + x^4 + \dots)
\]

Si consideramos $A(x) = (1 + x^2 + x^3 + x^4 + \dots)$ entonces $1-x$ sería un inverso multiplicativo de $A(x)$. En este contexto, diremos que $(1-x)^{-1}$ \textit{genera} la sucesión $1, 1, 1, 1, \dots$ de los coeficientes de $A(x)$. En general,

\begin{defn}
    Una \ul{suma formal infinita} en la indeterminada $x$ tiene la forma
    
    \[
    \sumtoinfty{n=0}{a_nx^n}
    \]
    
    \noindent donde $\{a_n\}_{n=0}^{\infty}$ es una sucesión de números reales o complejos.
    
    Sean $\displaystyle \sumtoinfty{n=0}{a_nx^n}$, $\displaystyle \sumtoinfty{n=0}{b_nx^n}$ y $\alpha \in \R$ ó $\alpha \in \C$. Se define
    
    \begin{itemize}
        \item $\displaystyle \sumtoinfty{n=0}{a_nx^n} + \sumtoinfty{n=0}{b_nx^n} = \sumtoinfty{n=0}{(a_n + b_n)x^n}$.
        \item $\displaystyle \alpha \left( \sumtoinfty{n=0}{a_nx^n} \right) = \sumtoinfty{n=0}{\alpha_nx^n}$.
        \item $\displaystyle \left( \sumtoinfty{n=0}{a_nx^n} \right)\left( \sumtoinfty{n=0}{b_nx^n} \right) = \left( \sumtoinfty{n=0}{c_nx^n} \right)$.
    \end{itemize}
    
    \noindent donde $\displaystyle c_n = \left( \sum_{i+j=n} a_ib_j \right)$.
    
    También diremos que dos sumas formales infinitas son iguales sii sus coeficientes son iguales.
\end{defn}

Con la multiplicación y la adición definidas de esa forma, el conjunto de las sumas formales infinitas es un anillo conmutativo con identidad y lo denotaremos por $\mathbb{F}[[x]]$, donde $\mathbb{F}$ es un campo (puede ser $\R$ ó $\C$).

\begin{teo}
    Sea $\mathbb{F}$ un campo. Entonces la suma
    
    \[
    \sumtoinfty{n=0}{a_nx^n}
    \]
    
    \noindent tiene inverso multiplicativo en $\mathbb{F}[[x]]$ sii $a_0 \neq 0$.
\end{teo}

\begin{proof}
    Si la serie es invertible, entonces hay otra serie perteneciente a $\mathbb{F}[[x]]$ tal que es su inverso, entonces
    
    \begin{gather*}
        \left(\sumtoinfty{n=0}{a_nx^n}\right)\left(\sumtoinfty{n=0}{b_nx^n}\right) = 1 \\
        (a_0 + a_1x + a_2x^2 + \dots)(b_0 + b_1x + b_2x^2) = 1
    \end{gather*}
    
    Igualando los coeficientes, vemos que $a_i = b_i = 0$ para $i = 1, \dots$. Luego $a_0b_0 = 1$, entonces $a_0 = 1 \neq 0$.
    
    Ahora, supongamos que $a_0 \neq 0$. Para hallar el inverso de la suma, consideremos las expresiones que se obtienen al revisar sus coeficientes,
    
    \begin{gather*}
        a_0b_0 = 1 \\
        a_0b_1 + a_1b_0 = 0 \\
        a_0b_2 + a_1b_1 + a_2b_0 = 0 \\
        \vdots
    \end{gather*}
    
    Como $a_0 \neq 0$, entonces existe $a_0^{-1}$ y podemos determinar los $b_i$ de forma recursiva, luego
    
    \begin{gather*}
        b_0 = a_0^{-1} \\
        b_1 = a_0^{-1}(-a_1b_0) \\
        b_2 = a_0^{-1}(-a_1b_1 - a_2b_0) \\
        \vdots
    \end{gather*}
    
    De esta forma, cada $b_i$ está unequívocamente determinado por $b_0^{-1}$ y luego la suma formal infinita es invertible.
\end{proof}

\begin{ejem}
    \begin{itemize}
        \item $\displaystyle \frac{1}{1+x} = 1 - x + x^2 - x^3 + \dots$.
        \item $\displaystyle \frac{1}{1-\alpha x} = 1 - \alpha x + \alpha^2x^2 - \alpha^3x^3 + \dots$.
    \end{itemize}
\end{ejem}

Vemos que dado $m \in \N$, si definimos $(1-x)^{-m}$ como la $m$-ésima potencia de $(1-x)^{-1}$, nos queda

\[
(1-x)^{-m} = (1+x+x^2+\dots)\dots(1+x+x^2+\dots)
\]

\begin{pre}
    ¿Cuál es el coeficiente de $x^n$ en el desarrollo de este producto? Como vimos en el ejercicio planteado al principío del capítulo, es la cantidad de maneras de distribuir $n$ objetos indistinguibles en $m$ casillasm lo cual equivale a la cantidad de $n$-selecciones no ordenadas con repetición de un conjunto de $m$ objetos, es decir
    
    \[
    \binom{m+n-1}{n}
    \]
    
    En conclusión
    
    \[
    (1-x)^{-m} = \sumtoinfty{n=0}{\binom{m+n-1}{n}x^n}
    \]
    
    Por lo que
    
    \[
    (1+x)^{-m} = \sumtoinfty{n=0}{(-1)^n\binom{m+n-1}{n}x^n}
    \]
\end{pre}

Ahora, si definimos para $\alpha \in \R$ y $n \in \N$ el \textit{coeficiente binomial generalizado}

\[
\binom{\alpha}{n} = \frac{\alpha(\alpha - 1)(\alpha - 2)\dots(\alpha - n + 1)}{n!}
\]

\noindent entonces para $m > 0$ tendremos

\[
\binom{-m}{n} = \frac{-m(m-1)(-m-2)\dots(-m-n+1)}{n!} = (-1)^n\frac{m(m+1)(m+2)\dots(m+n-1)}{n!} = (-1)^n\binom{m+n-1}{n}
\]

Si además definimos $\displaystyle \binom{\alpha}{0} = 1$ nos queda

\[
(1+x)^k = \sumtoinfty{n=0}{\binom{k}{n}x^n} \quad \text{para todo $k \in \Z$}
\]