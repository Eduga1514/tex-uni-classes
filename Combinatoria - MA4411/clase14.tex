\subsection{Recurrencia lineal}

Estudiemos el siguiente problema:

\begin{pro}
    Supongamos que la sucesión $\{a_n\}_{n=0}^{\infty}$ está definida por
    
    \[
    a_0 = 0, \quad a_1 = 1, \quad a_{n+2} - 5a_{n+1} + 6a_n = 0
    \]
    
    ¿Existirá una fórmula \textit{eficiente} para $\{a_n\}_{n=0}^{\infty}$? Es decir, una fórmula que depende únicamente de $n$.
\end{pro}

\begin{proof}[Solución]
    Digamos que $f(x)$ es dicha fórmula. Entonces
    
    \begin{align*}
        f(x) &= a_0 + a_1x + a_2x^2 + \dots \\
             &= x + 5x^2 + (5a_2 - 6a_1)x^3 + \dots \\
             &= x + 5(a_1x^2 + a_2x^3 + \dots) - 6(a_0x^2 + a_1x^3 + \dots)
             &= x + 5xf(x) - 6x^2f(x)
    \end{align*}
    
    Por lo que
    
    \begin{align*}
        f(x) &= \frac{x}{1 - 5x + 6x^2} = \frac{1}{1-3x} - \frac{1}{1-2x} \\
             &= \sumtoinfty{n=0}{3^nx^n} - \sumtoinfty{n=0}{2^nx^n}
    \end{align*}
    
    De esta forma $a_n = 3^n - 2^n$. Y hemos encontrado una fórmula eficiente para una sucesión dada por una \textit{relación de recurrencia lineal} (RRL).
\end{proof}

\begin{defn}
    Dado un entero positivo $k$, una \ul{relación de recurrencia lineal homogénea} (RRLH) es una sucesión $\{a_n\}_{n=0}^{\infty}$ definida en la forma
    
    \begin{gather*}
        a_0 = c_0, \quad a_1 = c_1, \quad \dots \quad a_{k-1} = c_{k-1} \\
        a_{n+k} + \alpha_1a_{n+k-1} + \dots + \alpha_ka_{n} = 0, \quad \text{donde $n \geq 0$}
    \end{gather*}
    
    \noindent donde las $c_i$'s y las $\alpha_j$'s son constantes conocidas.
\end{defn}

A raíz del problema anterior, se sugiere analizar el producto

\[
(1 + \alpha_1 x + \alpha_2 x^2 + \dots + \alpha_n x^n)f(x)
\]

Lo que equivale a 

\[
(1 + \alpha_1 x + \alpha_2 x^2 + \dots + \alpha_n x^n)(a_0 + a_1x + a_2x^2 + \dots)
\]

Para $n \geq 0$, el coeficiente de $x^{n+k}$ viene siendo

\[
a_{n+k} + \alpha_1a_{n+k+1} + \dots + \alpha_ka_n = 0
\]

Es decir, que el producto que estamos estudiando es un polinomio de grado a lo sumo $k-1$, digamos $q(x)$. En particular

\[
q(x) = c_0 + (c_1 + \alpha_1c_0)x + \dots + (c_{k-1} + \alpha_1c_{k-2} + \dots + c_0\alpha_{k-1})x^{k-1}
\]

Por lo tanto, $f(x)$ nos queda como

\begin{equation}\label{eq:f(x)}
    f(x) = \frac{q(x)}{1 + \alpha_1x + \dots + \alpha_kx^k}
\end{equation}

\noindent y esto se puede descomponer como la suma de fracciones simples. De forma que para hallar una fórmula generadora de la sucesión dada, basta analizar las raíces del denominador.

En la sección 12.2 del Biggs, se muestra que cuando $k = 2$, la forma que toma la solución de una RRLH depende de que sean distintas las raíces $\alpha$ y $\beta$ de la ecuación auxiliar

\[
t^2 + a_1 t + a_2 = 0
\]

Si $\alpha \neq \beta$. La solución tiene forma $u_n = A\alpha^n + B\beta^n$. Si $\alpha = \beta$, la solución tiene la forma $u_n = (Cn + D)\alpha^n$.

Para el caso general, definamos la \textit{ecuación auxiliar} de una RRLH como

\[
t^k + a_1 t^{k-1} + \dots a_k = 0
\]

\noindent y la forma de su solución dependerá del comportamiento de sus raíces.

Al trabajar en el campo de los complejos, la ecuación auxiliar tendrá $k$ raíces. Sin embargo, estas raíces no necesariamente han de ser distintas, y supondremos que los valores distintos son $\lambda_1, \dots, \lambda_s$, apareciendo con multiplicidad $m_1, \dots, m_s$ respectivamente. Es decir que la ecuación auxiliar la podemos reescribir como

\[
(t- \lambda_1)^{m_1}\dots(t- \lambda_s)^{m_s} = 0
\]

\noindent donde $m_1 + \dots + m_s = k$.

El denominador de la ecuación \ref{eq:f(x)} se puede obtener dividiendo la ecuación auxiliar por $t^k$ y posteriormente sustituyendo $t$ por $1/x$. Entonces, la factorización de la ecuación auxiliar queda como

\[
\frac{(1- \lambda_1)^{m_1}\dots(1- \lambda_s)^{m_s}}{x^k} = 0 \implies (1- \lambda_1)^{m_1}\dots(1- \lambda_s)^{m_s} = 0
\]

De esta forma, tendremos que $f(x)$ es la suma de expresiones

\[
\frac{\gamma_{i1}}{1 - \lambda_ix} + \frac{\gamma_{i2}}{(1 - \lambda_ix)^2} + \dots + \frac{\gamma_{im_i}}{(1 - \lambda_ix)^{m_i}}
\]

\noindent con $i = 1, 2, \dots, s$.

Para hallar el coeficiente de $x^n$ en el desarrollo de cada una de estas expresiones, utilizaremos la fórmula del binomio generalizado. Luego

\[
\left[ \gamma_{i1} \binom{1+n-1}{n} + \gamma_{i2} \binom{2+n-1}{n} + \dots + \gamma_{im_i} \binom{m+n-1}{n} \right]\lambda_i^n
\]

Lo que equivale a

\[
\left[ \gamma_{i1} \binom{n}{0} + \gamma_{i2} \binom{n+1}{1} + \dots + \gamma_{im_i} \binom{m_i + n - 1}{m_i - 1} \right] \lambda_i^n
\]

Este resultado se puede resumir en el siguiente teorema:

\begin{teo}
    Dada una RRLH, si $\lambda_1, \dots, \lambda_s$ son las raíces de la ecuación auxiliar con multiplicidades $m_1, \dots, m_s$, entonces
    
    \[
    a_n = p_1(n)\lambda_1^n + p_2(n)\lambda_2^n + \dots + p_s(n)\lambda_s^n
    \]
    
    \noindent donde cada $p_i(n)$ es un polinomio en $n$ con grado a lo sumo $m_i - 1$.
\end{teo}

\begin{defn}
    Dado un entero positivo $k$, una \ul{relación de recurrencia lineal no homogénea} (RRLNH) es una sucesión $\{a_n\}_{n=0}^{\infty}$ definida de la forma
    
    \begin{gather*}
        a_0 = c_0, a_1 = c_0, \dots, a_{k-1} = c_{k-1} \\
        a_{n+k} + \alpha_1a_{n+k-1} + \dots + \alpha_ka_{n} = g(n), \quad \text{donde $n \geq 0$}
    \end{gather*}
    
    \noindent donde las $c_i$'s y las $\alpha_j$'s son constantes conocidas y $g(x) \neq 0(x)$.
\end{defn}