\section{Grafos y conteo}
\stepcounter{sec}

Algunas de las preguntas que ahora son relevantes y que contestaremos en esta sección son las siguientes:

\begin{itemize}
    \item ¿Cómo se relaciona la cantidad de lados de un grafo con la cantidad de vértices?
    \item ¿Cuántos subgrafos con características dadas, tiene un grafo?
    \item ¿Cuántas clases de isomorfismo tiene un grafo dado?
\end{itemize}

\begin{defn}
    Antes de hacer los análisis pertinentes, vale la pena recordar algunos conceptos y notación e introduciremos algunos más:
    
    \begin{enumerate}
        \item El \ul{grado} de $v \in V(G)$ es la cantidad de lados de $G$ incidentes con $v$ (si $v$ tiene bucles, estos contribuyen en $2$ al grado). Lo denotaremos por $d_G(v)$ ó $d(v)$.
        \item $\Delta(G)$ es el máximo de los grados en $G$.
        \item $\delta(G)$ es el mínimo de los grados en $G$.
        \item Si $\Delta(G) = \delta(G) = k$, decimos que $G$ es \ul{$k$-regular}.
        \item La \ul{vecindad} de $v \in V(G)$ es el conjunto $N_G(v)$ de los vecinos de $v$ en $G$.
        \item El \ul{orden} de $G$ es $n(G) = |V(G)|$.
        \item El \ul{tamaño} de $G$ es $e(G) = |E(G)|$.
        \item Dado $n \in \N$, denotamos $[n] = \{1, 2, \dots, n\}$.
    \end{enumerate}
\end{defn}

\subsection{Conteo y biyecciones}
\stepcounter{subsec}

Introduciremos esta subsección con un resultado bastante relevante: Si queremos calcular la suma de los grados de los vértices de $G$, cada lado se cuenta dos veces. Hemos demostrado el siguiente resultado:

\begin{teo}[Primer Teorema de la Teoría de Grafos]\label{teo:primer}
    Si $G$ es un grafo, entonces
    
    \[
    \sum_{v \in V(G)} d(v) = 2e(G)
    \]
\end{teo}

De este teorema se pueden sacar varios corolarios:

\begin{cor}
    En un grafo $G$, el promedio de los grados es
    
    \[
    \frac{2e(G)}{n(G)}
    \]
    
    Por lo tanto,
    
    \[
    \delta(G) \leq \frac{2e(G)}{n(G)} \leq \Delta(G)
    \]
\end{cor}

\begin{cor}
    $G$ tiene una cantidad par de vértices con grado impar.
\end{cor}

\begin{cor}
    Si $G$ es $k$-regular, entonces $e(G) = kn(G)/2$.
\end{cor}

\begin{cor}
    Si $n(G)$ es impar y $G$ es $k$-regular, entonces $k$ no puede ser impar.
\end{cor}

\begin{defn}
    El \ul{cubo $k$-dimensional} o \ul{hipercubo} $Q_k$, es el grafo simple cuyos vértices son las $k$-tuplas con entradas en $\{0,1\}$ y cuyos lados son las parejas de $k$-tuplas que difieren en \textbf{exactamente} una posición.
\end{defn}

\begin{figure}
    \centering
    \begin{tikzpicture}
        \SetGraphUnit{2}
        \SetVertexNoLabel
        \grCycle[rotation=45, prefix=a, RA=1]{4}
        \grCycle[x=3, y=1.5, rotation=45, prefix=b, RA=1]{4}
        \AssignVertexLabel{a}{000,100,110,010}
        \AssignVertexLabel{b}{001,101,111,011}
        \SetUpEdge[style={dashed}]
        \Edge(a0)(b0)
        \Edge(a1)(b1)
        \Edge(a2)(b2)
        \Edge(a3)(b3)
    \end{tikzpicture}
    \caption{Acá se puede apreciar una representación de $Q_3$}
    \label{fig:hipercubo}
\end{figure}

\begin{prob}[Estructura de los hipercubos]
    Con respecto a los hipercubos surjen muchísimas propiedades interesantes:
    
    \begin{itemize}
        \item Podemos establecer la \textit{paridad} de un vértice de $Q_k$ la siguiente manera: Si tiene una cantidad par de $1$s es par, y de lo contrario es impar. Cada lado de $Q_k$ incide en un vértice par y otro impar. De esta forma, los vértices pares son adyacentes únicamente a los impares, y viceversa. Por lo tanto ambos conjuntos de vértices forman un conjunto independiente y tenemos como resultado que $Q_k$ es bipartito.
        \item Cada posición en las $k$-tuplas se puede elegir de dos maneras, por lo tanto $n(Q_k) = 2^k$. Además, los vecinos de cada vértice se pueden obtener al cambiar una de las $k$ posiciones de la tupla, por lo tanto $Q_k$ es $k$-regular, y se tiene que $e(Q_k) = k2^{k-1}$.
        \item Vemos en la figura \ref{fig:hipercubo} que los lados no punteados son dos subgrafos de $Q_3$ isomorfos a $Q_2$, formados al mantener fijas la última coordenada de las tuplas. ¿De qué manera podemos generalizar este resultado?: Podemos formar un subcubo $j$-dimensional al fijar $k-j$ coordenadas y hacer que los valores en las $j$ sean escogidos a partir de una de las $2^j$ $j$-tuplas restantes. El subgrafo inducido por este conjunto de vértices es isomorfo a $Q_j$. Como hay $\binom{k}{j}$ maneras para escoger las $j$ coordenadas a variar, y $2^{k-j}$ maneras de escojer las coordenadas fijas y sus valores, en total tenemos $\binom{k}{j}2^{k-j}$ subcubos. Así, para cada $j \leq k$, $Q_k$ tiene $\binom{k}{j}2^{k-j}$ subgrafos isomorfos a $Q_j$.
    \end{itemize}
    
    % Maldita sea como me excitan los hipercubos
    Ahora, ¿cómo podemos construir un hipercubo $Q_k$?: Agregar $0$ a las tuplas de una copia de $Q_{k-1}$, agregar $1$ a las tuplas de otra copia de $Q_{k-1}$, y unir los vértices cuyas primeras $k-1$ coordenadas sean iguales. El resultado será $Q_k$.
\end{prob}

Vimos que un hipercubo es un grafo regular y bipartito, a continuación estableceremos una observación fundamental sobre esos grafos.

\begin{teo}
    Si $k > 0$, entonces un grafo $k$-regular y bipartito tiene la misma cantidad de vértices en cada partición.
\end{teo}

\begin{proof}
    Sea $G$ un grafo $k$-regular y bipartito. Supongamos que $\{A, B\}$ es la bipartición. Entonces cada lado de $G$ tiene un extremo en $A$, por lo tanto $e(G) = k|A|$. También cada lado de $G$ tiene un extremo en $B$, por lo tanto $e(G) = k|B|$. Esto implica que $|A| = |B|$.
\end{proof}

\break

Otra técnica de conteo para conjuntos involucra establecer una biyección desde el conjunto a otro de un tamaño conocido. El siguiente problema utiliza este enfoque.

\begin{prob}
    ¿Cuántos $6$-ciclos tiene el grafo de Petersen?
\end{prob}

