\section{Clase 2}
\subsection{Separación de Variables}

Recordemos que la ecuación de calor que intentamos resolver es la siguiente:

\[
    \begin{cases*}
        \displaystyle\frac{\partial u}{\partial t} = \kappa \frac{\partial^2 u}{\partial x^2} & \\
        u(0, t) = u(L, t) = 0 & \textbf{Condiciones de Frontera} \\
        u(x, 0) = f(x) & \textbf{Condición Inicial}
    \end{cases*}    
\]

donde $x \in (0, L)$ y $t > 0$.

El método de \ul{Separación de Variables} consiste en suponer que podemos reescribir $u(x,t)$ como el producto $u(x,t) = \phi(x)G(t)$. Esto va a implicar que 

\[
    u(x,t) = \phi(x)G(t) \implies \begin{cases*}
                               u_t = \phi G_t \\
                               u_{xx} = \phi_{xx}G
                           \end{cases*}
\]

Como $\frac{\partial u}{\partial t} = \kappa \frac{\partial^2 u}{\partial x^2}$ entonces sustituyendo nos queda

\[
    \phi G_t = \kappa \phi_{xx}G \implies \displaystyle \frac{G_t}{G} = \kappa\frac{\phi_{xx}}{\phi}
\]

El lado izquierdo solo depende de $t$ y el derecho de $x$. Luego, $\exists \lambda$ constante tal que

\[
    \frac{1}{G} \frac{\partial G}{\partial t} = \frac{\kappa}{\phi}\frac{\partial^2 \phi}{\partial x^2} = -\lambda \quad \footnotemark
\]\footnotetext{$\lambda$ es negativo porque queremos hacer énfasis en que la temperatura decae a lo largo que pasa el tiempo.}

De acá generaremos dos ecuaciones:

\begin{equation}\label{ec:separables1}
    \begin{cases*}
        \displaystyle\frac{\partial^2 \phi}{\partial x^2} = -\lambda \phi \\
        \phi(0) = \phi(L) = 0 \quad \footnotemark
    \end{cases*}
\end{equation}\footnotetext{Recordemos que las condiciones de frontera son $u(0, t) = u(L, t) = 0$.}

\begin{equation}\label{ec:separables2}
    \displaystyle\frac{\partial G}{\partial t} = -\lambda \kappa G
\end{equation}

Resolvamos en primer lugar la ecuación \ref{ec:separables2}: Como $G$ depende solo de $t$, y $\phi$ solo de $x$. $u$ depende de ambas y hemos supuesto que $u = \phi G$. Luego, aplicando integración a ambos lados de la ecuación nos queda

\[
    \begin{split}
        \frac{\partial G}{\partial t} = -\lambda \kappa G &\implies \frac{\partial G}{G} = - \lambda \kappa t + \beta \implies \ln G = -\lambda \kappa t + \beta \\
        &\implies G(t) = c e^{-\lambda \kappa t}
    \end{split}
\]

Ahora examinemos \ref{ec:separables1}. Esta es una ecuación diferencial ordinaria de segundo orden. Un candidato para resolver esta ecuación es $\phi = e^{rx}$, entonces nos queda que:

\[
    \frac{\partial^2 \phi}{\partial x^2} = -\lambda \phi \implies r^2e^{rx} = -\lambda e^{rx} \implies r^2 + 1 = 0 \quad \footnotemark
\]\footnotetext{Este es el polinomio característico asociado a la ecuación diferencial ordinaria de segundo orden.}

En el caso $\lambda > 0$, entonces

\[
    \phi(x) = \pm e^{\sqrt{\lambda ix}}
\]

Por la fórmula de Euler, esto es

\[
    \phi(x) = c_1 \cos(\sqrt{\lambda} x) + c_2 \sin(\sqrt{\lambda} x) \quad c_1, c_2 \in \R
\]

Como $\phi(0) = \phi(L) = 0$, entonces al sustituir en la ecuación anterior

\[
    \begin{split}
        &\phi(0) = 0 \implies c_1 = 0 \\
        &\phi(L) = 0 \implies c_2\sin(\sqrt{\lambda} L) = 0 \implies \sin(\sqrt{\lambda} L) = 0 \implies \sqrt{\lambda} L = n\pi
    \end{split}
\]

\noindent
con $n \in \Z$.

Así, $\lambda = (n\pi/L)^2$ y queda que

\[
    \displaystyle\phi(x) = c_2\sin\left( \frac{n \pi x}{L} \right)
\]

Si $\lambda = 0$ entonces

\[
    \phi(x) = 0
\]

Ahora, gracias al principio de la superposición, concluímos que cada solución va a tener la forma

\[
    \LaTeXunderbrace{B_n\sin\left( \frac{n\pi x}{L} \right)}_{\phi(x)} \cdot \LaTeXunderbrace{\vphantom{\left( \frac{n\pi x}{L} \right)}e^{-\kappa \left( \frac{n\pi}{L} \right)^2 }}_{G(t)}
\]

Finalmente, tendremos que

\[
    u(x,t) = \sum_{n = 1}^{\infty} B_n\sin\left( \frac{n\pi x}{L} \right) e^{-\kappa \left( \frac{n\pi}{L} \right)^2 t}
\]

\noindent
como $u(x,0) = f(x)$ entonces

\[
    f(x) = \LaTeXunderbrace{\sum_{n = 1}^{\infty} B_n\sin\left( \frac{n\pi x}{L} \right)}_{\text{Serie de Fourier de senos de $f$}}
\]

\subsection{Sistema Trigonométrico}

Queremos ahora calcular estos $B_n$. Primero consideremos en la familia de funciones

\[
    \left\{ \sin\left( \frac{n\pi x}{L} \right) \right\}^{\infty}_{n = 1}
\]

Diremos que esta familia es  \ul{ortogonal}.\footnote{Ortogonal quiere decir que

\[
    0 = <f, g>_{L^2} = \int_0^L f(t)g(t)dt
\]} en el espacio $L^2[0, L].$\footnote{Son aquellas funciones tales que

\[
    \left\{ f: [0, L] \rightarrow \R \Bigg| \int_0^L |f(t)|^2 dt < \infty \right\}
\]

Y además

\[
    ||f||_2 = \left( \int_0^1 |f|^2 dy \right)^{1/2}
\]} Pasemos a demostrar que en efecto, esta familia es ortogonal:

%TODO: Copiar la demostración de ortogonalidad de esa familia de senos

Como la familia es ortogonal, entonces si partimos de

\[
    f(x) = \sum_{n = 1}^{\infty} B_n\sin\left( \frac{n\pi x}{L} \right)
\]

Integrando ambos lados,

%TODO: Copiar el desarrollo de f(x)

Así, $B_k = \frac{2}{L} \int_0^L f(x) \sin(\frac{k\pi x}{L}) dx$. Y estos son los coeficientes de Fourier de $f$.

%TODO: Terminar de copiar esta clase









