\documentclass{tufte-handout}
\usepackage[utf8]{inputenc}
\usepackage{parskip} % Varios paquetes para símbolos y fuentes
\usepackage{amssymb}
\usepackage{amsmath}
\usepackage{amsthm}
\usepackage{fdsymbol}
\usepackage{mathtools}
\usepackage{amsthm}
\usepackage{tikz, pgfplots}
\usetikzlibrary{decorations.pathreplacing, calligraphy, arrows.meta}
\usepackage{titling} % Para estilizar el título
\usepackage{scrextend} % Añade márgenes para hacer bloques de texto
\usepackage{enumitem} % Para enumerar sin sangría
\usepackage{graphicx} % Maneja las imágenes
\usepackage{float}
\usepackage{wrapfig, subcaption}
\usepackage{lastpage}
\usepackage{fancyhdr} % Para hacer encabezados y pie de página más estilizados
\usepackage{color} % Para usar colores en el texto
\usepackage{soul} % Para subrayar con colores
\usepackage{soulutf8}
\usepackage{cancel} % Para tachar expresiones
\usepackage{titling} % Cambia los parámetros del título

\newcounter{fakecnt}[section]
\def\thefakecnt{\arabic{section}}

\theoremstyle{plain}
\newtheorem{teo}{Teorema}
\newtheorem{cor}[teo]{Corolario}
\newtheorem{lem}[teo]{Lema}
\newtheorem{pro}[teo]{Proposición}
\newtheorem{pre}{Pregunta}

\theoremstyle{definition}
\newtheorem{defn}{Definición}
\newtheorem{ejem}{Ejemplo}
\newtheorem{ejer}{Ejercicio}
\newtheorem{nota}{Notación}
\newtheorem{aco}{Acotación}

\counterwithin{teo}{fakecnt}
\counterwithin{pre}{fakecnt}
\counterwithin{defn}{fakecnt}
\counterwithin{ejem}{fakecnt}
\counterwithin{ejer}{fakecnt}
\counterwithin{nota}{fakecnt}
\counterwithin{aco}{fakecnt}

\renewcommand{\contentsname}{Contenido}
\renewcommand*{\proofname}{Demostración}
\renewcommand{\figurename}{Fig.}
\newcommand{\marginfootnote}[1]{\footnotemark\footnotetext{#1}}

\newcommand{\R}{\mathbb{R}}
\newcommand{\N}{\mathbb{N}}
\newcommand{\F}{\mathcal{F}}
\newcommand{\Tau}{\mathcal{T}}

% Establece el directorio para las imagenes
\graphicspath{ {img/} }

% Establecemos cómo será el encabezado y el pie de página
\fancyhf{}
\pagestyle{fancy}
\fancyhf{}
\fancyhead[L]{Notas de Topología}
\fancyhead[C]{Eduardo José Gavazut Pinto}
\fancyhead[R]{13-10524}
\fancyfoot[L]{}
\fancyfoot[R]{}
\fancyfoot[C]{\thepage\ de \pageref{LastPage}}
\renewcommand{\headrulewidth}{2pt} 
\renewcommand{\footrulewidth}{2pt}

\geometry{
	left=13mm, % left margin
	textwidth=130mm, % main text block
	marginparsep=8mm, % gutter between main text block and margin notes
	marginparwidth=55mm % width of margin notes
}

% Establece el subrayado de color rojo
\definecolor{ferrari}{rgb}{1,0.17,0}
\setulcolor{ferrari}

% Reduce el espacio entre el título y el header
\setlength{\droptitle}{-5.5em}
\renewcommand\maketitlehookc{\vspace{-3ex}}

% Define el espaciado entre párrafos
\setlength{\parskip}{1.5em}

% Definimos nuestro título
\pretitle{\begin{flushleft}\LARGE\sffamily}
\title{
Notas de Topología
}
\posttitle{\par\end{flushleft}\vskip 0.5em}
\preauthor{\begin{flushleft}\large\scshape}
\author{
Eduardo Gavazut \\
Carnet: 13-10524}
\postauthor{\par\end{flushleft}}
\predate{\begin{flushleft}\large\scshape}
\date{}
\postdate{\par\end{flushleft}}

% Aquí empieza el documento
\begin{document}

\maketitle
\thispagestyle{fancy}

\tableofcontents
\break
\section{Introducción}

\begin{aco}
    Estas notas son hechas por mi persona solamente como apoyo para leer el libro \textit{Topology Without Tears} de Sydney Morris, y no como apoyo para algún curso en particular, es por eso que el carácter de estas notas es \ul{enteramente personal}. El libro mencionado hace un excelente trabajo autoreferenciandose, pero obviamente las cosas se olvidan y a veces resulta menos laborioso revisar las cosas más relevantes directamente desde un documento, además transcribir las definiciones, teoremas y ejercicios relevantes es un excelente método para fijar ideas (al menos para mí).
\end{aco}

\section{Capítulo 1: Espacios topológicos}
\setcounter{section}{1}

\subsection{Primeras definiciones}

\begin{defn}\label{defn:1.1.1}
    Sea $X$ un conjunto no-vacío. Un conjunto $\Tau$ de subconjuntos de $X$ se dice que es una \ul{topología sobre $X$} si:
    
    \begin{enumerate}
        \item $X$ y $\emptyset$ pertenecen a $\Tau$.
        \item La unión de cualquier número (finito o infinito) de conjuntos de $\Tau$ pertenece a $\Tau$.
        \item La intersección de dos conjuntos de $\Tau$ pertenece a $\Tau$.
    \end{enumerate}
    
    El par $(X, \Tau)$ se llama un \ul{espacio topológico}.
\end{defn}

\begin{ejem}
    Sea $\N$ el conjunto de los números naturales (enteros positivos), y sea $\Tau$ tal que $\N \in \Tau$, $\emptyset \in \Tau$ y además todos los subconjuntos finitos de $\N$ pertenecen a $\Tau$. Entonces $\Tau$ \textbf{no} es una topología sobre $\N$ ya que la unión infinita
    
    \[
    \{2\} \cup \{3\} \cup \dots \cup \{n\} \cup \dots = \{2, 3, \dots, n, \dots\}
    \]
    
    \noindent no están en $\Tau$, es decir que $\Tau$ no cumple la propiedad 2 de \ref{defn:1.1.1}.
\end{ejem}

\begin{defn}
    Sea $X$ un subconjunto no-vacío y $\Tau$ la familia de \textbf{todos} los subconjuntos de $X$. Entonces $\Tau$ es la \ul{topología discreta} sobre $X$. El espacio $(X, \Tau)$ se denomina \ul{espacio discreto}.
\end{defn}

\begin{defn}
    Sea $X$ un conjunto no-vacío y $\Tau = \{X, \emptyset\}$. Entonces $\Tau$ se denomina la \ul{topología indiscreta} y $(X, \Tau)$ se denomina \ul{espacio indiscreto}.
\end{defn}

\begin{pro}
    Si $(X, \Tau)$ es un espacio topológico tal que para cada $x \in X$, el conjunto $\{x\}$ está en $\Tau$, entonces $\Tau$ es la topología discreta.
\end{pro}

\begin{proof}
    Primero, todo conjunto es la unión de todos sus subconjuntos "singletons"; entonces si $S$ es un subconjunto cualquiera de $X$ tenemos que
    
    \[
    S = \bigcup_{x \in S} \{x\}
    \]
    
    Por otro lado, como $\{x\} \in \Tau$ para todo $x \in X$, entonces por el segundo inciso de \ref{defn:1.1.1} tenemos que $S \in \Tau$. Como $S$ es un subconjunto arbitrario de $X$, entonces $\Tau$ es la topología discreta.
\end{proof}

\begin{ejer}\label{ejer:1.1.1}
    Sea $(X, \Tau)$ un espacio topológico. Verificar que la intersección finita de miembros de $\Tau$ es un miembro de $\Tau$.
\end{ejer}

\begin{proof}[Solución]
    Consideremos la intersección finita
    
    \[
    I = \bigcap_{i = 1}^n U_i, \quad \text{con $U_i \in \Tau$ para $i = 1, \dots, n$}
    \]
    
    Si $n=1$, por \ref{defn:1.1.1} tenemos que $I \in \Tau$. Supongamos ahora que $I \in \Tau$ para $n = k$. Si $n = k+1$, tenemos que
    
    \[
    I = \bigcap_{i = 1}^{k+1} U_i = \left( \bigcap_{i=1}^k U_i \right) \cap U_{k+1}
    \]
    
    Como $I \in \Tau$ si $n=k$, entonces $\bigcap_{i=1}^k U_i \in \Tau$. También tenemos que $U_{k+1} \in \Tau$. Entonces por \ref{defn:1.1.1} tenemos que
    
    \[
    I = \left( \bigcap_{i=1}^k U_i \right) \cap U_{k+1} \in \Tau
    \]
    
    Y queda demostrado.
\end{proof}

\begin{ejer}
    Sea $\N$ el conjunto de todos los enteros positivos. Demostrar que cada una de las siguientes colecciones sobre $\N$ es una topología:
    
    \begin{enumerate}
        \item $T_1$ está formado por $\N$, $\emptyset$ y todo conjunto de la forma $\{1, 2, \dots, n\}$ para cualquier $n$ positivo (esta se llama la \ul{topología de segmento inicial}).
        
        \item $T_2$ está formado por $\N$, $\emptyset$ y todo conjunto de la forma $\{n, n+1, \dots\}$ para cualquier $n$ positivo (esta se llama la \ul{topología de segmento final}).
    \end{enumerate}
\end{ejer}

\begin{proof}[Solución]

    Tanto para $\Tau_1$ como para $\Tau_2$, la primera condición se cumple, entonces queda verificar que se cumplen la segunda y tercera.
    
    \begin{enumerate}
        \item Consideremos la familia
        
        \[
        \F = \{B_j \in \Tau_1 : j \in J\}, \quad \text{con $J$ una familia de subíndices cualquiera}
        \]
        
        Entonces, como cada $B_j \in \Tau_1$, tenemos que para cada $j \in J$
        
        \[
        B_j = \{1, 2, \dots, n_j\}
        \]
        
        Como $n_j \in \N$, entonces podemos escoger el más grande de todos ellos. Sea entonces $n = \max_{j \in J} (n_j)$. Luego
        
        \[
        \bigcup \F = \bigcup_{\substack{j \in J \\ B_j \in \Tau_1}} B_j = \{1, 2, \dots, n_{j_1}\} \cup \{1, 2, \dots, n_{j_2}\} \cup \dots = \{1, 2, \dots, n\}
        \]
        
        Luego, $\{1, 2, \dots, n\} \in \Tau_1$ y queda demostrado que la unión arbitraria está en $\Tau_1$.
        
        De forma análoga, sean $B_1, B_2 \in \Tau_1$. Supongamos que $n = \min (n_1, n_2)$. Entonces
        
        \[
        B_1 \cap B_2 = \{1, 2, \dots, n_1\} \cap \{1, 2, \dots, n_2\} = \{1, 2, \dots, n\}
        \]
        
        Luego, $\{1, 2, \dots, n\} \in \Tau_1$, y queda demostrado que la intersección de dos elementos de $\Tau_1$ está en $\Tau_1$.
        
        Por lo tanto, $\Tau_1$ es una topología.
        
        \item Consideremos la familia
        
        \[
        \F = \{B_j \in \Tau_2 : j \in J\}, \quad \text{con $J$ una familia de subíndices cualquiera}
        \]
        
        Entonces, como cada $B_j \in \Tau_2$, tenemos que para cada $j \in J$
        
        \[
        B_j = \{n_{j}, n_{j}+1, \dots \}
        \]
        
        Como $n_j \in \N$, entonces podemos escoger el más pequeño de todos ellos. Sea entonces $n = \min_{j \in J} (n_j)$. Luego
        
        \[
        \bigcup \F = \bigcup_{\substack{j \in J \\ B_j \in \Tau_2}} B_j = \{n_{j_1}, n_{j_1}+1, \dots \} \cup \{n_{j_2}, n_{j_2}+1, \dots \} \cup \dots = \{n, n+1, \dots\}
        \]
        
        Luego, $\{n, n+1, \dots\} \in \Tau_2$ y queda demostrado que la unión arbitraria está en $\Tau_2$.
        
        De forma análoga, sean $B_1, B_2 \in \Tau_2$. Supongamos que $n = \max (n_1, n_2)$. Entonces
        
        \[
        B_1 \cap B_2 = \{n_{1}, n_{1}+1, \dots \} \cap \{n_{2}, n_{2}+1, \dots \} = \{n, n+1, \dots\}
        \]
        
        Luego, $\{n, n+1, \dots\} \in \Tau_2$, y queda demostrado que la intersección de dos elementos de $\Tau_2$ está en $\Tau_2$.
        
        Por lo tanto, $\Tau_2$ es una topología.
    \end{enumerate}
    
    Así, queda demostrado.
\end{proof}

\begin{ejer}
    Sea $X$ un conjunto infinito y $\Tau$ una topología sobre $X$. Si todo conjunto infinito de $X$ está en $\Tau$, demostrar que $\Tau$ es la topología discreta.
\end{ejer}

\begin{proof}[Solución]
    Primero, $\Tau$ es una topología, entonces toda intersección finita de conjuntos de $\Tau$ está en $\Tau$. También tenemos que todo subconjunto infinito de $X$ está en $\Tau$. Luego la familia
    
    \[
    \F = \{A \subset X : S < \infty, S \subset A, |A| = \infty\}
    \]
    
    Está contenida en $\Tau$. Luego podemos escoger $A_1, A_2 \in \F$ tales que
    
    \[
    A_1 \cap A_2 = S
    \]
    
    Y esto implica que $S \in \Tau$, y para todo subconjunto $S$ finito y no-vacío contenido en $X$. Así, $\Tau$ está conformado por todos los conjuntos finitos e infinitos de $X$, entonces $\Tau$ es la topología discreta.
\end{proof}

\subsection{Conjuntos abiertos, Conjuntos cerrados, Conjuntos clopen}

\begin{defn}
    Sea $(X, \Tau)$ un espacio topológico cualquier. Entonces los miembros de $\Tau$ se dice que son \ul{conjuntos abiertos}.
\end{defn}

\begin{pro}
    Si $(X, \Tau)$ es cualquier espacio topológico, entonces
    
    \begin{enumerate}
        \item $X$ y $\emptyset$ son conjuntos abiertos.
        \item La unión arbitraria (finita o infinita) de cualquier número de abiertos es abiertos.
        \item La intersección de cualquier número finito de abiertos es abiertos.
    \end{enumerate}
\end{pro}

\begin{proof}
    La demostración a esta proposición se desprende de la definición \ref{defn:1.1.1} y el ejercicio \ref{ejer:1.1.1}.
\end{proof}

\begin{ejem}
    Sea $\N$ el conjunto de los enteros positivos y sea $\Tau$ formado por $\emptyset$ y todo subconjunto $S \subseteq \N$ tal que el complemento de $S$, $\N - S$ es un conjunto finito. Se verifica fácilmente por \ref{defn:1.1.1} que $\Tau$ es una topología sobre $\N$. Ahora, para cada número natural $n$, definamos el conjunto $S_n$ como sigue:
    
    \[
    S_n = \{1\} \cup \{n+1\} \cup \dots = \{1\} \cup \bigcup_{m = n+1}^{\infty} \{m\}
    \]
    
    Como el complemento de cada $S_n$ es finito, entonces $S_n$ es abierto en la topología $\Tau$. Sin embargo
    
    \[
    \bigcap_{n = 1}^{\infty} S_n = \{1\}
    \]
    
    Y como el complemento de $\{1\}$ no es ni $\N$ ni un conjunto finito, entonces $\{1\}$ no es abierto. De esta manera, vemos que la intersección infinita de conjuntos abiertos $S_n$ no es abierta.
\end{ejem}

\begin{defn}\label{defn:1.1.5}
    Sea $(X, \Tau)$ un espacio topológico. Un subconjunto $S \subseteq X$ se dice que es un \ul{conjunto cerrado} en $(X, \Tau)$ si su complemento en $X$, $X - S$, es abierto en $(X, \Tau)$.
\end{defn}

\begin{pro}
    Si $(X, \Tau)$ es cualquier espacio topológico, entonces
    
    \begin{enumerate}
        \item $\emptyset$ y $X$ son conjuntos cerrados.
        \item La intersección arbitraria (finita o infinita) de conjuntos cerrados, es cerrada.
        \item La unión de cualquier número finito de conjuntos cerrados es cerrada.
    \end{enumerate}
\end{pro}

\begin{proof}
    Para demostrar esta proposición, usaremos la definición \ref{defn:1.1.1}:
    
    \begin{enumerate}
        \item El complemento de $X$ es $\emptyset$ y viceversa, por lo que ambos son abiertos por definición.
        \item Consideremos la siguiente familia arbitraria de conjuntos cerrados:
        
        \[
        \F = \{B_j \in \Tau_1 : j \in J\}, \quad \text{con $J$ una familia de subíndices cualquiera}
        \]
        
        Entonces para cada elemento $B_j$ de dicha familia, $X - B_j \in \Tau$ (con $j \in J$), por la definición \ref{defn:1.1.5}. Por lo tanto, por la definición \ref{defn:1.1.1}
        
        \[
        \bigcup_{\substack{j \in J \\ B_j \in \Tau}} (X - B_j) \in \Tau
        \]
        
        Pero por las leyes de De Morgan, nos queda que
        
        \[
        \bigcup_{\substack{j \in J \\ B_j \in \Tau}} (X - B_j) = X - \bigcap_{\substack{j \in J \\ B_j \in \Tau}} B_j
        \]
        
        Como esta es una intersección arbitraria de conjuntos cerrados, entonces queda demostrado.
        \item Sean $S_1, \dots, S_n$ conjuntos cerrados. Queremos probar que $S_1 \cup \dots \cup S_n$ es un conjunto cerrado. Por la definición \ref{defn:1.1.5}, basta probar que $X - (S_1 \cup \dots \cup S_n)$ es abierto.
        
        Como $S_1, \dots, S_n$ son conjuntos cerrados, sus complementos $X - S_1$, \dots, $X - S_n$ son abiertos, pero por De Morgan, tenemos que
        
        \[
        X - (S_1 \cup \dots \cup S_n) = (X - S_1) \cap \dots \cap (X - S_n)
        \]
        
        Esta es una intersección finita de conjuntos abiertos, por lo tanto es un conjunto abierto. De esta forma, $S_1 \cup \dots \cup S_n$ es un conjunto cerrado, y queda demostrado el inciso.
    \end{enumerate}
\end{proof}

\end{document}